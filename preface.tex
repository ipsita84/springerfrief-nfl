%%%%%%%%%%%%%%%%%%%%%%preface.tex%%%%%%%%%%%%%%%%%%%%%%%%%%%%%%%%%%%%%%%%%
% sample preface
%
% Use this file as a template for your own input.
%
%%%%%%%%%%%%%%%%%%%%%%%% Springer %%%%%%%%%%%%%%%%%%%%%%%%%%

\preface


Condensed matter physics is the study of the complex behaviour of a large number of interacting particles such that their collective behaviour gives rise to emergent properties. We will discuss some interesting quantum condensed matter systems wherein intriguing emergent phenomena arise due to strong coupling, showing up as non-Fermi liquids (NFLs). Reviewing the Landau paradigm of Fermi liquid theory, we will understand the distinctive origin of NFLs in contrast with  normal metals (or Fermi liquids). We will outline a framework to extract the low-energy physics of such systems in a controlled approximation, using the tool of dimensional regularization. We will demonstrate how this technique can be used to extract the low-energy properties of NFL phases in various strongly correlated systems. In the entire book, we will focus on critical-Fermi-surface states, where there is a well-defined Fermi surface, but no well-defines quasiparticles, because the latter get destroyed as a consequence of strong interactions between the Fermi surface and some emergent massless boson(s). We will focus on the quantum critical points where the dynamics of an order parameter couples with the itinerant fermionic degrees of freedom to cause an NFL behaviour. The intended audience is scientists working in the area of strongly-correlated systems. This monograph will also enable researchers, who are new to quantum field theory, to carry out explicitly the basic steps involving dimensional regularization, renormalization group (RG) flows, minimal subtraction scheme, and so on.

\vspace{\baselineskip}
\begin{flushright}\noindent
Dadri,\hfill {\it Ipsita  Mandal}\\
February 2026\hfill {\it Ipsita  Mandal}\\
\end{flushright}



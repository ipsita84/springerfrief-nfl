\chapter{Non-Fermi liquids by critical cavity photons at the onset of superradiance}
\label{chapcavity}

\abstract{We investigate the emergence of a non-Fermi liquid (NFL) at a quantum 
critical point which marks the onset of superradiance in a cavity quantum 
electrodynamics set-up. While the finite cavity, bounded by reflecting 
mirrors, endows the photons with an effective mass, this mass vanishes 
precisely at the continuous phase transition, rendering the photons 
critical. The matter sector comprises fermions on a honeycomb lattice 
near half-filling, with low-energy excitations described by doped Dirac 
cones at two sets of inequivalent valleys. This choice is motivated by 
the presence of a fermion-boson interaction vertex that generates Landau 
damping of the critical bosons and thereby drives the system into an 
NFL phase. To construct the quantum effective action, we identify the 
hot-spots on the generically anisotropic, trigonally-warped Fermi 
surfaces (FSs) --- defined as the sets of points with mutually parallel 
or antiparallel tangent vectors. The cavity photons play the role of 
charge-density-wave (CDW) order parameters, connecting pairs of 
hot-spots on the FS within a single valley. With these ingredients in 
hand, we characterize the resulting NFL phases using dimensional 
regularization and renormalization group (RG) flow equations. Our 
analysis reveals the existence of stable NFL fixed points in the 
low-energy limit, as determined by examining the RG flows along the 
direction of the CDW coupling constant.}


\section{Introduction}

Cavity-confined photons have emerged as a powerful and versatile tool 
for engineering strong electron-electron interactions via light-matter 
coupling \cite{jaksch, kollath, eckstein, farokh, demler, diehl, ahana}, 
motivating us to explore the possibility of non-Fermi liquid (NFL) 
phases arising in cavity quantum electrodynamics (QED) set-ups involving 
two-dimensional (2d) crystalline lattices \cite{roux, piazza_qed, zhang, 
piazza_superrad, bhaseen, basko, polini, pascal, peng}. The rapidly 
advancing field of cavity QED offers the prospect of realizing strong 
coupling between cavity photons and fermionic matter, from 2d layered 
heterostructures \cite{sentef} to synthetic ultracold atomic arrays 
\cite{piazza_qed, roux, zhang}. Although the finite spatial extent of 
the cavity endows the photonic modes with an effective mass, the system 
can be driven toward a superradiant phase in which the cavity photons 
become massless at a continuous phase transition \cite{piazza_superrad, 
bhaseen, basko, polini, pascal, peng}. In the superradiant phase, the 
ground state hosts a macroscopic occupation of coherent photons --- 
effectively, a photon condensate --- coupling the lattice atoms to a 
spatially uniform, single-mode electromagnetic field. Accessing this 
transition demands an exceptionally strong atom-field coupling. 
Superradiant phase transitions have been realized experimentally across 
a range of platforms, including optically pumped gases \cite{ref22cav}, 
photoexcited semiconducting quantum dots \cite{ref23cav, ref24cav}, and 
pumped ultracold gases confined in ultrahigh-finesse optical cavities 
\cite{ref25cav}. It is important to note, however, that all these 
experiments rely on an external drive, and a true equilibrium 
realization of the superradiant transition remains elusive. We proceed 
nonetheless with the expectation that such an equilibrium set-up will be 
achieved in the near future.

%%%%%%%%%%%%%%%%%%%%%%%%%%%%%%%%%%%%%%%
\section{Model}
\label{secmodel-cav}

Our starting point is the non-interacting tight-binding Hamiltonian on 
the honeycomb lattice, the paradigmatic model for graphene, which has 
been studied extensively in the literature \cite{neto}. At half-filling, 
Dirac cones emerge at two types of inequivalent valleys, labelled as $K_\varsigma$ 
(with $\varsigma = \pm$), and the two types of Dirac points are exchanged under 
time-reversal symmetry. We focus on the low-energy dynamics of fermions 
with momenta close to $K_\varsigma$, which is captured by taking the 
continuum limit of the tight-binding model. Upon doping away from the 
Dirac points, the system evolves into a Fermi liquid with a nonparabolic 
dispersion and singly-connected convex Fermi surfaces (FSs). At sufficiently 
low doping, the dispersion remains linear and isotropic, placing the 
system in the regime of the so-called Dirac Fermi liquid (DFL) 
\cite{maslov-dfl}.

The band structure is obtained by expanding the momentum $\mathbf{P}$ 
around the valley $K_\varsigma$ as $\mathbf{P} = \mathbf{K}_\varsigma 
+ \mathbf{p}$, where $|\mathbf{p}|/|\mathbf{K}_\varsigma| \ll 1$. At 
leading order in this small parameter, one recovers the linear dispersion 
characteristic of the DFL. Retaining corrections up to order 
$(|\mathbf{p}|/|\mathbf{K}_\varsigma|)^2$ introduces trigonal-warping 
of the bands \cite{ando1998, dresselhaus, neto} in the vicinity of 
$K_\varsigma$. Working in a diagonal electron-hole basis and suppressing 
the spin degrees of freedom, the noninteracting Hamiltonian takes the 
form \cite{maslov-dfl},
\begin{align}
\label{eqheh}
H_0 = \sum_{\varsigma,\mathbf{p}}
\left[
\left(\epsilon_{\varsigma,\mathbf{p},+} - \mu_c\right)
\alpha^\dagger_{\varsigma,\mathbf{p}}\,
\alpha^{\phantom{\dagger}}_{\varsigma,\mathbf{p}}
+ \left(\epsilon_{\varsigma,\mathbf{p},-} - \mu_c\right)
\beta^\dagger_{\varsigma,\mathbf{p}}\,
\beta^{\phantom{\dagger}}_{\varsigma,\mathbf{p}}
\right],
\end{align}
where $\mu_c$ is the chemical potential. The dispersion, accounting for trigonal-warping corrections, reads
\begin{align}
\label{twd}
& \epsilon_{\varsigma,\mathbf{p},\lambda}(\theta_p) =
\epsilon^\mathrm{D}_{\mathbf{p},\lambda}
+ \epsilon^\mathrm{TW}_{\varsigma,\mathbf{p},\lambda}(\theta_p)\,, \quad
\epsilon^\mathrm{D}_{\mathbf{p},\lambda} = 
\lambda\,v_\mathrm{D}\,|\mathbf{p}|\,, \nonumber \\
& \epsilon^\mathrm{TW}_{\varsigma,\mathbf{p},\lambda}(\theta_p) = 
\lambda\,\varsigma\,\frac{v_\mathrm{D}\,|\mathbf{p}|^2}{2\tilde{\rho}}
\cos(3\theta_p)\,, \quad
\theta_p = \arctan\!\left(\frac{p_y}{p_x}\right),
\end{align}
%%%%%%%%%%%%%%%%%
where $\alpha^\dagger_{\varsigma,\mathbf{p}}$ 
($\beta^\dagger_{\varsigma,\mathbf{p}}$) creates an electron (hole) 
in the conduction (valence) band in the vicinity of valley 
$K_\varsigma$. The parameter $\lambda = \pm$ labels the band, 
$v_\mathrm{D}$ denotes the Fermi velocity at the Dirac point, and 
$\tilde{\rho}$ encodes the strength of trigonal-warping, scaling 
inversely with the nearest-neighbor hopping distance. Our coordinate 
choice ensures that the Dirac and warping terms enter with the same 
sign for the valley $K_+$. Provided the warping remains sufficiently 
weak, both FSs retain global convexity, as depicted in 
Fig.~\ref{fig_fscavity}.



%%%%%%%%%%%%%%%%%%%%%%%%%%%%%
\subsection{Location of hot-spots paired by CDW ordering}

The electron-photon coupling is implemented through the Peierls 
substitution, which ties the fermion-boson vertex to the gradient of 
the Hamiltonian with respect to the vector potential \cite{peng}. For 
the Hamiltonian in Eq.~\eqref{eqheh}, the current operator is extracted 
via the standard formula $\mathbf{j} = -\frac{\delta}{\delta\mathbf{A}}
H_0(\mathbf{p} - e\mathbf{A})$, where $\mathbf{A}$ denotes the vector 
potential. The completely filled valence band contributes no net 
current, so we focus exclusively on $\lambda = +$, yielding the 
intraband current
\begin{align}
\label{eqcur}
\mathbf{j} = \sum_{\varsigma,\mathbf{p}} \mathbf{v}_{\varsigma,\mathbf{p}}\,
\alpha_{\varsigma,\mathbf{p}}^\dagger\,\alpha^{\phantom{\dagger}}_{\varsigma,\mathbf{p}}\,, 
\quad
\mathbf{v}_{\varsigma,\mathbf{p}} = 
\boldsymbol{\nabla}_{\mathbf{k}}\epsilon_{\varsigma,\mathbf{p},+}\,.
\end{align}
For brevity, we drop the spin label throughout what follows. An 
important consequence is that the polarization function --- which 
controls the possible appearance of a CDW gap --- arises from the 
current-current correlator and therefore involves no intervalley 
scattering: CDW bosons can only link hot-spots belonging to the same 
valley.

The derivation of the polarization function is a standard exercise 
well covered in existing literature and serves to establish the 
conditions under which a CDW gap opens. The mechanism relies on an 
enhancement of the low-energy phase space when electrons scatter 
between hot-spots whose tangent vectors are either parallel or 
antiparallel. The setup bears a close resemblance to the 
current-current correlator analysis of Ref.~\cite{maslov-dfl}, which 
treated doped monolayer graphene with an effective four-fermion 
interaction generated by the bare Coulomb potential, also incorporating 
trigonal-warping. The crucial difference is that our model contains no 
four-fermion interaction, thereby eliminating intervalley scattering 
processes altogether. Had such a term been present, CDW instabilities 
with wavevector $\boldsymbol{\mathcal{Q}}$ linking time-reversed 
patches on the two distinct FSs centered at $K_+$ and $K_-$ 
would have been accessible, as explored in Ref.~\cite{levitov}. We 
further remark that the supplemental calculations in Ref.~\cite{peng} 
assume circular FSs at each valley, with hot-spot pairs 
sharing the same curvature. In our case, the FS are 
noncircular due to trigonal-warping, and the hot-spots connected by 
$\boldsymbol{\mathcal{Q}}$ typically possess different curvatures.
 
%%%%%%%%%%%%%%%%%%
\begin{figure}[t!]
\centering
\includegraphics[width = 0.65 \textwidth]{patches2} 
\caption{Schematic illustration of the two trigonally-warped Fermi surfaces 
situated at the neighboring valleys, $K_+$ and $K_-$. The hot-spot 
pairs on each surface, linked by the incommensurate wavevector 
$\boldsymbol{Q}$ (depicted by the black arrow), are emphasised. In 
the $K_+$ ($K_-$) valley, the fermionic degrees of freedom localised 
near the right and left hot-spots are denoted $\psi_+^{(1)}$ 
($\psi_-^{(2)}$) and $\psi_+^{(2)}$ ($\psi_-^{(1)}$), respectively. 
These fermions interact with the critical cavity-photon modes, which carry
momenta centered about $\boldsymbol{\mathcal{Q}}$.}\label{fig_fscavity}
\end{figure}
%%%%%%%%%%%%%%%%%%%%%%%%%%%%%%%

%%%%%%%%%%%%%%%%%%%%%%%%%%%%%%%%%%%
\subsection{Patch theory using time-reversed partners}


To construct a patch theory, we introduce two-component spinors formed 
from time-reversed partners \cite{Lee-Dalid, ips-uv-ir1}, which in this particular setting
case effectively correspond to fermionic operators on the conjugate 
valleys $K_+$ and $K_-$. Each valley supports three distinct pairs of 
hot-spots, with the members of each pair connected by a CDW boson. 
These three pairs are related to one another by $2\pi/3$ rotations, 
reflecting the threefold rotational symmetry inherent to the honeycomb 
lattice. The momenta of the three CDW bosons, centered at wavevectors 
$\boldsymbol{\mathcal{Q}}$, are similarly related by $2\pi/3$ rotations. 
Since the cavity photons carry no self-interactions, the theory 
factorizes into three decoupled sectors, each containing two pairs of 
hot-spots with parallel or antiparallel tangent vectors, coupled to a 
single CDW boson species. It therefore suffices to study a single such 
sector, to which we restrict our attention throughout the remainder of 
this chapter. Working in patch coordinates \cite{Lee-Dalid, ips-uv-ir1, 
ips-u1, ips-rafael, ips-fflo, ips-2kf}, the effective action for the 
patches located at $\theta_p = 0$ and $\theta_p = \pi$ reads 
\cite{ips-cavity},
%%%%%%%%%%%%%%%%%%%%%%
\begin{align}
 S &= \sum_{\substack{ \varsigma = \pm \\ 
n = 1, 2 }  }
 \int_{k} \left \lbrace \psi^{(n)}_{\varsigma } (k) \right  \rbrace^{\dagger}
  \left [ -i\, k_0 -(-1)^n
  \, \varsigma \,v_F^{(n)} \,k_1 
  + \kappa^{(n)} \, k_2^2 \right ] \psi_{\varsigma }^{(n)} (k) 
\nn &	\quad +
%%%%%%%%%%%%
\frac{1} {2}	\int_k \phi(k) \left( k_0^2 + k_1^2 + k_2^2 \right) \phi(-k) \nn
%%%%%%%
& \quad + \frac{e} {2}  \int_k  \int_q \,
\Big[ \,\phi(q) \left \lbrace 
\psi^{(1)}_{+ } (k+q) \right \rbrace^{\dagger} 
\psi^{(2)}_{+ }(k) 
+ \phi (-q) \left \lbrace 
\psi^{(2)}_{+ } (k-q) \right \rbrace^{\dagger} 
\psi^{(1)}_{+ }(k) 
\,\Big] \nn
%%%%%%%
& \quad + \frac{ e} {2}  \int_k  \int_q \,
\Big[ \,\phi (q) \left \lbrace 
\psi^{(2)}_{- } (k+q) \right \rbrace^{\dagger} 
\psi^{(1)}_{- }(k) 
+ \phi (-q) \left \lbrace 
\psi^{(1)}_{- } (k-q) \right \rbrace^{\dagger} 
\psi^{(2)}_{- }(k) 
\,\Big]\,.
\label{eqs0}
\end{align}
%%%%%%%%%%%%%%%%%%%%%%%%%%%%%%%%
The three-vector $k = (k_0, \boldsymbol{k})$ comprises the Matsubara 
frequency, $k_0$, and the spatial momentum, 
$\boldsymbol{k} = (k_1, k_2) \equiv (k_\perp, k_\parallel)$. The integrals are denoted via the 
shorthand $\int_k \equiv \int dk_0\,d^d\boldsymbol{k}/(2\pi)^{d+1}$ 
and $d = 2$ spatial dimensions. In the neighborhood of $K_\varsigma$, 
the low-energy fermionic excitations localized near the two hot-spots 
are labeled $\psi_\varsigma^{(1)}(k)$ and $\psi_\varsigma^{(2)}(k)$, 
as illustrated in Fig.~\ref{fig_fscavity}. The CDW bosonic field 
$\phi(k)$, arising from the cavity photons, carries frequency $k_0$ 
and momentum $\boldsymbol{\mathcal{Q}} + \boldsymbol{k}$. Precisely 
at the superradiant quantum critical point, these cavity bosons lose 
their mass, as manifest in the purely quadratic bosonic sector of the 
action. We rescale fermionic momenta such that for the fields 
$\psi_+^{(1)}(k)$ and $\psi_-^{(1)}(k)$, both the Fermi velocity and 
curvature are set to unity: $v_F^{(1)} = \kappa^{(1)} = 1$. For the 
remaining hot-spots, we adopt the notation $v_F^{(2)} = \upsilon$ and 
$\kappa^{(2)} = \kappa$. The curvature $\kappa$ can become negative 
when the corresponding hot-spot is concave. Although the bare bosonic 
velocity differs in general from its fermionic counterpart, we normalize 
it to unity, as it does not enter the infrared effective theory: near 
the quantum critical point, the bosonic propagator is governed entirely 
by particle-hole fluctuations of the FS at low energies.

Before rescaling, the Fermi velocity at angular position $\theta_p$ 
reads $\upsilon_{F,\varsigma}(\theta_p) = v_D\left[1 + 
\varsigma\cos(3\theta_p)/\rho\right]$, with $\rho \simeq \tilde{\rho}\,
v_D/\mu_c$, and we write $\kappa_\varsigma(\theta_p)$ for half the 
curvature. Upon evaluation at $\theta_p = 0$ and $\theta_p = \pi$ 
[cf.\ Fig.~\ref{fig_fscavity}], these become $v_F^{(n)}$ and 
$\kappa^{(n)}$. Exploiting the freedom to rescale, we pass to the 
dimensionless ratios $\upsilon_{F,\varsigma}(\theta_p) \rightarrow 
\upsilon_{F,\varsigma}(\theta_p)/\upsilon_{F,+}(0)$ and 
$\kappa_\varsigma(\theta_p) \rightarrow \kappa_\varsigma(\theta_p)/
\kappa_+(0)$, as depicted in Fig.~\ref{figfs-cavity2} for the valley 
$K_+$. This choice renders $v_F^{(1)} \equiv \upsilon_{F,+}(0)$ and 
$\kappa^{(1)} \equiv \kappa_+(0)$ both equal to unity --- these 
correspond to the right-hand hot-spot of $K_+$ and the left-hand 
hot-spot of $K_-$ --- while the conjugate hot-spots carry 
$v_F^{(2)} \equiv \upsilon = \upsilon_{F,+}(\pi)$ and 
$\kappa^{(2)} \equiv \kappa = \kappa_+(\pi)$, subject to the global 
convexity requirement $1 - \kappa/\upsilon \geq 0$. Throughout, we 
work in parameter regimes where both FSs remain everywhere convex.

Within the patch-coordinate framework, essential for captuirng the correct energy-windows
of the fermions and bosons, the key scaling-dimension assignments are: 
$[\mathbf{K}] = 1$ and $[k_d] = 1/2$. Examining the kinetic terms in 
the action then reveals the engineering dimensions of
$[\psi^{(n)}_\varsigma] = [\phi_\pm] = -7/4$. Substituting these values
into the interaction vertex yields $[e] = 1/4$. This establishes $e$ 
as a relevant operator and presages the emergence of a NFL, in direct 
analogy with the systems analyzed in Refs.~\cite{Lee-Dalid, ips-uv-ir1, 
ips-u1, ips-2kf}. To gain analytical control, we invoke dimensional 
regularization, continuously varying the co-dimension of the FS as an 
auxiliary mathematical tool. This procedure identifies the upper critical 
dimension $d = d_c$, defined as the point at which the one-loop fermionic 
self-energy acquires a logarithmic divergence in the Wilsonian cutoff 
$\Lambda$, separating the NFL regime from marginal-Fermi-liquid behaviour.



%%%%%%%%%%%%%%%%%%
\begin{figure}[t!]
\centering
\includegraphics[width = 0.35 \textwidth]{figvk.png} 
\caption{Representative parameters characterising the FS at valley $K_+$, shown 
for a configuration in which all hot-spots exhibit positive curvature.}\label{figfs-cavity2}
\end{figure}
%%%%%%%%%%%%%%%%%%%%%%%%%%%%%%%



Ensuring that the theory remains analytic in momentum space (a 
requirement equivalent to preserving locality in position space), when the 
co-dimension is extended to generic values, necessitates the introduction 
of two-component spinors as follows \cite{Lee-Dalid, ips-uv-ir1,ips-cavity}
\begin{align}
\Psi_1^T(k) = \left[\psi_{+}^{(1)}(k) \quad 
\left\{\psi_{-}^{(1)}\right\}^\dagger(-k)\right]
\text{ and }
\Psi_2^T(k) = \left[\psi_{-}^{(2)}(k) \quad 
\left\{\psi_{+}^{(2)}\right\}^\dagger(-k)\right],
\end{align}
together with their conjugates,
\begin{align}
\bar{\Psi}_n \equiv \Psi_n^\dagger\,\gamma_0 \quad \text{for } 
n \in \{1, 2\}\,.
\end{align}
With these spinors in hand, we construct an effective action describing 
the patches of the one-dimensional FS near the hot-spots, now embedded 
in a $d$-dimensional momentum space \cite{Lee-Dalid, ips-uv-ir1, ips-u1, 
ips-2kf, ips-fflo}. The resulting low-energy effective action reads,
%%%%%%%%%%%%%%%%%%
\begin{align}
\label{eqs1}
S  &=   \sum_n  \int_k \bar \Psi_n  (k) \,i
\left [  \mathbf  \Gamma \cdot \mathbf  K  +  \gamma_{d-1} \, \delta_k^{(n)} \right  ] \Psi_n (k)  
 +
 \frac{1}{2}\int_k k_d^2 \,  \phi (k) \, \phi (-k) 
 \nn & \hspace{ 0.5 cm } 
- \left[ 
\frac{ i\, e \, \mu^{x/2} } {2} 
\int_{k} \int_q \phi (q) \,
 \bar{\Psi}_1 (k+q) \, {\Psi}_2(-k) 
+ \text{h.c.} \right ] ,
%%%%%
\nn x &=  \frac{ 5 } {2} - d \,, \quad 
\delta_k^{(1)} = k_{d-1} +  k_d^2 \,, \quad
\delta_k^{(2)} = \upsilon \,  k_{d-1} + \kappa\, k_d^2 \,.
\end{align}
%%%%%%%%%%%%%%%%%
The $(d-1)$-dimensional vector $\mathbf{K} \equiv (k_0, k_1, \ldots, 
k_{d-2})$ assembles the frequency and the $(d-2)$ momentum components 
introduced by extending the co-dimension. The original momentum 
components along the $k_1$- and $k_2$-directions are relabelled as 
$k_{d-1}$ and $k_d$, respectively. Adopting these notations implies that, in the
$d$-dimensional momentum space, $\{k_1, \ldots, k_{d-1}\}$ spans the 
$(d-1)$ directions perpendicular to the FS, while $k_d$ runs parallel 
to it. The symbol $\boldsymbol{\Gamma} \equiv (\gamma_0, \gamma_1, 
\ldots, \gamma_{d-2})$ denotes a $(d-1)$-dimensional vector of matrices 
whose contraction with $\mathbf{K}$ enters the fermionic kinetic term. 
For our purpose, it suffices to work with $2 \times 2$ Pauli 
matrices, taking $\gamma_0 = \sigma_y$ and $\gamma_{d-1} = \sigma_x$, 
since our ultimate goal is to analytically continue to $d = 2$, which is the 
physical spatial-dimensionality of the system. A floating mass scale, 
$\mu \sim \Lambda$, raised to the power $x/2$, is introduced to render 
the coupling constant $e$ dimensionless.

The kinetic parts of the action in Eq.~\eqref{eqs1} remain 
invariant under the following scaling transformations:
\begin{align}
& \mathbf  K  =  \frac{\mathbf  K'}{b} \,, 
\quad k_{ d-1 } =\frac{k_{ d-1 }'}{b} \,,
\quad k_d = \frac{k_d'}{\sqrt{b}} \,,
\nn &
\Psi_n (k)  =  b^{ \frac{2 d + 3} {4}}  \, \Psi'_n (k') \,, \quad 
\phi  (k) = b^{\frac{2 d + 3}{4}}  \, \phi' (k')\,.
 \end{align}
%%%%%%%%%%%%%%%%%%%%%% 
This follows from the fact that $[\mathbf{K}] = 1$ and $[k_d] = 1/2$, 
which are the defining characteristics of the patch coordinates. We note 
that the fermions near the two hot-spots, which interact strongly with 
the bosons, satisfy $|k_d| \gg k_{d-1}$, since any scattering event away 
from the FS incurs a large energy-cost. In the bosonic kinetic term, only 
the contribution proportional to $k_d^2$ is retained, as the part 
involving $(\mathbf{K}^2 + k_{d-1}^2)$ is irrelevant under the scaling-relations written above.

From Eq.~\eqref{eqs1}, the bare fermionic propagator is deduced to be
\begin{align}
\label{propf}
G_n(k) \equiv \left\langle \Psi_n(k)\,\bar{\Psi}_n(k) \right\rangle_0
= -\,i\,\frac{\boldsymbol{\Gamma} \cdot \mathbf{K} + 
\gamma_{d-1}\,\delta_k^{(n)}}{\mathbf{K}^2 + \delta_k^2}\,.
\end{align}
Similarly, the bare bosonic propagator is given by $D_{(0)}(k) = 1/k_d^2$.

The parameter $x$ establishes that the marginality of the coupling constant, 
$e$, occurs at the upper critical dimension, $d_c = 5/2$. It also tells us that $e$ 
behaves as a relevant (irrelevant) operator for $d < 5/2$ ($d > 5/2$). 
The strategy is to construct a controlled perturbative 
description of the interacting phase by working in $d = 5/2 - \epsilon$ 
dimensions, treating $\epsilon$ as an expansion parameter. The physical 
theory is then accessed by setting $\epsilon = 1/2$ 
upon the implementation of a systematic order-by-order expansion in $\epsilon$.

%%%%%%%%%%%%%%%%%%%%%%%%%%%%%
\section{One-loop Feynman diagrams}
\label{seconeloop}



We now present the results for all Feynman diagrams contributing at 
one-loop order, albeit with the loop-ordering for the fermionic self-energy and vertex corrections being dictated by considering the dressed bosonic propagator. This amounts to rearranging the perturbative expansion such that the one-loop bosonic self-energy
is already included at the `zero'-th order.


%%%%%%%%%%%%%%%%%%%%%%%%%%%%%%%%%%%
\subsection{One-loop boson self-energy}
\label{oneloopbos}

We begin by computing the one-loop bosonic self-energy, which starts from the expression,
%%%%%%%%%%%%%%%%
\begin{align}
\label{bosloop}
\Pi_1 (q) & = - \frac{
\left ( i \, e  \,\mu^{\frac {x} {2} } \right )^2} {2}
 \int_k \mbox{Tr}
\left[  G_1 (k+q)\, G_2 (k) \right] 
%%%%%%%%%%
\nn & = e^2 \, \mu^{x}  \int_k 
\frac{\mathbf  K \cdot (\mathbf  K +{\boldsymbol Q}) + \delta_k^{(1)} \, \delta_{k+q}^{(2)}
}
{ \left [\mathbf  K^2 + \delta_k^2 \right ] 
\left [(\mathbf  K +{\boldsymbol Q})^2 + \delta_{k+q}^2 \right ]}
%%%%%%%%%%%%%%%%%%%%%%%%%%%%%%%
\nn & = e^2 \, \mu^{x}  \int_k 
\frac{\mathbf  K \cdot (\mathbf  K +{\boldsymbol Q}) 
+ \upsilon \,\delta_k^{(1)} \, 
 \left \lbrace  k_{d-1} + q_{d-1} 
+  \frac{\kappa} {\upsilon} \left( k_d + q_d \right)^2 \right \rbrace
}
{ \upsilon^2
\left [ \mathbf  K^2 + \delta_k^2 \right ] 
\left [ \frac{(\mathbf  K +{\boldsymbol Q}) ^2 }
{\upsilon^2} + 
 \left \lbrace k_{d-1} + q_{d-1} 
+  \frac{\kappa} {\upsilon} 
\left( k_d + q_d \right)^2  \right \rbrace^2 \right ]} \,.
\end{align}
%%%%%%%%%%%%%%%%
Proceeding with the integration over $k_{d-1}$ then yields \cite{ips-cavity}
%%%%%%%%%%%%%%%%%%%%%%%
\begin{align}
\label{eqpi0}
\Pi_1 (q)  &  = \frac{e^2 \, \mu^{x}} {2} 
\int \frac{ d k_{d} \, d\mathbf  K}{(2 \, \pi)^d} 
\frac{ \left(  {\mathbf  K}
+ \frac{|\mathbf  K + {\boldsymbol Q}|} {\upsilon}  \right) 
\; \left[ \,\mathbf  K \cdot (\mathbf  K +{\boldsymbol Q}) 
+ {\mathbf  K}\;  \frac{ |\mathbf  K +{\boldsymbol Q}|} {\upsilon}   \right]  
}
{ 
\upsilon^2\, {\mathbf  K}\; \frac{ |\mathbf  K +{\boldsymbol Q}|} {\upsilon}
\, 
\left[ \left \lbrace 
 q_{d-1} 
+  \frac{\kappa} {\upsilon} \left( k_d + q_d \right)^2 
-k_d^2 \right \rbrace^2
 + \left ( {\mathbf  K}
+\frac{ |\mathbf  K +{\boldsymbol Q}|} {\upsilon} \right)^2
\right] }    \nn
%%%%%%%%%%%%%%%%%%%%%%
&  = \frac{e^2 \, \mu^{x}} {2} 
\int \frac{ d k_{d} \, d\mathbf  K}{(2 \, \pi)^d} 
\frac{ \left(  {\mathbf  K}
+ \frac{|\mathbf  K + {\boldsymbol Q}|} {\upsilon}  \right) 
\; \left[ \,\mathbf  K \cdot (\mathbf  K +{\boldsymbol Q}) 
+ {\mathbf  K}\;\frac{ |\mathbf  K +{\boldsymbol Q}|} {\upsilon} \,  \right]  
}
{ \upsilon \, {\mathbf  K}\; |\mathbf  K +{\boldsymbol Q}|\,
\left[ \Upsilon^2 (k,q) +  \left ( {\mathbf  K}
+\frac{ |\mathbf  K +{\boldsymbol Q}|} {\upsilon} \right)^2
\right ] }  \,,
%%%%%%%%%%%%%%%%%%%%%%
\end{align}
%%%%%%%%%%%%%%%%%%%%%%%%%%
where
\begin{align}
\Upsilon(q,k) =  \begin{cases}
\left (1 - \frac {\kappa} {\upsilon} \right)
\left [ k_d^2 - \frac {\upsilon \, e_q}
{  \upsilon - \kappa  } \right ]
%%%%%%%%%%%%%%%%%%% 
& \text{ for } \upsilon \neq \kappa \\
%%%%%%%%%%%%%%%
 \delta_q^{(1)} + 2\, k_d \, q_d 
& \text{ for } \upsilon = \kappa 
\end{cases} \,,
\end{align}
and
\begin{align}
e_q = \frac {\kappa \,  q_d^2} 
{\upsilon  - \kappa  }   \,
+ \,q_ {d-1} \,.
\end{align}
%%%%%%%%%%%%%%%%%%%%%%5

In the special case where $\upsilon = \kappa$, this reduces to
%%%%%%%%%%%%%%%%%%%%%%%
\begin{align}
\label{eqPi1}
\Pi_1 (q)  & = \frac{e^2 \, \mu^{x}} {4} 
\int  \frac{  \, d\mathbf  K}{(2 \, \pi)^d} 
\int_{-\infty }^\infty dk_d \,
\frac{ \left(  {\mathbf  K}
+ \frac{|\mathbf  K + {\boldsymbol Q}|} {\upsilon}  \right) 
\; \left[ \,\mathbf  K \cdot (\mathbf  K +{\boldsymbol Q}) 
+ {\mathbf  K}\; |\mathbf  K +{\boldsymbol Q}| \,  \right]  
}
{\sqrt {|q_d |}\; \upsilon \, {\mathbf  K}\; |\mathbf  K +{\boldsymbol Q}|\,
\left[   k_d^2 +  \left ( {\mathbf  K}
+\frac{ |\mathbf  K +{\boldsymbol Q}|} {\upsilon} \right)^2
\right ] }   
%%%%%%%%%%%%%
 \nn & = \frac{e^2 \, \mu^{x}} { 8 \, |q_d| \, \upsilon } 
\; I_1 (d, {\boldsymbol Q}) \, ,
\end{align}
%%%%%%%%%%%%%%%%%%%%%%%%%%
with
\begin{align}
\label{eqi1}
I_1 (d, {\boldsymbol Q}) = \int  \frac{  \, d^{d-1}\mathbf  K}{(2 \, \pi)^{d-1} } 
\left[ \frac{ \upsilon\,
 \mathbf  K \cdot (\mathbf  K +{\boldsymbol Q}) }
{ {\mathbf  K}\; |\mathbf  K +{\boldsymbol Q}|}
+ 1 \right ] . 
 \end{align}


When $\upsilon \neq \kappa$, we perform the change of variables, 
$u = k_d^2$, which yields a Jacobian factor $1/(2\sqrt{u}) = 1/(2|k_d|)$. 
Inspecting the denominator of the second factor in the integrand reveals 
that the dominant contribution is concentrated near $u \sim e_q$ in the 
regime $|\boldsymbol{Q}| \ll \kappa\,q_d^2$. Treating $e_q$ as positive 
and noting that the typical energy scales enforce $q_d \gg q_{d-1}$, we 
replace $|k_d|$ in the Jacobian by $\sqrt{\upsilon\,e_q/(\upsilon - \kappa)}$. 
The integral then evaluates to \cite{ips-cavity}
 \begin{align}
\label{api}
&\Pi_1 (q) = - \,\beta_d \, e^2 \, \mu^{x}  \,
 \frac{  |{\boldsymbol Q}|^{ d - 1} }
{ f(q)} \,, \quad
f(q) = \begin{cases}
\sqrt {\upsilon\,( \upsilon-\kappa) } 
 \,\sqrt e_q\, \Theta(e_q)
& \text{ for } \upsilon \neq \kappa \\
%%%%%%%%%%%%%%%
 2\,|q_d| & \text{ for } \upsilon = \kappa \\
\end{cases} \,,\nn
 %%%%%%%%%%%%%%%%%
&\beta_d = \frac{  \Gamma^2 \big (\frac{d} {2} \big )}
{ 2^{d} \, \pi^{ \frac{d-1} {2} }\;
| \cos \big (  \frac{\pi \,d} {2} \big ) |  
\; \Gamma(\frac{d-1}{2}) \,\Gamma (d)} \,.
\end{align}


The bare bosonic propagator, $D_{(0)}(k)$, lacks any $\mathbf{K}$-dependence, 
rendering loop integrals involving it divergent, unless one 
performs a resummation that generates a dynamical dispersion along these 
directions. We remedy this by dressing the propagator with the 
lowest-order finite correction $\Pi_1(k)$ from the one-loop bosonic 
self-energy, which scales as $|\mathbf{K}|^{d-1}/f(k)$, and incorporate 
this correction into all subsequent loop computations. Operationally, 
this means adopting the dressed propagator 
$D_{(1)}(k) = \left[\left(D_{(0)}(k)\right)^{-1} - \Pi_1(k)\right]^{-1}$, 
which effectively reorganizes the perturbative series by promoting the 
$\mathbf{K}$-dependent finite piece of the one-loop bosonic self-energy 
to zeroth order. The correction $\Pi_1(k)$ is none other than the 
well-known \textit{Landau-damping} term, responsible for inducing the 
distinctive $\text{sgn}(k_0)|k_0|^{2/3}$ frequency scaling in the 
fermionic self-energy --- a universal hallmark of NFL physics at quantum 
critical points in diverse strongly correlated systems
\cite{max-isn, max-sdw, Lee-Dalid, ips-uv-ir1, ips-uv-ir2, ips-fflo, ips-u1}. 


%%%%%%%%%%%%%%%%%
\subsection{One-loop fermion self-energies}
\label{secferm}


Two distinct one-loop diagrams contribute to the fermionic self-energy, 
which we compute separately below. Incorporating the dressed bosonic propagator,
the corresponding starting expressions are:
%%%%%%%%%%%%%%%%%%%
\begin{align}
\label{eqsigma1}
\Sigma_1 (q) &= \frac{  \left (i\,e \, \mu^{\frac{x} {2} } \right )^2 }  {2} 
\int_k  G_2 (q-k)\, D_{(1)} (k) 
\nn & =   \frac{i \,e^2 \,  \mu^{x}} {2}  \int_k
\frac{1 } {k_d^2
+   \beta_d \, e^2 \, \mu^{x} \,
 \frac{  |{\mathbf K}|^{d- 1} } { f(k)} }
  \,\frac{\gamma_{d-1} \, \delta_{q-k}^{(2)} 
+ \mathbf  \Gamma \cdot ({\boldsymbol  Q} -\mathbf  K)}
{({\boldsymbol  Q} -\mathbf  K)^2 + \left[\delta_{q-k}^{(2)} \right]^2
}  \, ,
\end{align}
and
\begin{align}
\label{eqsigma2}
\Sigma_2 (q) &= \frac{ \left (i\,e \, \mu^{\frac{x} {2} } \right )^2  }  {2} 
\int_k  G_1 (q-k)\, D_{(1)} (k) 
\nn &  = 
 \frac{i \,e^2 \,  \mu^{x}} {2}  \int_k
\frac{1 } {k_d^2
+   \beta_d \, e^2 \, \mu^{x}  \,
 \frac{  |{\mathbf K}|^{ d - 1} } { f(k)} } 
\,\frac{\gamma_{d-1} \, \delta_{q-k}^{(1)} 
+ \mathbf  \Gamma \cdot ({\boldsymbol  Q} -\mathbf  K)}
{({\boldsymbol  Q} -\mathbf  K)^2 + \left[\delta_{q-k}^{(1)} \right]^2 }  \, ,
\end{align}
where 
\begin{align}
\delta_{q-k}^{(2)} & = -\, \upsilon \, k_{d-1}
+ \delta_q^{(2)} + \kappa \left( k_d^2 - 2  \, q_d \, k_d \right)
\nn & =  \upsilon
\left [ q_{d-1} +
\frac{  \kappa \, q_d^2   } 
{\upsilon} + \frac{ \kappa} {\upsilon} 
\left( k_d^2 - 2  \, q_d \, k_d \right) - \, k_{d-1}
\right ].
\end{align}
%%%%%%%%%%%%%%
Evaluating at $d = d_c - \epsilon$, the singular components of the 
self-energy can be extracted as follows:
\begin{enumerate}



\item For $ \upsilon \neq \kappa $:
%%%%%%%%%%%%%%%%%%%%%%%%
\begin{align}
\Sigma_1 (q) & = -\, 
 \frac{  {\mathcal U}_1 \,e^{\frac{4} {3}} 
 }   { \epsilon}  \,
\frac{ \left [ 
\kappa  \, (2  \,\upsilon - \kappa)
\right]^{\frac{1} {6}} }
{\upsilon}
\, i\left( \mathbf{\Gamma} \cdot \boldsymbol Q \right)
+\mathcal{O}\big(\epsilon^0\big) \,,\nn
%%%%%%%%%%%%%%%%%%%%%
\Sigma_2 (q) & = -\, 
 \frac{  {\mathcal U}_1 \,e^{\frac{4} {3}} 
 }   { \upsilon^{2/3} \, \epsilon}
\, i\left( \mathbf{\Gamma} \cdot \boldsymbol Q \right)
+\mathcal{O}\big(\epsilon^0\big) \,,\quad
%%%%%%%%%%%%%%%%%
{\mathcal U}_1 = \frac {\left[
\Gamma\big (\frac {1} {4} \big) \,
\Gamma\big (\frac {5} {4} \big) \right ]^{1/3}
} 
{ 6 \times  {3}^{1/6} \, \pi^{4/3}} \,.
\end{align}

\item For $ \upsilon = \kappa $:
%%%%%%%%%%%%%%%%%%%%%%%%
\begin{align}
\Sigma_2 (q) =\upsilon \,\Sigma_1 (q)\,,\quad
\Sigma_1 (q)  = -\, 
 \frac{  {\mathcal U}_2 \,e^{\frac{4} {3}} 
 }   { \upsilon \, \epsilon}
\, i\left( \mathbf{\Gamma} \cdot \boldsymbol Q \right)
+\mathcal{O}\big(\epsilon^0\big) 
\,,\quad
%%%%%%%%%%%%%%%%
{\mathcal U}_2 = \frac{2^{1/3}}
{  3^{7/6} } \,.
\end{align}


\end{enumerate}
In this expression, the logarithmic divergences that would emerge from 
integrating out high-energy modes in a Wilsonian approach manifest 
themselves as simple poles at $\epsilon = 0$ (within our dimensional 
regularization framework).


%%%%%%%%%%%%%%%%%%%%%%%%%%%%%%%%%%5
\subsection{One-loop vertex-corrections}

The one-loop fermion-boson vertex functions, $\Gamma_{12}(q,p)$ and 
$\Gamma_{21}(q,p)$, are generically functions of the two external 
frequency-momentum variables, $p$ and $q$. For the purpose of extracting 
the leading-order singular behavior proportional to $1/\epsilon$, we need only 
evaluate these vertices in the limit $p \rightarrow 0$, whereupon the 
corresponding loop-integrals reduce to
\begin{align}
\Gamma_{n_1 n_2}(q,0) &= \frac{e^2\,\mu^{x}}{2} 
\int_k G_{n_1}(k)\,G_{n_2}(k)\,D_{(1)}(k-q) \nonumber \\
&= \frac{e^2\,\mu^{x}}{2} \int_k D_{(1)}(k-q)\,
\frac{\delta_k^{(n_1)}\,\delta_k^{(n_2)} + \mathbf{K}^2 
- \gamma_{d-1}\left(\boldsymbol{\Gamma}\cdot\mathbf{K}\right)
\left[\delta_k^{(n_1)} + \delta_k^{(n_2)}\right]}
{\left[\mathbf{K}^2 + \left\{\delta_k^{(n_1)}\right\}^2\right]
\left[\mathbf{K}^2 + \left\{\delta_k^{(n_2)}\right\}^2\right]}\,.
\end{align}
Since $\Gamma_{12}(q,0) = \Gamma_{21}(q,0)$, it suffices to evaluate
\begin{align}
\label{eqgamint0}
\Gamma_{12}(q,0) &= \frac{e^2\,\mu^{x}}{2} \int_k
\frac{\delta_k^{(1)}\,\delta_k^{(2)} + \mathbf{K}^2 
- \gamma_{d-1}\left(\boldsymbol{\Gamma}\cdot\mathbf{K}\right)
\left[\delta_k^{(1)} + \delta_k^{(2)}\right]}
{\left[\mathbf{K}^2 + \left\{\delta_k^{(1)}\right\}^2\right]
\left[\mathbf{K}^2 + \left\{\delta_k^{(2)}\right\}^2\right]}\,.
\end{align}
The integral vanishes identically in the special case $\upsilon = \kappa$, 
prompting us to concentrate on the generic situation where 
$\upsilon \neq \kappa$. In both scenarios, however, the outcome is the 
same: no singular contribution emerges.



%%%%%%%%%%%%%%%%%%%%%%%%%%%%%%%%%%%%%%%%%5
\section{RG flows for $\upsilon \neq  \kappa $}
\label{secrg}


In our QFT treatment, the action appearing in Eq.~\eqref{eqs1} is 
designated as the \textit{physical action}, formulated at an energy 
scale, $\mu \sim \Lambda$, comprising non-divergent physically-observable 
quantities. Loop corrections, however, generate contributions that 
exhibit either logarithmic or power-law divergences in $\Lambda$. We 
regulate these ultraviolet singularities through renormalization, 
implemented via dimensional regularization, a procedure in which 
divergences re-emerge as poles in the parameter $\epsilon$ in the limit 
$\epsilon \rightarrow 0$. Our computations have been carried out at 
one-loop order. The calculations are conducted within the minimal 
subtraction ($\mathrm{MS}$) renormalization scheme \cite{thooft, weinberg}, 
wherein divergent contributions from loop diagrams are systematically 
removed through the introduction of counterterms. We specifically employ 
the modified minimal subtraction ($\overline{\mathrm{MS}}$) variant, 
which absorbs not only the $1/\epsilon$ pole itself, but also the 
universal finite piece proportional to $\epsilon^0$ that
accompanies it.

The \textit{counterterm action}, constructed to absorb the singular 
contributions arising from quantum corrections, mirrors the structure of 
the physical action in Eq.~\eqref{eqs1} and takes the explicit form,
\begin{align}
\label{actcount}
S_{CT}  = &     \int_k \bar \Psi_1  (k) \,i
\left [ A_1\, \mathbf  \Gamma \cdot \mathbf  K  + 
\gamma_{d-1} \left( A_2 \, k_{d-1} + A_3\,k_d^2 \right)
 \right  ] \Psi_1 (k)  
 \nn & +  \int_k \bar \Psi_2  (k) \,i
\left [ A_4 \, \mathbf  \Gamma \cdot \mathbf  K  +  \gamma_{d-1} \left(
A_5 \, \upsilon \,k_{d-1} 
+ A_6 \, \kappa \, k_d^2  \right) \right  ] \Psi_2 (k)  
%%%%%%%%%%%%%%%%%%%
 \nn & + \frac{1}{2} \int_k A_7 \, k_d^2 \,  \phi (k) \, \phi(-k) 
%%%%%%%%%%
- \left[ \frac{ i\, e \, \mu^{x/2} } {2} 
\int_{k} \int_q A_8 \, \phi (q) \,
 \bar{\Psi}_1 (k+q) \, {\Psi}_2(-k) 
+ \text{h.c.} \right ].
\end{align}
%%%%%%%%%%%%%%%%%%%%%%%%%%%%%%
The coefficients appearing in the counterterms are given by the power 
series,
\begin{align}
A_{\zeta} = \sum_{n=1}^\infty \frac{Z^{(n)}_{\zeta}}{\epsilon^n} 
\quad \text{with} \quad \zeta \in [1, 8]\,,
\end{align}
chosen so as to cancel the divergent contributions $\propto 1/\epsilon^n$ 
arising from the loop-level Feynman diagrams. The $(d-1)$-dimensional 
rotational invariance in the space perpendicular to the FS ensures that 
each term in $\boldsymbol{\Gamma} \cdot \mathbf{K}$ is renormalized 
identically.


With these elements in hand, we formally subtract $S_{CT}$ from the 
so-called \textit{bare} action,
\begin{align}
\label{actren}
S_{\text{bare}} = & \sum_n \int_{k^B} \bar{\Psi}^B_n(k^B)
\, i \left[\boldsymbol{\Gamma} \cdot \mathbf{K}^B 
+ \gamma_{d-1}\,\delta^{(n)}_{k^B}\right] \Psi^B_n(k^B)
+ \int_{k^B} \frac{\left(k^B_d\right)^2 
\phi^B(k^B)\,\phi^B(-k^B)}{2} \nonumber \\
& - \left[\frac{i\,e^B}{2} \int_{k^B} \int_{q^B} \phi^B(q^B)\,
\bar{\Psi}^B_1(k^B+q^B)\,\gamma_{d-1}\,\Psi^B_2(-k^B) 
+ \text{h.c.}\right].
\end{align}
By construction, the resulting \textit{physical} effective action $S$ 
contains only well-defined, non-divergent quantum parameters, whereas 
$S_{\text{bare}}$ is formulated in terms of \textit{bare quantities} 
that may diverge. Throughout, the superscript ``$B$'' designates bare 
fields, couplings, frequencies, and momenta. This framework enables 
physical observables to be extracted from renormalized coupling constants, 
whose evolution is governed by RG flow equations --- differential 
equations that track how the couplings vary as the floating energy scale 
$\mu\,e^{-l}$ changes, or equivalently, as the logarithmic length scale 
$l$ increases.

The RG flow equations are derived by connecting the bare quantities to 
their renormalized counterparts (those without the superscript ``$B$'') 
through multiplicative $Z_\zeta$-factors:
\begin{align}
S_{\text{bare}} = S + S_{CT}\,, \quad Z_{\zeta} = 1 + A_{\zeta}\,,
\end{align}
\begin{align}
&\mathbf{K}^B = \mathbf{K}\,, \quad
k_{d-1}^B = \frac{Z_2}{Z_1}\,k_{d-1}\,, \quad
k^B_d = \sqrt{\frac{Z_3}{Z_1}}\,k_d\,, \nonumber \\
&\Psi_n^B(k^B) = Z_{\Psi_n}^{1/2}\,\Psi_n(k)\,, \quad
\phi_{\pm}^B(k^B) = Z_{\phi}^{1/2}\,\phi_{\pm}\,,
\end{align}
and
\begin{align}
& Z_{\Psi_1} = Z_1\left(\frac{Z_1}{Z_2}\right)
\sqrt{\frac{Z_1}{Z_3}}\,, \quad
Z_{\Psi_2} = Z_4\left(\frac{Z_1}{Z_2}\right)
\sqrt{\frac{Z_1}{Z_3}}\,, \quad
Z_{\phi} = Z_7\left(\frac{Z_1}{Z_2}\right)
\left(\frac{Z_1}{Z_3}\right)^{3/2}\,, \nonumber \\
& \upsilon^B = \frac{Z_5}{Z_4}\left(\frac{Z_1}{Z_2}\right)\upsilon\,, \quad
\kappa^B = \frac{Z_6}{Z_4}\left(\frac{Z_1}{Z_3}\right)\kappa\,, \quad
e^B = Z_e\,e\,\mu^{\frac{\epsilon}{2}}\,, \quad
Z_e = \frac{Z_8\,\sqrt{\frac{Z_1}{Z_2}}}
{\left(\frac{Z_1}{Z_3}\right)^{1/4}\sqrt{Z_1\,Z_4\,Z_7}}\,.
\end{align}
%%%%%%%%%%%%%%%%%%%%%
A gauge freedom exists that allows us to rescale fields and momenta 
independently without altering the action. We fix this freedom by 
imposing $\mathbf{K}^B = \mathbf{K}$, which amounts to measuring the 
scaling dimensions of all other quantities relative to that of 
$\mathbf{K}$. The result is the renormalized action $S$ --- also called 
the Wilsonian effective action --- expressed entirely in terms of 
renormalized, non-divergent parameters.

%%%%%%%%%%%%%%%%%%%%%%%%%%%%%%
\subsection{RG-flow equations from one-loop diagrams}
%%%%%%%%%%%%%%%%%


Gathering our results from the calculations performed at one-loop order, the singular contributions yield
\begin{align}
\label{eqZvals}
Z_1 &= 1 - \frac{\mathcal{U}_1\,e^{\frac{4}{3}}}{\epsilon}\,
\frac{\left[\kappa\,(2\,\upsilon - \kappa)\right]^{\frac{1}{6}}}{\upsilon}\,, 
\quad Z_4 = 1 - \frac{\mathcal{U}_1\,e^{\frac{4}{3}}}{\upsilon^{2/3}\,\epsilon}\,, 
\nonumber \\
Z_2 &= Z_3 = Z_5 = Z_6 = Z_7 = Z_8 = 1\,, \quad
\mathcal{U}_1 = \frac{\left[\Gamma\!\left(\tfrac{1}{4}\right)
\Gamma\!\left(\tfrac{5}{4}\right)\right]^{1/3}}
{6 \times 3^{1/6}\,\pi^{4/3}}\,.
\end{align}
%%%%%%%%%%%%%%%%%%%%%%%%%%%%
The fermionic sector is, in general, must be characterised by two independent 
dynamical critical exponents,
\begin{align}
z = 1 + \frac{\partial \ln\!\left(\frac{Z_1}{Z_2}\right)}{\partial \ln \mu}\,, 
\quad \tilde{z} = 1 + 
\frac{\partial \ln\!\left(\frac{Z_1}{Z_3}\right)}{\partial \ln \mu}\,.
\end{align}
Since the one-loop calculation yields $Z_2 = Z_3 = 1$, which implies 
$\tilde{z} = z$, we treat the two exponents as identical at this 
order. Anomalous dimensions for the fermionic and bosonic fields are 
defined through
\begin{align}
\eta_{\psi_n} = \frac{1}{2}\frac{\partial \ln Z_{\psi_n}}{\partial \ln \mu} 
\quad \text{and} \quad
\eta_\phi = \frac{1}{2}\frac{\partial \ln Z_\phi}{\partial \ln \mu}\,.
\end{align}
The beta-functions describing the RG flows of the three 
coupling constants are
\begin{align}
\beta_e = \frac{de}{d\ln\mu}\,, \quad 
\beta_\upsilon = \frac{d\upsilon}{d\ln\mu}\,, \quad 
\beta_\kappa = \frac{d\kappa}{d\ln\mu}\,.
\end{align}
%%%%%%%%%%%%%%%%%%%%%
The scale $\mu$ entered our formalism purely as a regularization device, 
introduced to tame the ultraviolet divergences that arise in loop 
integrals. Since it does not appear in the underlying microscopic theory, 
physical quantities must be insensitive to its value, and consistency 
demands that the bare parameters entering $S_{\text{bare}}$ likewise 
exhibit no $\mu$-dependence. Enforcing this condition, along with the 
requirement that the finite components of the renormalized quantities 
take the form of systematic expansions in small $\epsilon$, we must parametrize
the above quantities as
\begin{align}
\label{eqexp}
& z = z^{(0)}\,, \quad
\eta_{\psi_n} = \eta_{\psi_n}^{(0)} + \eta_{\psi_n}^{(1)}\,\epsilon\,, \quad
\eta_\phi = \eta_\phi^{(0)} + \eta_\phi^{(1)}\,\epsilon\,, \nonumber \\
& \beta_e = \beta_e^{(0)} + \beta_e^{(1)}\,\epsilon\,, \quad
\beta_\upsilon = \beta_\upsilon^{(0)} + \beta_\upsilon^{(1)}\,\epsilon\,, \quad
\beta_\kappa = \beta_\upsilon^{(0)} + \beta_\kappa^{(1)}\,\epsilon\,,
\end{align}
in the limit $\epsilon \rightarrow 0$. To arrive at the differential 
equations governing the RG flow, we proceed through four 
sequential steps: (1) imposing the condition 
$\frac{d}{d\ln\mu}(\text{bare quantity}) = 0$; (2) substituting the 
expressions from Eqs.~\eqref{eqZvals} and \eqref{eqexp}; (3) expanding 
each equation in powers of $\epsilon$; and (4) matching coefficients of 
regular powers of $\epsilon$ on both sides to determine all the 
quantities in Eq.~\eqref{eqexp}. Carrying out this procedure yields
\begin{align}
& \beta_\upsilon^{(1)} = \beta_\kappa^{(1)} = \eta_{\psi_n}^{(1)} 
= \eta_\phi^{(1)} = 0\,, \quad
\beta_e^{(1)} = -\frac{e}{2}\,, \quad
z = 1 + \beta_e^{(1)}\,\frac{\partial Z_1^{(1)}}{\partial e}\,, \nonumber \\
& \beta_e^{(0)} = -\frac{e}{4}\left[3(z-1) + e\left(
\frac{\partial Z_1^{(1)}}{\partial e} + \frac{\partial Z_4^{(1)}}{\partial e}
\right)\right], \quad
\beta_\upsilon^{(0)} = \upsilon\left(1 - z + \beta_e^{(1)}\,
\frac{\partial Z_4^{(1)}}{\partial e}\right), \nonumber \\
& \beta_\kappa^{(0)} = \kappa\left(1 - z + \beta_e^{(1)}\,
\frac{\partial Z_4^{(1)}}{\partial e}\right), \quad
\eta_{\psi_1}^{(0)} = \frac{3(z-1) + 2\,\beta_e^{(1)}\,
\frac{\partial Z_1^{(1)}}{\partial e}}{4}\,, \nonumber \\
& \eta_{\psi_2}^{(0)} = \frac{3(z-1) + 2\,\beta_e^{(1)}\,
\frac{\partial Z_4^{(1)}}{\partial e}}{4}\,, \quad
\eta_\phi^{(0)} = \frac{5(z-1)}{4}\,.
\end{align}
%%%%%%%%%%%%%%%%%%%%%%%%%
Solving these equations yields
%%%%%%%%%%%%%%%
\begin{align}
\label{eqbeta}
& z = 1 + \frac{2\,\mathcal{U}_1\,\tilde{e}\,\kappa^{1/6}
(2\,\upsilon - \kappa)^{1/6}}{3\,\upsilon}\,, \quad
\eta_{\psi_1} = \frac{5\,\mathcal{U}_1\,\tilde{e}\,\kappa^{1/6}
(2\,\upsilon - \kappa)^{1/6}}{6\,\upsilon}\,, \nonumber \\
& \eta_{\psi_2} = \frac{\mathcal{U}_1\,\tilde{e}
\left[2\,\upsilon^{1/3} + 3\,\kappa^{1/6}(2\,\upsilon - \kappa)^{1/6}\right]}
{6\,\upsilon}\,, \quad
\eta_\phi = \frac{5\,\mathcal{U}_1\,\tilde{e}\,\kappa^{1/6}
(2\,\upsilon - \kappa)^{1/6}}{6\,\upsilon}\,, \nonumber \\
& \frac{\beta_e}{e} = \frac{\mathcal{U}_1\,\tilde{e}
\left[2\,\upsilon^{1/3} - \kappa^{1/6}(2\,\upsilon - \kappa)^{1/6}\right]}
{6\,\upsilon} - \frac{\epsilon}{2}\,, \nonumber \\
& \beta_\upsilon = \frac{2\,\mathcal{U}_1\,\tilde{e}
\left[\upsilon^{1/3} - \kappa^{1/6}(2\,\upsilon - \kappa)^{1/6}\right]}{3}\,, 
\quad
\beta_\kappa = \frac{2\,\mathcal{U}_1\,\tilde{e}\,\kappa
\left[\upsilon^{1/3} - \kappa^{1/6}(2\,\upsilon - \kappa)^{1/6}\right]}
{3\,\upsilon}\,,
\end{align}
where $\tilde{e} = e^{4/3}$. To characterize the infrared physics, we 
track the RG flows with respect to the logarithmic length scale $l$, 
defined through the derivatives,
\begin{align}
\frac{ d e}{ d l} \equiv - \,\beta_e\,,\quad
\frac{ d \upsilon} {d l} \equiv - \, \beta_\upsilon \,,\quad
\frac{ d \kappa} {d l} \equiv - \,\beta_\kappa \,.
\end{align}
%%%%%%%%%%%%%%%%%%%%%%%%%%%%%%%% 

%%%%%%%%%%%%%%%% fixed points %%%%%%%%%%%%%%%%%
\begin{figure}[t!]
\centering
\includegraphics[width=0.4 \textwidth]{sol1.png}
\caption{Illustration of the relationshop between $\upsilon$ and $\kappa$, as dictated by the 
trigonal-warping dispersion (dark pink trajectory) and the fixed-point 
requirement $\kappa = \upsilon$ (dark cyan line). The dark pink 
trajectory is constructed parametrically by varying the warping parameter 
across the interval $\rho \in [10, 1500]$, with $\upsilon$ and $\kappa$ 
evolving according to $\upsilon = (\rho - 1)/(\rho + 1)$ and 
$\kappa = (\rho - 10)(\rho + 1)^2/[(\rho - 1)^2(\rho + 10)]$, 
respectively.}\label{figsolns}
\end{figure}
%%%%%%%%%%%%%%%%%%%%%%%%%%%%%%%%%

%%%%%%%%%%%%%%%%%%%%%%%%%%%%%%%%%%%%%%%%%%%%%%%%%%
\subsection{Nature of the interacting fixed points}
 
We denote the fixed-point values of the coupling constants with a superscript 
``$*$''. For a non-Gaussian fixed point (i.e., one with $\tilde{e} \neq 0$), 
the last two expressions in Eq.~\eqref{eqbeta} require
\begin{align}
{\upsilon^*}^{1/3} = (\kappa^*)^{1/6}(2\,\upsilon^* - \kappa^*)^{1/6}
\quad \Rightarrow \quad \kappa^* = \upsilon^*\,.
\end{align}
Inserting this into the condition $\beta_e = 0$ yields
\begin{align}
\tilde{e}^* = \frac{3\,(\upsilon^*)^{2/3}\,\epsilon}{\mathcal{U}_1}\,.
\end{align}
Rather than an isolated fixed point, this defines a fixed line in the 
three-dimensional space $\{e, \upsilon, \kappa\}$, arising from the 
continuous family of solutions parameterised by $\kappa^* = \upsilon^*$.

For a trigonally-warped FS, the value of $\kappa$ corresponding to a 
given $\upsilon$ is extracted from $\epsilon_{\varsigma,\mathbf{p},+}$ 
[cf.\ Eq.~\eqref{twd}] by varying the warping parameter. Since the FS 
contour scales as $\left[1 + \frac{\varsigma}{\rho}\cos(3\theta_p)\right]$, 
one obtains the relations $\upsilon = \frac{\rho-1}{\rho+1}$ and 
$\kappa = \frac{(\rho-10)(\rho+1)^2}{(\rho-1)^2(\rho+10)}$. A flat 
patch (with $\kappa = 0$) occurs at $\rho = 10$, implying that for 
$\upsilon < 9/11 \approx 0.818$, each FS develops a concave region at 
the hot-spots labeled by superscript ``$(2)$'', corresponding to negative 
$\kappa$. Since our analysis is restricted to convex FS patches, we work 
exclusively in the regime $\upsilon \gtrsim 0.82$. Comparing the value 
of $\kappa$ imposed by the trigonal-warping relation with that obtained 
at the fixed point, we find that the fixed-point value consistently 
exceeds the noninteracting FS value (see Fig.~\ref{figsolns}). Since 
$\kappa = 0$ signifies a flat patch, larger $\kappa$ indicates stronger 
curvature of the FS patch near the hot-spots labeled by superscript 
``$(2)$''.

To assess the stability of the fixed line, we consider, for a given 
$\upsilon$, small deviations from the fixed-point values parameterized 
by $\{\delta\tilde{e},\,\delta\kappa\}$. Substituting into the 
expressions for $\frac{d\tilde{e}}{dl}$ (the negative of the beta 
function for $\tilde{e}$, obtained from $\frac{de}{dl}$) and 
$\frac{d\kappa}{dl}$, and linearizing in the deviation parameters, we 
construct the stability matrix $\mathcal{M}$ from the coefficients of 
$\{\delta\tilde{e},\,\delta\kappa\}$ in the two linearized equations. 
The eigenvalues of $\mathcal{M}$ encode the stability of the fixed 
point. For the non-Gaussian fixed points, the eigenvalue along the 
$\tilde{e}$-direction is always negative, while it vanishes along the 
$\kappa$-direction, indicating that the fixed point is stable with 
respect to perturbations in $\tilde{e}$ but neutral along the 
$\kappa$-axis. The same behavior emerges from the stability matrix 
constructed using $\{\delta\tilde{e},\,\delta\upsilon\}$ when $\kappa$ 
is held fixed.

%%%%%%%%%%%%%%%%%%%%%%%%%%%%%%%%%%%%%%%%%5
\section{RG flows for $\kappa = \upsilon $}
\label{secrg2}


When $\kappa = \upsilon$, the counterterm action assumes the simplified form
\begin{align}
\label{actcount2}
S_{CT} = & \int_k \bar{\Psi}_1(k)\,i\left[A_1\,\boldsymbol{\Gamma}
\cdot\mathbf{K} + \gamma_{d-1}\left(A_2\,k_{d-1} + 
A_3\,k_d^2\right)\right]\Psi_1(k) \nonumber \\
& + \int_k \bar{\Psi}_2(k)\,i\left[A_4\,\boldsymbol{\Gamma}
\cdot\mathbf{K} + \gamma_{d-1}\,A_5\,\upsilon
\left(k_{d-1} + k_d^2\right)\right]\Psi_2(k) \nonumber \\
& + \frac{1}{2}\int_k A_7\,k_d^2\,\phi(k)\,\phi(-k)
- \left[\frac{i\,e\,\mu^{x/2}}{2}\int_k\int_q A_8\,\phi(q)\,
\bar{\Psi}_1(k+q)\,\Psi_2(-k) + \text{h.c.}\right],
\end{align}
with the one-loop renormalization constants reading
\begin{align}
\label{eqZvals}
Z_1 &= 1 - \frac{\mathcal{U}_2\,e^{\frac{4}{3}}}{\upsilon\,\epsilon}\,, 
\quad Z_4 = 1 - \frac{\mathcal{U}_2\,e^{\frac{4}{3}}}{\epsilon}\,, \quad
Z_2 = Z_3 = Z_5 = Z_7 = Z_8 = 1\,, \quad
\mathcal{U}_2 = \frac{2^{1/3}}{3^{7/6}}\,.
\end{align}
The resulting RG flow equations admit solutions of the form
\begin{align}
& z = 1 + \frac{2\,\mathcal{U}_2\,\tilde{e}}{3\,\upsilon}\,, \quad
\eta_{\psi_1} = \frac{5\,\mathcal{U}_2\,\tilde{e}}{6\,\upsilon}\,, \quad
\eta_{\psi_2} = \frac{\mathcal{U}_2\,\tilde{e}\,(3 + 2\,\upsilon)}
{6\,\upsilon}\,, \nonumber \\
\eta_\phi &= \frac{5\,\mathcal{U}_2\,\tilde{e}}{6\,\upsilon}\,, \quad
\frac{\beta_e}{e} = \frac{\mathcal{U}_2\,\tilde{e}\,(2\,\upsilon - 1)}
{6\,\upsilon} - \frac{\epsilon}{2}\,, \quad
\beta_\upsilon = \frac{2\,\mathcal{U}_2\,\tilde{e}\,(\upsilon - 1)}{3}\,.
\end{align}
The zeros of the beta functions are located at $\upsilon^* = 1$ and 
$\tilde{e}^* = 3\,\epsilon/\mathcal{U}_2$. Linearizing about this 
fixed point, the stability matrix evaluates to 
$\text{diag}(-\epsilon/2,\,-2\,\epsilon) + \mathcal{O}(\epsilon^2)$. 
Both eigenvalues are negative, confirming that this is an infrared-stable 
fixed point in the $\tilde{e}\,\upsilon$-plane. The condition 
$\kappa = \upsilon = 1$ corresponds to circular FS patches, signaling 
that the interaction with the CDW boson renormalizes the hot-spots 
labeled ``$(2)$'' until they acquire the same local geometry as those 
labeled ``$(1)$''.

Returning to the more general situation where $\kappa \neq \upsilon$ 
initially, we found that the infrared fixed points are characterized by 
$\kappa^* = \upsilon^*$, which forces the RG trajectory into the 
$\kappa = \upsilon$ subspace irrespective of the starting configuration. 
In the deep infrared, all hot-spots are thus attracted to the 
circular-patch fixed point --- hence, it collapses to the feature that governs 
the system in the absence of trigonal-warping.

%%%%%%%%%%%%%%%%%%%%%%%%%%%%%%%%%%%%%%%%%%%%%
\section{Conclusion}
\label{secsum}

This chapter has been devoted to investigating a quantum critical point 
that emerges at the superradiant transition in a cavity QED set-up. 
Taking as our starting point a honeycomb lattice doped away from 
half-filling, with low-energy excitations described by Dirac cones, we 
have identified the CDW wavevectors that connect hot-spots on the 
resulting FSs, paying careful attention to the role of trigonal-warping 
in shaping the FS geometry. With this framework in place, we sought to 
establish rigorously the existence of the NFL phases anticipated by 
earlier RPA treatments \cite{peng}. Employing a systematic perturbative 
approach grounded in dimensional regularization and RG flow analysis, 
we demonstrated that infrared-stable NFL fixed points do indeed arise 
when the RG trajectories are examined along the direction of the coupling 
constant $e$. The synthetic character of cavity QED platforms affords 
remarkable flexibility in tuning electronic properties and suppressing 
unwanted corrections, making them ideally suited for exploring strongly 
correlated many-body physics through the deliberate manipulation of 
light-matter coupling. Our results thus contribute to a broader program 
aimed at establishing the viability of realizing and detecting NFL 
phases in experimentally accessible cavity QED architectures.

%\bibliographystyle{spphys.bst}
%\bibliography{ref.bib}

\begin{thebibliography}{10}
\providecommand{\url}[1]{{#1}}
\providecommand{\urlprefix}{URL }
\expandafter\ifx\csname urlstyle\endcsname\relax
  \providecommand{\doi}[1]{DOI \discretionary{}{}{}#1}\else
  \providecommand{\doi}{DOI \discretionary{}{}{}\begingroup
  \urlstyle{rm}\Url}\fi

\bibitem{jaksch}
F.~Schlawin, D.~Jaksch, Phys. Rev. Lett. \textbf{123}, 133601 (2019).
\newblock \doi{10.1103/PhysRevLett.123.133601}.
\newblock
  \urlprefix\url{https://link.aps.org/doi/10.1103/PhysRevLett.123.133601}

\bibitem{kollath}
A.~Sheikhan, C.~Kollath, Phys. Rev. A \textbf{99}, 053611 (2019).
\newblock \doi{10.1103/PhysRevA.99.053611}.
\newblock \urlprefix\url{https://link.aps.org/doi/10.1103/PhysRevA.99.053611}

\bibitem{eckstein}
J.~Li, M.~Eckstein, Phys. Rev. Lett. \textbf{125}, 217402 (2020).
\newblock \doi{10.1103/PhysRevLett.125.217402}.
\newblock
  \urlprefix\url{https://link.aps.org/doi/10.1103/PhysRevLett.125.217402}

\bibitem{farokh}
F.~Mivehvar, H.~Ritsch, F.~Piazza, Phys. Rev. Lett. \textbf{122}, 113603
  (2019).
\newblock \doi{10.1103/PhysRevLett.122.113603}.
\newblock
  \urlprefix\url{https://link.aps.org/doi/10.1103/PhysRevLett.122.113603}

\bibitem{demler}
Y.~Ashida, A.m.c. \ifmmode \dot{I}\else \.{I}\fi{}mamo\ifmmode~\breve{g}\else
  \u{g}\fi{}lu, J.~Faist, D.~Jaksch, A.~Cavalleri, E.~Demler, Phys. Rev. X
  \textbf{10}, 041027 (2020).
\newblock \doi{10.1103/PhysRevX.10.041027}.
\newblock \urlprefix\url{https://link.aps.org/doi/10.1103/PhysRevX.10.041027}

\bibitem{diehl}
A.~{Chiocchetta}, D.~{Kiese}, C.P. {Zelle}, F.~{Piazza}, S.~{Diehl}, Nature
  Communications \textbf{12}, 5901 (2021).
\newblock \doi{10.1038/s41467-021-26076-3}

\bibitem{ahana}
A.~Chakraborty, F.~Piazza, Phys. Rev. Lett. \textbf{127}, 177002 (2021).
\newblock \doi{10.1103/PhysRevLett.127.177002}.
\newblock
  \urlprefix\url{https://link.aps.org/doi/10.1103/PhysRevLett.127.177002}

\bibitem{roux}
K.~{Roux}, H.~{Konishi}, V.~{Helson}, J.P. {Brantut}, Nature Communications
  \textbf{11}, 2974 (2020).
\newblock \doi{10.1038/s41467-020-16767-8}

\bibitem{piazza_qed}
F.~{Mivehvar}, F.~{Piazza}, T.~{Donner}, H.~{Ritsch}, Advances in Physics
  \textbf{70}(1), 1 (2021).
\newblock \doi{10.1080/00018732.2021.1969727}

\bibitem{zhang}
X.~Zhang, Y.~Chen, Z.~Wu, J.~Wang, J.~Fan, S.~Deng, H.~Wu, Science
  \textbf{373}(6561), 1359 (2021).
\newblock \doi{10.1126/science.abd4385}.
\newblock
  \urlprefix\url{https://www.science.org/doi/abs/10.1126/science.abd4385}

\bibitem{piazza_superrad}
F.~Piazza, P.~Strack, Phys. Rev. Lett. \textbf{112}, 143003 (2014).
\newblock \doi{10.1103/PhysRevLett.112.143003}.
\newblock
  \urlprefix\url{https://link.aps.org/doi/10.1103/PhysRevLett.112.143003}

\bibitem{bhaseen}
J.~Keeling, M.J. Bhaseen, B.D. Simons, Phys. Rev. Lett. \textbf{112}, 143002
  (2014).
\newblock \doi{10.1103/PhysRevLett.112.143002}.
\newblock
  \urlprefix\url{https://link.aps.org/doi/10.1103/PhysRevLett.112.143002}

\bibitem{basko}
P.~Nataf, T.~Champel, G.~Blatter, D.M. Basko, Phys. Rev. Lett. \textbf{123},
  207402 (2019).
\newblock \doi{10.1103/PhysRevLett.123.207402}.
\newblock
  \urlprefix\url{https://link.aps.org/doi/10.1103/PhysRevLett.123.207402}

\bibitem{polini}
G.M. Andolina, F.M.D. Pellegrino, V.~Giovannetti, A.H. MacDonald, M.~Polini,
  Phys. Rev. B \textbf{100}, 121109 (2019).
\newblock \doi{10.1103/PhysRevB.100.121109}.
\newblock \urlprefix\url{https://link.aps.org/doi/10.1103/PhysRevB.100.121109}

\bibitem{pascal}
D.~Guerci, P.~Simon, C.~Mora, Phys. Rev. Lett. \textbf{125}, 257604 (2020).
\newblock \doi{10.1103/PhysRevLett.125.257604}.
\newblock
  \urlprefix\url{https://link.aps.org/doi/10.1103/PhysRevLett.125.257604}

\bibitem{peng}
P.~Rao, F.~Piazza, Phys. Rev. Lett. \textbf{130}, 083603 (2023).
\newblock \doi{10.1103/PhysRevLett.130.083603}.
\newblock
  \urlprefix\url{https://link.aps.org/doi/10.1103/PhysRevLett.130.083603}

\bibitem{sentef}
F.~Schlawin, D.M. Kennes, M.A. Sentef, Applied Physics Reviews \textbf{9}(1),
  011312 (2022).
\newblock \doi{10.1063/5.0083825}.
\newblock \urlprefix\url{https://doi.org/10.1063/5.0083825}

\bibitem{ref22cav}
N.~Skribanowitz, I.P. Herman, J.C. MacGillivray, M.S. Feld, Phys. Rev. Lett.
  \textbf{30}, 309 (1973).
\newblock \doi{10.1103/PhysRevLett.30.309}.
\newblock \urlprefix\url{https://link.aps.org/doi/10.1103/PhysRevLett.30.309}

\bibitem{ref23cav}
M.~Scheibner, T.~Schmidt, L.~Worschech, A.~Forchel, G.~Bacher, T.~Passow,
  D.~Hommel, Nat. Phys. \textbf{3}(2), 106 (2007).
\newblock \doi{10.1038/nphys494}.
\newblock \urlprefix\url{https://doi.org/10.1038/nphys494}

\bibitem{ref24cav}
I.~{Timothy Noe}, G., J.H. {Kim}, J.~{Lee}, Y.~{Wang}, A.K. {W{\'o}jcik}, S.A.
  {McGill}, D.H. {Reitze}, A.A. {Belyanin}, J.~{Kono}, Nat. Phys.
  \textbf{8}(3), 219 (2012).
\newblock \doi{10.1038/nphys2207}

\bibitem{ref25cav}
K.~{Baumann}, C.~{Guerlin}, F.~{Brennecke}, T.~{Esslinger}, Nature
  \textbf{464}(7293), 1301 (2010).
\newblock \doi{10.1038/nature09009}

\bibitem{neto}
A.H. Castro~Neto, F.~Guinea, N.M.R. Peres, K.S. Novoselov, A.K. Geim, Rev. Mod.
  Phys. \textbf{81}, 109 (2009).
\newblock \doi{10.1103/RevModPhys.81.109}.
\newblock \urlprefix\url{https://link.aps.org/doi/10.1103/RevModPhys.81.109}

\bibitem{maslov-dfl}
P.~Sharma, A.~Principi, D.L. Maslov, Phys. Rev. B \textbf{104}, 045142 (2021).
\newblock \doi{10.1103/PhysRevB.104.045142}.
\newblock \urlprefix\url{https://link.aps.org/doi/10.1103/PhysRevB.104.045142}

\bibitem{ando1998}
T.~Ando, T.~Nakanishi, R.~Saito, Journal of the Physical Society of Japan
  \textbf{67}(8), 2857 (1998).
\newblock \doi{10.1143/JPSJ.67.2857}.
\newblock \urlprefix\url{https://doi.org/10.1143/JPSJ.67.2857}

\bibitem{dresselhaus}
M.S. {Dresselhaus}, G.~{Dresselhaus}, Advances in Physics \textbf{51}(1), 1
  (2002).
\newblock \doi{10.1080/00018730110113644}

\bibitem{levitov}
Z.~{Dong}, P.A. {Lee}, L.~{Levitov}, arXiv e-prints  (2024)

\bibitem{Lee-Dalid}
D.~Dalidovich, S.S. Lee, Phys. Rev. B \textbf{88}, 245106 (2013).
\newblock \doi{10.1103/PhysRevB.88.245106}

\bibitem{ips-uv-ir1}
I.~Mandal, S.S. Lee, Phys. Rev. B \textbf{92}, 035141 (2015).
\newblock \doi{10.1103/PhysRevB.92.035141}

\bibitem{ips-u1}
I.~Mandal, Phys. Rev. Research \textbf{2}, 043277 (2020).
\newblock \doi{10.1103/PhysRevResearch.2.043277}.
\newblock
  \urlprefix\url{https://link.aps.org/doi/10.1103/PhysRevResearch.2.043277}

\bibitem{ips-rafael}
I.~Mandal, R.M. Fernandes, Phys. Rev. B \textbf{107}, 125142 (2023).
\newblock \doi{10.1103/PhysRevB.107.125142}.
\newblock \urlprefix\url{https://link.aps.org/doi/10.1103/PhysRevB.107.125142}

\bibitem{ips-fflo}
D.~Pimenov, I.~Mandal, F.~Piazza, M.~Punk, Phys. Rev. B \textbf{98}, 024510
  (2018).
\newblock \doi{10.1103/PhysRevB.98.024510}

\bibitem{ips-2kf}
I.~Mandal, Nucl. Phys. B \textbf{1005}, 116586 (2024).
\newblock \doi{10.1016/j.nuclphysb.2024.116586}

\bibitem{ips-cavity}
I.~{Mandal}, Ann. Phys. \textbf{474}, 169925 (2025).
\newblock \doi{10.1016/j.aop.2025.169925}

\bibitem{max-isn}
M.A. Metlitski, S.~Sachdev, Phys. Rev. B \textbf{82}, 075127 (2010).
\newblock \doi{10.1103/PhysRevB.82.075127}

\bibitem{max-sdw}
M.A. Metlitski, S.~Sachdev, Phys. Rev. B \textbf{82}, 075128 (2010).
\newblock \doi{10.1103/PhysRevB.82.075128}

\bibitem{ips-uv-ir2}
I.~Mandal, Eur. Phys. J. B \textbf{89}(12), 278 (2016).
\newblock \doi{10.1140/epjb/e2016-70509-4}

\bibitem{thooft}
G.~'t~Hooft, Nucl. Phys. B \textbf{61}, 455 (1973).
\newblock \doi{https://doi.org/10.1016/0550-3213(73)90376-3}

\bibitem{weinberg}
S.~Weinberg, Phys. Rev. D \textbf{8}, 3497 (1973).
\newblock \doi{10.1103/PhysRevD.8.3497}

\end{thebibliography}


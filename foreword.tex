%%%%%%%%%%%%%%%%%%%%%%foreword.tex%%%%%%%%%%%%%%%%%%%%%%%%%%%%%%%%%
% sample foreword
%
% Use this file as a template for your own input.
%
%%%%%%%%%%%%%%%%%%%%%%%% Springer %%%%%%%%%%%%%%%%%%%%%%%%%%

\foreword

It is with pleasure and a little trepidation that I agreed to write the Foreword to the monograph
entitled `Controlled Description of Non-Fermi Liquids Using Field Theoretical Methods' by
Dr. Hab. Ipsita Mandal, a brilliant young theorist working now at Shiv Nadar of Eminence in India.
It is a pleasure because the book is a pleasure to read, in the choice of the topics, the 
layout of the book, and the beautiful manner in which the topics have been presented,
which is clearly a labour of love by the author.  The author is now a condensed matter
physicist with basic training in string theory during her thesis work which was done under
the supervision of one of the world's leading authorities and celebrated physicist Prof. Ashoke
Sen at the Harish-Chandra Institute (HRI), India.  This unusual trajectory has led to
Dr. Mandal being uniquely qualified to write this tract, which uses methods of field theory
to explain condensed matter phenomena.  

The book explores non-Fermi liquids (NFLs), a paradigm of strongly-coupled phenomena arising in many-body systems.
The formulation of the quantum field theory (QFT) techniques is a bit different from what we find in the textbooks of high-energy physics
because condensed-matter systems are non-relativistic. A reduced amount of symmetries, for example, compared to the those applicable for relativistic fermions, make the calculations harder to execute.
The relevant energy-scale flows to be considered are towards the
infrared rather than towards the ultraviolet. By studying NFLs arising in myriad systems which, nevertheless, exhibit some robust analogous features, Dr. Mandal emphasises on the overarching universal features in seemingly unrelated systems. Each chapter has an extensive bibliography
to aid the readers. 

I agreed to write this foreword because I wear two hats, one as a professional colleague who has familiarity with methods of QFT,
and another as a member of the editorial board of the Springer Briefs in Physics on which
I have served for a number of years, although this is the first time I am writing one.  I normally
combine the aforementioned roles by seeking and identifying topics and authors who may be
ideally suited to write expository tracts and books and as a colleague, and also as a member
of the editorial board of a series which serves the important role of bridging the gap between
existing literature in the form of well known textbooks, and material that is known to experts
in the form of, e.g., review articles or lecture notes.  This series plays an important role in 
bringing out monographs on a short scale while leaving the possibility of a larger and more
detailed treatment to the authors for the future.  The present book fits neatly in this niche,
and it is my belief that the author will also consider such a longer treatment, and for the
moment I take this opportunity to congratulate her and the team at Springer including
Lisa Scalone, as well as the readers who are sure to embark on a highly enjoyable cruise.
I am sure the technical parts will enrich the researcher as well as the student, and the
narrative will kindle their desire to learn the subject.  It is also my hope that this book will serve
as an introductory text to students.





\vspace{\baselineskip}
\begin{flushright}\noindent
Bengaluru, February 2026\hfill {\it Balasubramanian Ananthanarayan}\\
\hfill{Professor \& Former Chairman}
\\\hfill{Centre for High Energy Physics}
\\\hfill{Indian Institute of Science}
\\\hfill{Bengaluru 560 012, India}

\end{flushright}



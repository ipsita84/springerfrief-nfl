%%%%%%%%%%%%%%%%%%%%%%%%%%%%%%%%%%%%%%%%%%%%%%%%%%%%%%%
\chapter{Enhancement Of Superconductivity By Ising-Nematic Order Parameter}
\label{chap-sc}


\abstract{This chapter demonstrates how  the techniques of dimensional regularization and renormalization-group 
(RG) are employed to investigate four-fermion interactions capable of inducing superconducting instabilities. The computations are carried out in the presence of Ising-nematic ordering, studied in the earlier chapter. The fermionic part is characterised by critical Fermi 
surfaces of generic dimensions ($m$) and co-dimensions ($d-m$). The superconducting 
instabilities are shown to be enhanced by the  fermionic quasiparticles' interactions with the gapless 
Ising-nematic order parameter. By analysing the RG-flow equations, the stable fixed points are identified as functions of $d$ and $m$. The results reveal that the flow toward a NFL fixed point is consistently preempted by 
Cooper-pair formation in the physical regimes of $(d=3, m=2)$ and 
$(d=2, m=1)$. Most importantly, the results demonstrate a significant 
enhancement of superconductivity driven by the order-parameter 
fluctuations.}


%%%%%%%%%%%%%%%%%%%%%%%%
\section{Introduction}

We continue our exploration of the Ising-nematic ordering \cite{Lee-Dalid, ips-uv-ir1} by examining 
its effect on a four-fermion interaction of strength $V$ in the pairing 
channel. Taking a system embedded in $d$ spatial dimensions (see Chapter 1)
possessing an $m$-dimensional Fermi surface (FS),
the tree-level scaling dimension of $V$ is $[V] = -d + 1 + m/2$. As we 
shall see, scatterings in the BCS channel are enhanced by powers of the 
FS volume, $\sim k_F^m$, so that the effective coupling governing the 
potential superconducting instability becomes $\tilde{V} = V\,k_F^{m/2}$, 
with an enhanced scaling dimension of $[\tilde V ] = -d + 1 + m$. This effective 
coupling is marginal precisely when the co-dimension takes the value 
$d - m = 1$. Our goal is to understand how the interactions between 
the fermions and the critical bosons conspire to enhance a pairing 
instability \cite{ips-sc}.


%%%%%%%%%%%%%%%%%%%%%
\section{Superconducting Instability}
\label{sec-sc-action}



To investigate the emergence of superconducting instabilities within the 
prescribed non-Fermi liquid framework, we augment the fermion-boson 
action introduced in Chapter 1 with the requisite four-fermion 
interaction terms. The following terms can give rise to Cooper pairing in the BCS channel:
\begin{align}
 & \int \left ( \prod_{s=1}^4 {dp_s }  \right)
  \frac{    \mathcal{F}(p_1,p_3;p_2,p_4)  }   { 4 } \, \Big [
 \big \lbrace \bar{\Psi}_{j_1} (p_3) \, {\Psi}_{j_2} (p_1) \big \rbrace \, 
 \big \lbrace \bar{\Psi}_{j_3}  (p_4 ) \, {\Psi}_{j_4}  (p_2 ) \big \rbrace
\nn & \hspace{ 5 cm}
%%%%%%%%%%%%%%%%%%%%%
- \big \lbrace \bar{\Psi}_{j_3}  (p_3)  \sigma_z  {\Psi}_{j_4} (p_1) \big \rbrace \, 
 \big \lbrace \bar{\Psi}_{j_1} (p_4 )  \sigma_z   {\Psi}_{j_2} (p_2 ) \big \rbrace
 \Big ]   \nn
%%%%%%%%%%%%%%%%%%%%%
& =  - \,\int \left ( \prod_{s=1}^4 {dp_s }  \right)
 \mathcal{F}(p_1,p_3;p_2,p_4)  
\Big [
\psi^{\dagger}_{+,j_1} (p_3)\,\psi^{\dagger}_{-,j_2} (- p_1)\, \psi_{-,j_3} (-p_4) \, \psi_{+,j_4} (p_2) 
\Big ]   \,.\nonumber
\end{align}
In this context, the vertex function $ \mathcal{F}(p_1,p_3;p_2,p_4) $ 
exhibits invariance under the simultaneous exchange of indices 
$(p_1, p_3) \leftrightarrow (p_2,p_4) $. Consequently, the action 
defined in Chapter 1 is supplemented by
%%%%%%%%%%%%%
\begin{align}
S^{\rm{SC}}_{\rm{gen}}
&=  \frac{ \mu^{d_v}} { 4  }  \sum_{j_1,j_2,j_3,j_4 }  
\int \left ( \prod_{s=1}^4 {dp_s }  \right)
\,(2\pi)^{d+1} \, \delta^{(d+1)} (p_1+p_2-p_3-p_4 ) \nn &
%%%%%%%%%%%%%%%%
\hspace{3 cm} \times \Big [
 \lbrace \bar \Psi_{j_1}(p_3) \Psi_{j_2 }(p_1) \rbrace  \, \lbrace \bar \Psi_{j_3 }(p_4)  \Psi_{j_4 }(p_2)\rbrace
\nn & \hspace{3.5 cm}
- \lbrace \bar \Psi_{j_3 }(p_3) \sigma_z  \Psi_{j_4 }(p_1) \rbrace  \, 
 \lbrace \bar \Psi_{j_1 }(p_4)  \sigma_z  \Psi_{j_2 }(p_2)\rbrace
 \Big ] \nn & \hspace{3 cm}
 %%%%%%%%%%%%%%
  \times  \Big[ V_{S}(\vec {p}_1,\vec{p}_3; \vec {p}_2,\vec {p}_4) \left( \delta_{j_1,j_3 } \delta_{j_2,j_4 } -  
  \delta_{j_1,j_4} \delta_{j_2,j_3 }\right)
 \nn & \hspace{3.5 cm}  +
  V_{A}(\vec {p}_1,\vec{p}_3; \vec {p}_2,\vec {p}_4) \left( \delta_{j_1,j_3 } \delta_{j_2,j_4 } +  \delta_{j_1,j_4} \delta_{j_2,j_3 }\right) \Big] ,\nonumber
\end{align}
%%%%%%%%%%%%%%%%
The subscripts $S$ and $A$ denote that the vertex functions 
$V_S(\vec{p}_1,\vec{p}_3; \vec{p}_2,\vec{p}_4)$ and 
$V_A(\vec{p}_1,\vec{p}_3; \vec{p}_2,\vec{p}_4)$ possess symmetric 
and antisymmetric properties, respectively, under the interchange 
$p_1 \leftrightarrow p_3$ or $p_2 \leftrightarrow p_4$. These 
couplings are rendered dimensionless via the mass scale $\mu$, 
where $d_v = -d+1+ \frac{m}{2}$ characterizes the scaling 
dimension of the interaction. A Renormalization group (RG) 
analysis will reveal that a superconducting instability 
necessitates the kinematic constraint $\vec{p}_1 = \vec{p}_3$ 
and $\vec{p}_2 = \vec{p}_4$. Furthermore, the assumption of rotational invariance permits 
the simplification $V_{S/A}(\vec{p}_1,\vec{p}_1; \vec{p}_2,\vec{p}_2) = 
V_{S/A}(\theta_1 - \theta_2)$, where $\theta_1$ and $\theta_2$ 
parameterize the angular positions on the Fermi surface. To 
facilitate an analytical treatment, we restrict our focus to 
the $s$-wave channel, characterized by a constant non-zero $V_S$. 
This channel requires a minimum of two fermion flavors; 
accordingly, for $N=2$, the effective action reduces to
%%%%%%%%%%%%%%
\begin{align}
\label{scaction}
S^{\rm{SC}}
&= \frac{ -\, \mu^{d_v}  V_{S}} { 4  }  \sum_{j_1,j_2 }  
\int \left ( \prod_{s=1}^4 {dp_s }  \right)
\,(2\pi)^{d+ 1} \, \delta^{(d+1)} (p_1+p_2-p_3-p_4 )  \,
\left( 1-  \delta_{j_1,j_2 } \right)\nn
& \hspace{2.5 cm} \times 
\Big [
 \lbrace \bar \Psi_{j_1}(p_3)   \Psi_{j_2 }(p_1) \rbrace  \, \lbrace \bar \Psi_{j_2  }(p_4)   \Psi_{j_1 }(p_2)\rbrace
\nn & \hspace{ 3 cm}   -
 \lbrace \bar \Psi_{j_1 }(p_3) \sigma_z  \Psi_{j_2 }(p_1) \rbrace  \, 
 \lbrace \bar \Psi_{j_2 }(p_4)  \sigma_z  \Psi_{j_2 }(p_1)\rbrace
 \Big ].
 \end{align}
%%%%%%%%%%%%%%%%
The results from this simple treatment can be readily generalized to systems 
characterized by an arbitrary number of fermion flavours, $N > 2$, 
and superconducting instability involving nonzero values of angular momentum.


%%%%%%%%%%
\begin{figure}[t!]
        \centering
 \includegraphics[width=0.35 \textwidth]{bcs-tree.png} 
\caption{Tree-level Feynman diagram proportional to $e^2$.}\label{bcstree}
\end{figure} 
%%%%%%%%%%%%%%%%

%%%%%%%%%%%%%%%%%%%%%%%%%%%%%%%%%%%

%%%%%%%%%%%%%%%%%%%%%%%%%%%%%%%%
\subsection{One-loop diagrams generating terms proportional to $V_{S}^2$ }
 
To account for contributions proportional to $V_{S}^2$, we evaluate 
the one-loop diagrams illustrated in Fig.~\ref{fig:loop-4VV}. 
The contribution from Fig.~\ref{fig:loop-4VV}(a) is determined to be proportional to
\begin{align*}
& - (1 -\delta_{j_1,j_2})^2 \int dk \, \mbox{Tr} \big[ G_0 (k+ \mathbf{P_1} - \mathbf{P_3}) \, G_0 (k)\big]
%%%%%%%%%%%%%%
\nn &  = \frac{  2^{ 1 -2d +\frac{m}{2} } 
\left ( 1- \delta_{j_1,j_2}  \right ) \, k_F^{m/2}  \, \pi^{ 1-\frac{d} {2} }   \, \sec \left ( \frac{ (d-m)\, \pi } {2} \right )
} 
{ \Gamma \left ( \frac{ d-m} { 2 } \right )   
\, | \mathbf{  P_3 -P_1} |^{-d+m+ 1}} \,,
\end{align*}
which is logarithmically-divergent at $d-m=1$. Hence, we express it as $\frac{ k_F^{m/2} \,\ln \left( \frac{\Lambda} {|\mathbf{P_1} - \mathbf{P_3} |}\right) }
{2^ { \frac{3m}{2}} \, \pi^{ 1 + \frac{m}{2}} }$.
The results from Figs.~\ref{fig:loop-4VV}(b) and \ref{fig:loop-4VV}(c) are suppressed by powers of $k_F$ and, hence, do not contribute to the beta-functions.
%%%%%%%%%
Noting that $\mbox{Tr} \big[ \sigma_z\, G_0 (k+ \mathbf{P_1} - \mathbf{P_3})\,\sigma_z \, G_0 (k)\big]
= - \mbox{Tr} \big[ G_0 (k+ \mathbf{P_1} - \mathbf{P_3}) \, G_0 (k)\big] $, the full contribution from all one-loop diagrams proportional to $V_{S }^2$ is given by
\begin{align}
& t_{VV} =
 \frac{4\times 2}{2 !} \frac{4}{ 4 \times 4  } \times \frac{ 2^{ 2 + 2d-\frac{m}{2}} \, k_F^{m/2}  \, \mu^{2 \, d_v}
 \, V_{S}^2 }
{  \pi^{ \frac{d}{2}}   \,\Gamma \left ( \frac{ d-m} { 2 } \right )  \epsilon} + \mathcal{O}\left( \epsilon\right)  
= \frac{ v_2 \, \mu^{2 \, d_v +\frac{m} {2} } \,\tilde{k}_F^{m/2} \, V_{S}^2 }{\epsilon}  + \mathcal{O}\left( \epsilon\right) ,
\nn & \text{where }
%%%%%%%%%%%
v_2 =  \frac{ 2^{ 2 -2d+\frac{m}{2}}  }
{ \pi  ^{ \frac{d}{2}  }  \,  \Gamma \left ( \frac{ d-m} { 2 } \right ) } 
\text{ and } d-m=1-\epsilon\,.
\end{align}


%%%%%%%%%%%%%%%%%%%%%%%%%%%%%%%%%%%%%%%%%%%%%%%%%%%%%%
\begin{figure}[t!]
\centering
\subfigure[]
{\includegraphics[width=0.32 \textwidth]{vv1}}
\subfigure[]
{\includegraphics[width=0.35 \textwidth]{vv2}}
\subfigure[]
{\includegraphics[width= 0.26 \textwidth]{vv3.png}}
\caption{One-loop diagrams proportional to $V_S^2.$ Here, $p_4 = p_1+p_2-p_3$.}\label{fig:loop-4VV}
\end{figure}

In accordance with previous treatments in the literature \cite{son, Max-cooper}, the 
analysis must incorporate tree-level contributions within the BCS 
channel, arising from long-range interactions between fermions. 
In the present context, these interactions are mediated by the 
exchange of massless bosons. Consistent with our treatment of 
competing terms, these contributions are evaluated using the 
formalism of dimensional regularization. Specifically, the 
diagrams in Fig.~\ref{bcstree} yield the terms,
\begin{align} 
& \sum_{j_1, j_2} \Big [ 
- \psi^{\dagger}_{+,j_1 } (p_2)\,\psi^{\dagger}_{-,j_2} (- p_2)  \, \psi_{-,j_2} (-p_1) \, \psi_{+,j_1} (p_1)
\nn & \qquad  -  \psi^{\dagger}_{+,j_1 } ( p_1 ) \, \psi^{\dagger}_{-,j_2 } ( -p_1)
     \, \psi_{-,j_2 } (-p_2)  \, \psi_{+,j_1 } (p_2)   \nn
&  \qquad - \, \psi^{\dagger}_{+,j_1} (p_2) \, \psi_{+,j_1 } (p_1 ) \,  \psi^{\dagger}_{+,j_2 } (p_1) \, \psi_{+,j_2 } (p_2 ) 
 \nn & \qquad -   \psi^{\dagger}_{-,j_2} ( -p_2)\, \psi_{-,j_2} ( -p_1 )
\,\psi^{\dagger}_{-,j_1 } (-p_1)  \, \psi_{-,j_1 } (- p_2 )
  \Big ]  \nonumber
\end{align}
which multiply $ \frac{(i\,e)^2  \mu^{x_e}   D_1(p_1-p_2) } {2 \, N}  $. Owing to the symmetry of $D_1(p_1-p_2)$ under the interchange 
$p_1 \leftrightarrow p_2$, this term contributes as 
$ \frac{ e^2  \mu^{x_e} D_1(p_1-p_2) } { N} $ to the components 
governing the pairing instability.

 %%%%%%%%%%
\begin{figure}[t!]
\centering
\subfigure[]{\includegraphics[width= 0.3  \textwidth]{ev1}}
\subfigure[]{\includegraphics[width= 0.3  \textwidth]{ev2}}
\subfigure[]{\includegraphics[width= 0.3  \textwidth]{ev3}}
\subfigure[]{ \includegraphics[width= 0.3  \textwidth]{ev4}}
 \subfigure[]{\includegraphics[width= 0.3  \textwidth]{ev5}}
\subfigure[]{\includegraphics[width=0.3 \textwidth]{ev6}}  
\caption{One-loop diagrams proportional to $ \tilde e \, V_{S}$, each consisting of two fermionic and one bosonic propagators forming the loop. Here $p_4=p_1+p_2-p_3$, $\vec p_1 = \vec p_3$ and $\vec p_2 = \vec p_4$.}\label{fig:loop-4V}
\end{figure}
%%%%%%%%%%%%%%%

Using $  \mathbf{L}_{(q)} ^ 2   \simeq 2\, k_F^2 (1- \cos \theta)/k_F\Rightarrow 
 |\mathbf{L}_{(q)}| \simeq  \sqrt{k_F} \, |\theta|$, the decomposition into angular-momentum channels for an 
$m$-dimensional FS yields a contribution 
proportional to
 \begin{eqnarray}
t_{ee}
&\simeq &
  \frac{ e^2 \, \Lambda^{x_e}} {2 \, N} \times 2  \int_{\theta >0} 
   { d \theta}   \, \frac{ \theta^{m-1} \, |\mathbf{L}_{(q)}|} 
   { \mathbf{L}_{(q)} ^ 3  + \tilde \alpha \, |\vec Q |^{d-m} } 
   %%%%%
=    \frac{  e^2 \, \Lambda^{x_e}} {  N\, k_F^{m/2}  }  \int_{\theta >0} 
   { d  |\mathbf{L}_{(q)}|}   \, \frac{ |\mathbf{L}_{(q)}|^ m } 
   { \mathbf{L}_{(q)} ^ 3  + \tilde \alpha \, |\vec Q |^{d-m} } \nn
%%%
&=&
  \frac{ e^2 \, \Lambda^{x_e}  \,   \Gamma \left( \frac{m+1} {3}\right ) \, \Gamma \left( \frac{ 2-m} {3}\right )} 
 {3 \, N\, k_F^{m/2} \, \tilde \alpha^{\frac{2-m}{3} }  \, |\vec Q |^{ \frac{(d-m)(2-m)}{3} }} \,.
 %%%%%%%%%%%%%%%%%%%%%  
\end{eqnarray}
This result is evidently independent of the specific angular-momentum channel under consideration. For $m = 2 - \delta$ and 
$d = m + 1 - \gamma \varepsilon$, the expression yields a pole of 
the form,
\begin{equation}
t_{ee} \simeq \frac{ \tilde{e} \, \Lambda^{\gamma \varepsilon + 
\frac{(2-m)(m-1)}{6}} \, \Gamma \left( \frac{m+1}{3} \right) } 
{ \beta_d^{\frac{2-m}{3}} N \, k_F^{m/2} } \frac{1}{\delta}\,,
\end{equation}
which is equivalent to the logarithmically-divergent term 
$\frac{ e^2 \Lambda^{x_e} \Gamma \left( \frac{m+1}{3} \right) } 
{ N k_F^{m/2} } \ln \left( \frac{\sqrt{k_F}} 
{ \tilde{\alpha}^{\frac{1}{3}} |\mathbf{Q}|^{\frac{d-m}{3}} } \right)$. 
Upon continuation to the strongly coupled regime at $m=1$, this 
term manifests as an infrared (IR) divergence. The corresponding 
counterterm is given by
%%%%%%%%%%%%%%%%
\bqa
\label{v1}
&& - \frac{    \mu^{\lambda_1   \varepsilon} \,  \tilde{ e } \, v_1 } { 4 \, N\, a \, \epsilon }  \sum_{j_1,j_2 }  
\int \left ( \prod_{s=1}^4 {dp_s }  \right)
 (2\pi)^{d+1 } \, \delta^{(d+1)} (p_1+p_2-p_3-p_4 )  
\left ( 1-  \delta_{j_1, j_2 }   \right)  \nn
&& \qquad \qquad \qquad  \times \,
\Big [
 \lbrace \bar \Psi_{j_1}(p_3)   \Psi_{j_2 }(p_1) \rbrace  \, \lbrace \bar \Psi_{j_1 }(p_4)    \Psi_{j_2 }(p_2)\rbrace
-
 \lbrace \bar \Psi_{j_1 }(p_3)  \sigma_z  \Psi_{j_2 }(p_1) \rbrace  \, 
 \lbrace \bar \Psi_{j_1 }(p_4)   \sigma_z  \Psi_{j_2 }(p_2)\rbrace
 \Big ]  ,\nn
 %%%%%%%%
&&  \lambda_1 =1-  \frac{a\,(7-m^2)} { 6 \, (m+1) } \,,\quad
a \, \epsilon = 2-m\, , \quad
v_1 = \frac{    \Gamma \left( \frac{m+1} {3}\right )}
{ \beta_d^{\frac{ 2-m }{3} }}  \,.
\eqa 
Using $   \mathbf{L}_{(q)} ^ 2   \simeq 2\, k_F^2 (1- \cos \theta)/k_F\Rightarrow 
 |\mathbf{L}_{(q)}| \simeq  \sqrt{k_F} \, |\theta|$, the decomposition into angular momentum channels for an $m$-dimensional FS leads to the contribution being proportional to:
 \begin{align}
t_{ee}
&\simeq 
  \frac{ e^2 \, \Lambda^{x_e}} {2 \, N} \times 2  \int_{\theta >0} 
   { d \theta}   \, \frac{ \theta^{m-1} \, |\mathbf{L}_{(q)}|} 
   { \mathbf{L}_{(q)} ^ 3  + \tilde \alpha \, |\vec Q |^{d-m} } 
   %%%%%
=    \frac{  e^2 \, \Lambda^{x_e}} {  N\, k_F^{m/2}  }  \int_{\theta >0} 
   { d  |\mathbf{L}_{(q)}|}   \, \frac{ |\mathbf{L}_{(q)}|^ m } 
   { \mathbf{L}_{(q)} ^ 3  + \tilde \alpha \, |\vec Q |^{d-m} } \nn
%%%
&=
  \frac{ e^2 \, \Lambda^{x_e}  \,   \Gamma \left( \frac{m+1} {3}\right ) \, \Gamma \left( \frac{ 2-m} {3}\right )} 
 {3 \, N\, k_F^{m/2} \, \tilde \alpha^{\frac{2-m}{3} }  \, |\vec Q |^{ \frac{(d-m)(2-m)}{3} } } \,,
 %%%%%%%%%%%%%%%%%%%%%  
\end{align}
which is independent of the angular momentum channel.
This expression tells us that, for $ m = 2 - \delta $ and $d= m+1-\gamma \, \varepsilon $, we get a pole (in $\delta$) parametrised as
$t_{ee} \simeq   \frac{  \tilde e \, \Lambda^{ \gamma \, \varepsilon+
( 2-m )\, (m-1) } {6 }  }  \,   \Gamma \left( \frac{m+1} {3}\right )\, 
\beta_d^{\frac{ m-2 }{3}}   N^{-1}\, k_F^{-m/2} \,\delta^{- 1}  $,
which is equivalent to the logarithmically divergent term, $  \frac{ e^2 \, \Lambda^{x_e}  \,   \Gamma \left( \frac{m+1} {3}\right ) } 
 {  N\, k_F^{m/2}} 
 \ln \left (  \frac{\sqrt{ k_F}}  {\tilde \alpha^{\frac{1}{3}} \, |\mathbf{Q}|^{\frac{d-m}{3}} }  \right ) $. This term, when continued to the strongly-coupled regime of $m=1$, will translate into the infrared divergence there. The resulting counterterm is given by 
 %%%%%%%%%%%%%%%%
\begin{align}
\label{eqv1}
& \frac{  -\, \mu^{\lambda_1   \varepsilon} \,  \tilde{ e } \, v_1 } { 4 \, N\, a \, \epsilon }  \sum_{j_1,j_2 }  
\int \left ( \prod_{s=1}^4 {dp_s }  \right)
 (2\pi)^{d+1 } \, \delta^{(d+1)} (p_1+p_2-p_3-p_4 )  
\left ( 1-  \delta_{j_1, j_2 }   \right)  \nn
& \hspace{2.5 cm}  \times \,
\Big [
 \lbrace \bar \Psi_{j_1}(p_3)   \Psi_{j_2 }(p_1) \rbrace  \, \lbrace \bar \Psi_{j_1 }(p_4)    \Psi_{j_2 }(p_2)\rbrace
\nn & \hspace{3 cm}  -
 \lbrace \bar \Psi_{j_1 }(p_3)  \sigma_z  \Psi_{j_2 }(p_1) \rbrace  \, 
 \lbrace \bar \Psi_{j_1 }(p_4)   \sigma_z  \Psi_{j_2 }(p_2)\rbrace
 \Big ]  , 
\end{align}
where $\lambda_1 =1-  \frac{a\,(7-m^2)} { 6 \, (m+1) } $, $a \, \epsilon = 2-m$, and $v_1 =    \Gamma \left( \frac{m+1} {3}\right )\, \beta_d^{\frac{ m-2 }{3} } $.


 %%%%%%%%%%%%%%%%%%%%%%%%%%%%%%
 \subsection{One-loop diagrams generating terms proportional to $\tilde e \, V_{S} $ }
 
 


%%%%%%%%%%
\begin{figure}[t!]
\centering
\subfigure[]{ \includegraphics[width= 0.37   \textwidth]{evc1.png}}
\subfigure[] { \includegraphics[width= 0.37 \textwidth]{evc2.png}}
\caption{One-loop diagrams proportional to $ \tilde e \, V_{S}$, resulting from counterterms. The counterterm vertex has been denoted by a blob. Here $p_4=p_1+p_2-p_3$.}\label{figevc}
\end{figure}
%%%%%%%%%%%%%%%%%%%%%%

%%%%%%%%%%%%%%%% 6 %%%%%%%%%%%%%%%%%%%%
\begin{figure}[t!]
\centering
 \subfigure[]  { \includegraphics[width= 0.4   \textwidth]{evzero1}}
\subfigure[] { \includegraphics[width= 0.4  \textwidth]{evzero2}}  
\caption{One-loop diagrams proportional to $ e^2 \, V_{S}$, each consisting of two Yukawa vertices and one fermionic loop.}\label{figevzero}
\end{figure}
%%%%%%%%%%%%%%%%%%%%%%




 The set of one-loop diagrams which can generate terms proportional to $\tilde e \, V_{S} $  are shown in Fig.~\ref{fig:loop-4V}. It turns out that only Figs.~\ref{figevc}(a) and \ref{figevc}(b) contribute \cite{ips-sc}. After cancellation with the appropriate diagrams in Fig.~\ref{figevc} consisting of counterterm-vertices, their net contribution to the beta-function is captured by
  %%%%%%%%%%%%%%%%
$\left (  \frac{v_3  }   {a}  +  v_4  \right)
\frac{ \tilde e \,V_S \,  \mu^{\left (\lambda_2 + \gamma \right )\, \varepsilon}   } {  N \, \varepsilon }$, where $
 v_3 =\frac{2\, \gamma_E }
 { \beta_d^{  \frac{2-m} {3}  } \,(2   \pi )^{\frac{3} {5-m}+m-2  }
 \, \Gamma \left (  \frac{m} {2}  \right)   \Gamma \left (  \frac{3} {2  \left(  m+1 \right ) }  \right) \pi } $, $ v_4=\frac{3 \,  v_3 } { m+1 } - \frac{ 2 } {3} $, and $ \lambda_2 =  { a\,(m-1) }  /  {6} $.
%%%%%%%%%%%%%%%%%%%%%%%%%%5 
 Furthermore, the contribution from Figs.~\ref{fig:loop-4V}(a) and \ref{fig:loop-4V}(b), after cancellation with Figs.~\ref{figevc}(a) and \ref{figevc}(b), is given by $\left (  \frac{v_5  }   {\gamma }  + \frac{ v_6 } {a}  \right) \frac{ \tilde e \,V_S \,  \mu^{ \lambda_1 \varepsilon}   }
{  N \, \varepsilon }$, where $ v_5 = -\, 2^{5 -2d+\frac{m} {2} } \, \gamma_E \,  \Gamma \left (  \frac{m+1} {3}  \right) \, \beta_d^{ \frac{m-2} {3}}  \,  \pi^{-\frac{d} {2}}   \, \Gamma^{-1} \left (  \frac{d-m} {2}  \right) $ and $ v_6=- v_5 $.
 %%%%%%%%%%%%%%%%%%%%
There are two more diagrams in this class, as shown in Fig.~\ref{figevzero}. Due to the vanishing of the trace of the gamma matrices in the fermionic loop, they are identically zero.



 
The counterterms essential to account for the four-fermion interactions are captured by
\begin{align}
S^{\rm{SC}}_{CT} & = -\, \frac{    \mu^{\gamma \, \varepsilon} 
\tilde{ A }_S \tilde{ V}_S} { 4  }  
\sum_{j_1,j_2 }  \int \left ( \prod_{s=1}^4 {dp_s }  \right)
 (2\pi)^{d+1 } \, \delta^{(d+1)} (p_1+p_2-p_3-p_4 )  
\left (  1- \delta_{j_1, j_2 }   \right)  \nn
%%%%%%%%%%%%%%%%%%
& \hspace{ 4 cm} \times \,\Big [
 \lbrace \bar \Psi_{j_1}(p_3)   \Psi_{j_2 }(p_1) \rbrace  \, 
 \lbrace \bar \Psi_{j_1 }(p_4)    \Psi_{j_2 }(p_2)\rbrace
\nn & \hspace{ 4.5 cm}
- \lbrace \bar \Psi_{j_1 }(p_3)  \sigma_z  \Psi_{j_2 }(p_1) \rbrace  \, 
 \lbrace \bar \Psi_{j_1 }(p_4)   \sigma_z  \Psi_{j_2 }(p_2)\rbrace
 \Big ]  ,    \nonumber
\end{align}
where $\tilde{A}_{S} \equiv  \tilde{Z}_S -1
=  \sum_{ \alpha_1   =1}^\infty \frac{\tilde{Z}_{S,\alpha_1 }(\tilde e,\tilde{k}_F, V_S )} { \varepsilon ^{\alpha_1}   }$. On the other hand, denoting the bare quantities by the superscript ``$B$", we obtain the bare four-fermion interaction as
\begin{align}
\label{actsc-ren}
S^{\rm{SC}}_{\rm{bare} } &=    -\,  \frac{  V_S^{\rm{B} } }  { 4 }  \sum_{j_1,j_2 }  
\int \left ( \prod_{s=1}^4 {dp^{\rm{B} }_s }  \right)
\,(2\pi)^{d+1 } \, \delta^{(d+1)} (p^{\rm{B} }_1+p^{\rm{B} }_2-p^{\rm{B} }_3-p^{\rm{B} }_4 ) 
\left ( 1-  \delta_{j_1, j_2 }  \right) \, \nn
& \hspace{2.5 cm}\times \,\Big [
 \lbrace \bar \Psi^{\rm{B} }_{j_1}(p_3) \Psi^{\rm{B} }_{j_2 }(p_1) \rbrace  \, \lbrace \bar \Psi^{\rm{B} }_{j_1 }(p_4)     \Psi^{\rm{B} }_{j_2 }(p_2)\rbrace
\nn  &  \hspace{3 cm} - \lbrace 
\bar \Psi^{\rm{B} }_{j_1 }(p_3)  \sigma_z  \Psi^{\rm{B} }_{j_2 }(p_1) \rbrace  \, 
 \lbrace \bar \Psi^{\rm{B} }_{j_1 }(p_4)  \sigma_z  \Psi^{\rm{B} }_{j_2 }(p_2)\rbrace
 \Big ] \,,
\end{align}
%%%%%%%%%%%%%%
where $  V_S^{\rm{B} } \, k_F^{m/2} = \tilde{Z}_S \, Z_{\psi}^{-2}  \left( \frac{Z_2}{Z_1} \right)^{3 (d-m)} \mu^{  \gamma \, \varepsilon} \, \tilde{V}_S $.
%%%%%%%%%%%%%%%%
%%%%%%%%%%%%%%%%%%%%% roots  %%%%%%%%%%%%%%%%%%%%%
\begin{figure}[t!]
\centering
 \subfigure[]{\includegraphics[width=0.45 \textwidth]{root1}} 
\hspace{1 cm}
\subfigure[]{\includegraphics[width=0.45 \textwidth]{root2}}
\caption{The contourplots of the two fixed points ($\tilde{ V }_S^*$) of $\beta_V$ as functions of $d$ and $m$. The intersection of the white areas in the two density-plots corresponds to regions where no perturbative fixed point exists. The dashed black (red) curve (line) represents $d =  d_c (d = \tilde d_c )$ in each plot.
}\label{roots}
\end{figure}
%%%%%%%%%%%%%%%%%%%%%%%%%%%%%%%%%%%%%%%%%%%%%%%%%%%%%%
Using the definition $2-m = a \, \varepsilon \Rightarrow a= \frac{ 3 \,(1 - \gamma )  } { 1 \, + \, (1- \gamma )
\, \varepsilon } $, we have
%%%%%%%%%%%%%%%%%%
$ \frac{ \tilde{Z}_{S, 1 }  \, \tilde{V}_S \, \mu^{\gamma \, \varepsilon } }
{    \varepsilon  }
=  \frac{ \left (  \frac{ v_1 } {a}   + \frac{v_5 \tilde{V}_S} {\gamma }  + \frac{ v_6 \tilde{V}_S } {a} \right ) 
\tilde{e} \, \mu^{\lambda_1 \, \varepsilon } } {N   \, \varepsilon }
+ \frac{v_2 \, \tilde{V}_S^2 \, \mu^{\gamma \, \varepsilon} } { \gamma \, \varepsilon }
+\frac{ \left(   \frac{v_3 } {a } +    v_4  \right) 
      \tilde e \, \tilde V_S \, \mu^{\left (\lambda_2 +\gamma \right ) \varepsilon }} 
{ N \, \varepsilon}$.
%%%%%%%%%%%%%%%%%%%%%%




%%%%%%%%%%%%%%%%%%%%% roots  %%%%%%%%%%%%%%%%%%%%%
\begin{figure}[t!]
\centering
\includegraphics[width=0.45 \textwidth]{root2_enlarged}
\caption{The contourplot illustrates the second root of the beta-function, 
$\beta_V = 0$, which corresponds to the fixed-point value 
$\tilde{V}_S^*$ as a function of the dimensions $m$ and $d$. To 
highlight the behavior in the vicinity of the physical limit, 
the region surrounding $m = 1$ has been magnified. The dashed black 
line denotes the critical dimension, $d = d_c$.}\label{roots2}
\end{figure}
%%%%%%%%%%%%%%%%%%%%%%%%%%%%%%%%%%%%%%%%%%%%%%%%%%%%%%

Since the bare quantities should not depend on the floating mass scale, $\mu$, we demand that $\frac{d}{d \ln \mu}\left(  V_S^{\rm{B} } \, k_F^{m/2} \right) =0 $. This leads to the beta-function for $\tilde V $ as
%%%%%%%%%%%
\begin{align}
&\beta_V +\frac{ \Big \lbrace   \frac{ v_1 } {a}  + 
\left ( \frac{v_3 +v_6 } {a } +    v_4 
+ \frac{v_5  } {\gamma }    \right ) \tilde{V}_S
\Big \rbrace  \, {\beta_{\tilde e} } 
%%%%%%
+ \left ( \frac{v_3 + v_6 } {a } +    v_4 +  \frac{v_5  } {\gamma }   \right )  \beta_V \, \tilde e  } { N \, \varepsilon}  
\nn & + \left \lbrace  \gamma \, \varepsilon - 4 \, \eta_\psi + 3 \left( d-m \right) \left( 1-z \right)
\right \rbrace  \left( \tilde V_S  + \frac{v_2 \, \tilde V_S^2 } {\gamma \, \varepsilon } \right) 
%%%%%%%%%%%%
 \nn & + \frac{   2 \,v_2 \, \tilde V_S \, \beta_V } 
 {\gamma \, \varepsilon}  
%%%%%%%%%%%
+ \frac{ 
 \big \lbrace  \lambda_1 \, \varepsilon - 4 \, \eta_\psi + 3  ( d-m ) ( 1-z )\big \rbrace 
\left(   \frac{v_1 } {a } +  \frac{ v_5   \tilde V_S}{ \gamma }  + 
 \frac{ v_6 \tilde{V}_S } {a}\right) \tilde e } { N \, \varepsilon} 
%%%%%%%%%%%
\nn & +  \big \lbrace \left( \lambda_2 + \gamma \right) \varepsilon - 4 \, \eta_\psi + 3 \left( d-m \right) \left( 1-z \right)
\big \rbrace 
\frac{ \left(   \frac{v_3 } {a } +  v_4   \right) 
              \tilde e \, \tilde V_S } 
{ N \, \varepsilon}
= 0  \,,
%%%%%%%%%%%%%%
\end{align}
Because there are two coupling constants, we have to deal with two beta-functions, $\beta_V$ and $\beta_{\tilde e}$.
Following the usual procedure, the dependence of the beta-functions on $\epsilon $ must be of the forms,
$\beta_V \equiv \frac{\partial \tilde{V}_S } {\partial \ln \mu}
= \beta_V ^{(0)} +  \beta_V ^{( 1 )}  \varepsilon $ and $ 
\beta_{\tilde e} \equiv \frac{\partial \tilde{e}  } {\partial \ln \mu} 
 = -\, \frac{m+1}{3} \,\varepsilon \, \tilde{e} + \mathcal{O } \left( \tilde{e}^2 \right)$.
%%%%%%%%%%%%%%%%%%%%%%
Hence, we need to compare the coefficients of the regular powers of $\varepsilon $, which lead to
\begin{align}
\frac{\partial \tilde{ V }_S }  { \partial l } & =
\gamma \, \varepsilon  \tilde V_S 
- v_2   \tilde V_S^2
- \frac{(7-m^2) \, v_1 } { 6 \,N \, ( m+1) }\, \tilde e  
%%%%%%%%%%%%%
\nn & \quad +  \Big[
\frac{  \left ( 5-m^2 -2m \right )\left  (3 \, v_5 - v_6 \right )   } 
{ m+1  }
+
(m-1)  \,  v_3   - 2\, (m+1)\, (  u_1 +  v_4  )
\Big]  \frac{\tilde e \,  \tilde{ V }_S } {6 \, N},\nonumber
\end{align}
%%%%%%%%%%%%%%%%%%%%%%
correct upto $\mathcal{O} \left(  \tilde{e}^2, \tilde{ e} \, \varepsilon, \varepsilon^2  \right)$ at the level one-loop corrections.
The final simplified form turns out to be
%%%%%%%%%%%%
\begin{align}
\label{betav}
\frac{\partial \tilde{ V }_S }  { \partial l } =
\begin{cases}
\left(  \varepsilon-\frac{1 } { 2} \right)  \tilde V_S \, 
-v_2 \tilde V_S^2
-\frac{ v_1 } { 2 \, N} \, \tilde{e}
+\frac{    3 \, v_5-  4\,  v_4 - v_6  - 4 \, u_1  } { 6\, N} 
\, \tilde{e}\, \tilde V_S
& \mbox{ for } m=1 \\
%%%%%%%%%%%%%%%%%%%%%%%%
\gamma \, \varepsilon  \tilde V_S
-v_2 \tilde V_S^2
-\frac{  v_1 } { 6 \, N}   \, \tilde{e}
+\frac{  v_3 + 6\, v_4 - 3 \, v_5 + v_6 - 6 \, u_1
  } { 6 \, N} \, \tilde{e}\, \tilde V_S
& \mbox{ for } m=2 \mbox{ and } \gamma = \pm 1 
\end{cases} . \nonumber
 \end{align}
%%%%%%%%%%%%%%%%%%%
The RG flows for some relevant values of $(d,m)$ have been displayed in Fig.~\ref{figflows}.

 



%%%%%%%%%%%%%%%%%%%%%%%%%%%%%%%%%%%%%%%%%%%%%%
\section{Stability of the solutions and final remarks}
\label{sec-sc-soln}








The fixed points of the theory are determined by the simultaneous solutions 
to the flow equations $\frac{\partial \tilde{V}_S}{\partial l} = 0$ and 
$\frac{\partial \tilde{e}}{\partial l} = 0$. The resulting quadratic 
equation for the fixed-point values $\tilde{V}_S^*$ yields two distinct 
roots. Figures~\ref{roots} and \ref{roots2} display the contour plots of 
these roots as functions of $m$ and $d$, where the analysis is 
constrained to the perturbative window $|\tilde{V}_S^*| < 1$. This 
restriction ensures the validity of the underlying expansion. In 
Fig.~\ref{roots}, the intersection of the white regions indicates 
parameter regimes where no perturbative fixed point exists. In these 
regimes, the system is invariably driven toward a superconducting state 
as $\tilde{V}_S^*$ flows into the strong-coupling limit, regardless of 
 the initial bare couplings. To elucidate the stability of these fixed 
points, representative RG flows for selected values of $m$ and $d$ are 
presented in Fig.~\ref{figflows}. Several key observations emerge:
%%%%%%%%%%%%%%%%%%%%%%%%%%%%%%%%%%%%%%%%%%%%%%
\begin{figure}[t!]
\centering
\subfigure[]{\includegraphics[width=0.44 \textwidth]{flow1} } \quad
\subfigure[]{\includegraphics[width=0.44 \textwidth]{flow3}}\\
\subfigure[]{\includegraphics[width=0.44 \textwidth]{flow2} } \quad
\subfigure[]{\includegraphics[width=0.44 \textwidth]{flow4}}
\caption{The representative RG flows in the various regions of the $(d,m)$-plane. The black dots represent the finite fixed points when they exist. The contour-shading conveys the magnitudes of the flow vector $\left(   \frac{\partial \tilde{ V }_S }  { \partial l } , \frac{\partial \tilde{ e}  }  { \partial l } \right)$ in different regions.}\label{figflows}
\end{figure}
%%%%%%%%%%%%%%%%%%%%%%%%%%%%%%%%%%%%%%%%%%%%%%%%%%%%%% 
\begin{enumerate}
\item At $(d=3, m=2)$, the only finite solution is the Gaussian fixed 
point $(\tilde{V}_S^* = 0, \tilde{e}^* = 0)$, which is IR unstable [cf. 
Fig.~\ref{figflows}(d)]. In the absence of gauge fluctuations 
($\tilde{e} = 0$), superconductivity only arises for attractive initial 
couplings ($\tilde{V}_S < 0$), consistent with standard BCS theory in a 
Fermi liquid. However, the introduction of a non-zero coupling 
$\tilde{e} > 0$ destabilizes the non-Fermi liquid (NFL) phase, 
driving the system toward strong-coupling superconductivity even for 
initially repulsive interactions. Consequently, order-parameter 
fluctuations significantly enhance the superconducting instability relative 
to the Fermi-liquid benchmark.

\item In the vicinity of $(d=5/2, m=1)$ (magnified in Fig.~\ref{roots2}), 
a narrow region exists below $d_c$ where two perturbative solutions 
are present: $(\tilde{V}_S^* = -f_1, \tilde{e}^* = \frac{N\varepsilon}{u_1})$ 
and the trivial fixed point [see Fig.~\ref{figflows}(a)]. Here, the NFL phase 
remains stable for a given initial $\tilde{e} > 0$. Conversely, in the 
broader vicinity of $(d=2, m=1)$, no perturbative fixed points exist, 
rendering the system unstable to superconductivity for any initial value of 
$\tilde{V}_S$ [cf. Fig.~\ref{figflows}(c)]. This behavior holds for 
$m=1$ and $d = d_c - \varepsilon$ with $\varepsilon \gtrsim 0.017$.

The fluctuations of the order parameter thus provide a mechanism for the 
marked enhancement of pairing in the physical regime of interest near 
$(d=2, m=1)$. While the expansion at $\varepsilon \sim 1/2$ technically 
lies at the boundary of perturbative control, a continuous extrapolation 
to the physical case suggests that the pairing instability is strongly 
amplified near the $(2+1)$-dimensional Ising-nematic critical point. This 
implies that the destruction of quasiparticles is preempted by Cooper 
pairing, a result consistent with the findings of Ref.~\cite{Max-cooper}.
\end{enumerate}

Several remarks regarding the enhanced superconductivity at $(d=3, m=2)$ 
and $(d=2, m=1)$ are warranted. The Yukawa-like coupling $\tilde{e}$ 
generates an effective attractive interaction that mediates Cooper 
pairing, ensuring that the RG flow targets the strong-coupling superconducting 
regime for any non-vanishing $\tilde{e}$. The energy scale associated 
with this instability consistently exceeds the NFL scale at which 
quasiparticle degradation becomes appreciable. While the magnitude of 
the pairing gap remains sensitive to the initial couplings, the qualitative 
behavior aligns with the estimates of Metlitski \textit{et al.} 
\cite{Max-cooper} upon setting $\varepsilon \simeq 0$ and 
$\varepsilon = 1/2$ for the $(d=3, m=2)$ and $(d=2, m=1)$ cases, respectively.



%\bibliographystyle{spphys.bst}
%\bibliography{ref.bib}

\begin{thebibliography}{1}
\providecommand{\url}[1]{{#1}}
\providecommand{\urlprefix}{URL }
\expandafter\ifx\csname urlstyle\endcsname\relax
  \providecommand{\doi}[1]{DOI \discretionary{}{}{}#1}\else
  \providecommand{\doi}{DOI \discretionary{}{}{}\begingroup
  \urlstyle{rm}\Url}\fi
  
  \bibitem{Lee-Dalid}
D.~Dalidovich, S.S. Lee, Phys. Rev. B \textbf{88}, 245106 (2013).
\newblock \doi{10.1103/PhysRevB.88.245106}


\bibitem{ips-uv-ir1}
I.~Mandal, S.S. Lee, Phys. Rev. B \textbf{92}, 035141 (2015).
\newblock \doi{10.1103/PhysRevB.92.035141}

\bibitem{ips-sc}
I.~Mandal, Phys. Rev. B \textbf{94}, 115138 (2016).
\newblock \doi{10.1103/PhysRevB.94.115138}

\bibitem{son}
D.T. Son, Phys. Rev. D \textbf{59}, 094019 (1999).
\newblock \doi{10.1103/PhysRevD.59.094019}

\bibitem{Max-cooper}
M.A. Metlitski, D.F. Mross, S.~Sachdev, T.~Senthil, Phys. Rev. B \textbf{91},
  115111 (2015).
\newblock \doi{10.1103/PhysRevB.91.115111}

\end{thebibliography}
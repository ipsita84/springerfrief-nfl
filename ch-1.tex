\chapter{Introduction}
\label{intro-intro}

\abstract{In this introductory chapter, we outline the unique and exotic 
characters of the strange-metallic systems that we are going to ponder 
over in the various chapters of the book. This encompasses a few examples
of non-Fermi liquid (NFL) phases that arise in quantum critical points (QCPs) or when electrons intricate with transverse gauge fields. In particular, they represent strongly-interacting 
metallic systems, closely connected with the breakdown of Landau's 
Fermi-liquid (FL) theory and the emergence of anomalous thermodynamic and 
transport properties. The hallmark signatures of the NFL behaviour involve various striking features such as
violation of the well-known $ \sim k_0^2$ dependence of the FLs' fermionic self-energy on the frequency $ k_0 $ (i.e., the energy deviation from the Fermi level), emergence of the 
characteristic $\sim \text{sgn}(k_0)|k_0|^{2/3}$ behaviour instead, and the enhanced affinity towards 
superconducting instabilities (characteristic of unconventional superconductors). We identify some principal scenarios in which 
such strange-metallic behaviour emerges in seemingly distinct circumstances, but encompassing the crucial feature that finite-density 
fermions (defining a sharp Fermi surface) are interacting strongly with massless order-parameter bosons or transverse gauge fields. 
The gaplessness of the bosonic fluctuations is the essential ingredient which endows the one-loop corrected bosonic propators with Landau-damping.
We critically survey the main theoretical
approaches that have been formulated to tackle the strong correlations in these problems, from the Hertz-Millis framework 
and large-$N$ expansions to RPA-based schemes and deformations of the 
bosonic dispersion, highlighting the limitations inherent in each. 
Against this backdrop, we advocate for the method of dimensional 
regularization as the framework of choice for obtaining controlled 
perturbative access to NFL fixed points, and outline how it is 
systematically implemented throughout the rest of the book.}


Non-Fermi liquids (NFLs) are metallic states that encapsulate, in 
striking fashion, the consequences of strong interactions in condensed 
matter systems \cite{max-isn, max-sdw, olav, Lee-Dalid, ips-uv-ir1, 
ips-uv-ir2, ips-fflo, ips-u1, metzner1, metzner2, ips-2kf, ips-cavity, 
ips-rafael, ips-moire-cdw}, and resist analysis by conventional 
theoretical methods. Chief among these is Landau's Fermi liquid (FL) 
theory, which provides an excellent description of normal metals but 
breaks down entirely for NFLs. The reason is fundamental: Landau's 
framework rests on the existence of long-lived quasiparticles, and when 
interactions are strong enough to destroy these quasiparticles, the 
entire edifice collapses. NFLs therefore exhibit thermodynamic and 
transport properties \cite{ips-subir, ips-c2, ips-hermann, ips-hermann2, 
ips-hermann3, ips-hermann-review} that are qualitatively distinct from 
those of ordinary metals. A telling example is the DC resistivity 
$\rho$: in a conventional FL, electron-electron interactions give rise 
to the characteristic $\rho \sim T^2$ dependence at low temperatures, 
whereas in an NFL one finds $\rho \sim T^\alpha$ with $\alpha \neq 2$, 
the most common cases being $\alpha = 1$ and $\alpha = 4/3$. Further 
distinctions manifest in properties such as the optical conductivity 
\cite{ips-subir, ips-c2} and an enhanced propensity toward 
superconducting instabilities \cite{ips-sudip1, ips-sc, ips-c2}. The 
principal physical scenarios in which such strange-metallic behavior 
emerges include: (1) finite-density fermions interacting with 
order-parameter bosons that become massless at a quantum critical point 
(QCP) \cite{max-isn, max-sdw, Lee-Dalid, ips-uv-ir1, ips-uv-ir2, 
ips-fflo, ips-rafael, ips-2kf, ips-cavity}; (2) finite-density fermions 
coupled to transverse gauge fields \cite{olav, ips-sudip1, ips-sudip2, 
ips-u1} that remain massless; and (3) Fermi points at band-touching 
semimetals subject to unscreened Coulomb interactions, probed at zero 
chemical potential \cite{Abrikosov, moon-xu, rahul-sid, ips-rahul, 
ips-hermann, ips-hermann2, ips-hermann3, malcolm-bitan, ips-birefringent, 
ips-hermann-review}. This last class involves NFL behaviour at isolated 
Fermi points (encoding points where the density-of-states goes to zero) and is therefore free from the complications associated 
with the presence of an extended Fermi surface (FS) accompanying finite-density systems.

Extensive efforts have been devoted to accessing NFL states through 
controlled approximations, employing a range of distinct and 
complementary approaches, which we now survey in turn:
\begin{enumerate}

\item The approach of Hertz and Millis \cite{hertz, millis} proceeds by 
performing a renormalization group (RG) analysis on an effective bosonic 
action for the order parameter, obtained by integrating out the fermions. 
This procedure is not well-justified, however, since one set of gapless 
degrees of freedom --- the fermionic fields --- is eliminated at the 
outset, and both fermions and bosons must instead be treated on an equal 
footing.

\item Setting aside the Hertz-Millis formulation, an early and natural 
attempt involved introducing a large number $N$ of fermion flavors 
\cite{ALTSHULER, polchinski, ybkim} as a mathematical device, motivated 
by the intuition that additional flavors should not qualitatively alter 
the physics beyond enlarging the flavor-symmetry group. The 
infinite-flavor limit was then to serve as the starting point for a 
controlled $1/N$ expansion, with the hope that setting $N$ to its 
physical value at the end would yield a reliable approximation. It was 
subsequently established, however, that the $N \rightarrow \infty$ limit 
is not described by a mean-field theory for finite-density fermions with 
a well-defined FS \cite{SSLee-largeN, max-isn}, contrary to the original 
assumption. The culprit is the proliferation of planar diagrams 
\cite{SSLee-largeN}, which renders even the large-$N$ theory strongly 
interacting: solving it requires a nontrivial resummation of infinitely 
many Feynman diagrams. These serious shortcomings prompted the search 
for alternative approaches.

\item The insights above also call into question physical pictures based 
on the random-phase approximation (RPA), since RPA amounts to nothing 
more than the leading-order term in the large-$N$ expansion. One 
alternative scheme that circumvents these difficulties involves 
deforming the bare dispersion of the quantum fields. In 
Refs.~\cite{nayak, mross}, this was accomplished by replacing the 
bosonic kinetic term $\boldsymbol{k}^2|\phi(k)|^2$ with 
$\boldsymbol{k}^{1+\epsilon}|\phi(k)|^2$. For $\epsilon \in (0,1)$, 
the density of states of the order-parameter boson is suppressed at low 
energies, reducing quantum fluctuations and making $\epsilon$ a viable 
perturbative parameter. This scheme retains a finite fermionic density 
of states and preserves all microscopic symmetries. Its drawback, 
however, is that the nonanalytic momentum dependence of the bosonic 
kinetic term corresponds to a nonlocal hopping in real space, which 
prevents the collective mode from acquiring an anomalous dimension, 
since short-distance quantum fluctuations cannot renormalize a nonlocal 
hopping term.

\item The approach that achieves genuine mathematical control over the 
perturbative expansion is dimensional regularization \cite{olav, 
Lee-Dalid, ips-uv-ir1}. In our non-relativistic setting, one can tune 
independently the number of dimensions perpendicular to the FS 
(denoted as $d_{co}$) and the dimension of the FS itself (denoted as $d_F$). 
Increasing $d_{co}$ amounts to continuously raising the number of 
spacetime dimensions in which the FS is embedded, which suppresses 
quantum fluctuations by reducing the density of states in higher 
dimensions.

\end{enumerate}
%%%%%%%%%%%%%%%%%%
Throughout this book, we adopt the dimensional regularization framework to analyse the systems under consideration, 
in the spirit of Refs.~\cite{Lee-Dalid, ips-uv-ir1}. A key virtue of 
this approach is that it preserves locality in real space and, for 
$d_F = 1$, an additional locality emerges in the momentum space as well. The 
next chapter lays out the most important building blocks for 
constructing the effective action that serves as the starting point for 
applying dimensional regularization, with the Ising-nematic quantum 
critical point as the guiding example. The NFL systems treated in the 
remaining chapters draw on minor variations of this framework, with the 
essential structure established in Chapter 2 remaining intact throughout.
The variation mainly lies in the formulation of the two-component spinors, which lead to differing forms of the one-loop bosonic self-energy.
Despite the different forms, all of them feature a term representing Landau-damping (analogous to an overdamped harmonic-oscillator mode) arising from one-loop corrections. The overarching consequence of this is the drastic modification of the dependence of the fermionic self-energy on the Matsubara-frequency ($k_0$) from the $\sim k_0^2 $ behaviour of the FLs --- the resulting one-loop fermionic self-energy shows a universal scaling form, viz.
$ \sim \text{sgn}(k_0)|k_0|^{2/3}$.


\begin{thebibliography}{10}
\providecommand{\url}[1]{{#1}}
\providecommand{\urlprefix}{URL }
\expandafter\ifx\csname urlstyle\endcsname\relax
  \providecommand{\doi}[1]{DOI \discretionary{}{}{}#1}\else
  \providecommand{\doi}{DOI \discretionary{}{}{}\begingroup
  \urlstyle{rm}\Url}\fi

\bibitem{max-isn}
M.A. Metlitski, S.~Sachdev, Phys. Rev. B \textbf{82}, 075127 (2010).
\newblock \doi{10.1103/PhysRevB.82.075127}

\bibitem{max-sdw}
M.A. Metlitski, S.~Sachdev, Phys. Rev. B \textbf{82}, 075128 (2010).
\newblock \doi{10.1103/PhysRevB.82.075128}

\bibitem{olav}
S.~Chakravarty, R.E. Norton, O.F. Sylju\aa{}sen, Phys. Rev. Lett. \textbf{74},
  1423 (1995).
\newblock \doi{10.1103/PhysRevLett.74.1423}.
\newblock \urlprefix\url{https://link.aps.org/doi/10.1103/PhysRevLett.74.1423}

\bibitem{Lee-Dalid}
D.~Dalidovich, S.S. Lee, Phys. Rev. B \textbf{88}, 245106 (2013).
\newblock \doi{10.1103/PhysRevB.88.245106}

\bibitem{ips-uv-ir1}
I.~Mandal, S.S. Lee, Phys. Rev. B \textbf{92}, 035141 (2015).
\newblock \doi{10.1103/PhysRevB.92.035141}

\bibitem{ips-uv-ir2}
I.~Mandal, Eur. Phys. J. B \textbf{89}(12), 278 (2016).
\newblock \doi{10.1140/epjb/e2016-70509-4}

\bibitem{ips-fflo}
D.~Pimenov, I.~Mandal, F.~Piazza, M.~Punk, Phys. Rev. B \textbf{98}, 024510
  (2018).
\newblock \doi{10.1103/PhysRevB.98.024510}

\bibitem{ips-u1}
I.~Mandal, Phys. Rev. Research \textbf{2}, 043277 (2020).
\newblock \doi{10.1103/PhysRevResearch.2.043277}.
\newblock
  \urlprefix\url{https://link.aps.org/doi/10.1103/PhysRevResearch.2.043277}

\bibitem{metzner1}
T.~Holder, W.~Metzner, Phys. Rev. B \textbf{90}, 161106 (2014).
\newblock \doi{10.1103/PhysRevB.90.161106}.
\newblock \urlprefix\url{https://link.aps.org/doi/10.1103/PhysRevB.90.161106}

\bibitem{metzner2}
J.~S\'ykora, T.~Holder, W.~Metzner, Phys. Rev. B \textbf{97}, 155159 (2018).
\newblock \doi{10.1103/PhysRevB.97.155159}.
\newblock \urlprefix\url{https://link.aps.org/doi/10.1103/PhysRevB.97.155159}

\bibitem{ips-2kf}
I.~Mandal, Nucl. Phys. B \textbf{1005}, 116586 (2024).
\newblock \doi{10.1016/j.nuclphysb.2024.116586}

\bibitem{ips-cavity}
I.~{Mandal}, Ann. Phys. \textbf{474}, 169925 (2025).
\newblock \doi{10.1016/j.aop.2025.169925}

\bibitem{ips-rafael}
I.~Mandal, R.M. Fernandes, Phys. Rev. B \textbf{107}, 125142 (2023).
\newblock \doi{10.1103/PhysRevB.107.125142}.
\newblock \urlprefix\url{https://link.aps.org/doi/10.1103/PhysRevB.107.125142}

\bibitem{ips-moire-cdw}
I.~{Mandal}, Annals of Physics \textbf{474}, 169925 (2025).
\newblock \doi{https://doi.org/10.1016/j.aop.2025.169925}.
\newblock
  \urlprefix\url{https://www.sciencedirect.com/science/article/pii/S0003491625000065}

\bibitem{ips-subir}
A.~Eberlein, I.~Mandal, S.~Sachdev, Phys. Rev. B \textbf{94}, 045133 (2016).
\newblock \doi{10.1103/PhysRevB.94.045133}.
\newblock \urlprefix\url{http://link.aps.org/doi/10.1103/PhysRevB.94.045133}

\bibitem{ips-c2}
I.~Mandal, Annals of Physics \textbf{376}, 89  (2017).
\newblock \doi{https://doi.org/10.1016/j.aop.2016.11.009}.
\newblock
  \urlprefix\url{http://www.sciencedirect.com/science/article/pii/S0003491616302585}

\bibitem{ips-hermann}
I.~Mandal, H.~Freire, Phys. Rev. B \textbf{103}, 195116 (2021).
\newblock \doi{10.1103/PhysRevB.103.195116}.
\newblock \urlprefix\url{https://link.aps.org/doi/10.1103/PhysRevB.103.195116}

\bibitem{ips-hermann2}
H.~Freire, I.~Mandal, Physics Letters A \textbf{407}, 127470 (2021).
\newblock \doi{https://doi.org/10.1016/j.physleta.2021.127470}.
\newblock
  \urlprefix\url{https://www.sciencedirect.com/science/article/pii/S0375960121003340}

\bibitem{ips-hermann3}
I.~Mandal, H.~Freire, Journal of Physics: Condensed Matter \textbf{34}(27),
  275604 (2022).
\newblock \doi{10.1088/1361-648x/ac6785}.
\newblock \urlprefix\url{https://doi.org/10.1088/1361-648x/ac6785}

\bibitem{ips-hermann-review}
I.~Mandal, H.~Freire, Journal of Physics: Condensed Matter \textbf{36}(44),
  443002 (2024).
\newblock \doi{10.1088/1361-648X/ad665e}.
\newblock \urlprefix\url{https://dx.doi.org/10.1088/1361-648X/ad665e}

\bibitem{ips-sudip1}
S.B. Chung, I.~Mandal, S.~Raghu, S.~Chakravarty, Phys. Rev. B \textbf{88},
  045127 (2013).
\newblock \doi{10.1103/PhysRevB.88.045127}.
\newblock \urlprefix\url{http://link.aps.org/doi/10.1103/PhysRevB.88.045127}

\bibitem{ips-sc}
I.~Mandal, Phys. Rev. B \textbf{94}, 115138 (2016).
\newblock \doi{10.1103/PhysRevB.94.115138}

\bibitem{ips-sudip2}
Z.~Wang, I.~Mandal, S.B. Chung, S.~Chakravarty, Annals of Physics \textbf{351},
  727  (2014).
\newblock \doi{http://dx.doi.org/10.1016/j.aop.2014.09.021}.
\newblock
  \urlprefix\url{http://www.sciencedirect.com/science/article/pii/S0003491614002814}

\bibitem{Abrikosov}
A.A. Abrikosov, Sov. Phys.-JETP \textbf{39}, 709 (1974)

\bibitem{moon-xu}
E.G. Moon, C.~Xu, Y.B. Kim, L.~Balents, Phys. Rev. Lett. \textbf{111}, 206401
  (2013).
\newblock \doi{10.1103/PhysRevLett.111.206401}.
\newblock
  \urlprefix\url{https://link.aps.org/doi/10.1103/PhysRevLett.111.206401}

\bibitem{rahul-sid}
R.M. Nandkishore, S.A. Parameswaran, Phys. Rev. B \textbf{95}, 205106 (2017).
\newblock \doi{10.1103/PhysRevB.95.205106}.
\newblock \urlprefix\url{https://link.aps.org/doi/10.1103/PhysRevB.95.205106}

\bibitem{ips-rahul}
I.~{Mandal}, R.M. {Nandkishore}, Pys. Rev. B \textbf{97}(12), 125121 (2018).
\newblock \doi{10.1103/PhysRevB.97.125121}

\bibitem{malcolm-bitan}
B.~Roy, M.P. Kennett, K.~Yang, V.~Juri\ifmmode \check{c}\else
  \v{c}\fi{}i\ifmmode~\acute{c}\else \'{c}\fi{}, Phys. Rev. Lett. \textbf{121},
  157602 (2018).
\newblock \doi{10.1103/PhysRevLett.121.157602}.
\newblock
  \urlprefix\url{https://link.aps.org/doi/10.1103/PhysRevLett.121.157602}

\bibitem{ips-birefringent}
I.~Mandal, Physics Letters A \textbf{418}, 127707 (2021).
\newblock \doi{https://doi.org/10.1016/j.physleta.2021.127707}.
\newblock
  \urlprefix\url{https://www.sciencedirect.com/science/article/pii/S0375960121005715}

\bibitem{hertz}
J.A. Hertz, Phys. Rev. B \textbf{14}, 1165 (1976).
\newblock \doi{10.1103/PhysRevB.14.1165}.
\newblock \urlprefix\url{https://link.aps.org/doi/10.1103/PhysRevB.14.1165}

\bibitem{millis}
A.J. Millis, Phys. Rev. B \textbf{48}, 7183 (1993).
\newblock \doi{10.1103/PhysRevB.48.7183}.
\newblock \urlprefix\url{https://link.aps.org/doi/10.1103/PhysRevB.48.7183}

\bibitem{ALTSHULER}
B.L. Altshuler, L.B. Ioffe, A.J. Millis, Phys. Rev. B \textbf{50}, 14048
  (1994).
\newblock \doi{10.1103/PhysRevB.50.14048}.
\newblock \urlprefix\url{http://link.aps.org/doi/10.1103/PhysRevB.50.14048}

\bibitem{polchinski}
J.~{Polchinski}, Nucl. Phys. B \textbf{422}, 617 (1994).
\newblock \doi{10.1016/0550-3213(94)90449-9}

\bibitem{ybkim}
E.A. {Kim}, M.J. {Lawler}, P.~{Oreto}, S.~{Sachdev}, E.~{Fradkin}, S.A.
  {Kivelson}, Phys. Rev. B \textbf{77}(18), 184514 (2008).
\newblock \doi{10.1103/PhysRevB.77.184514}

\bibitem{SSLee-largeN}
S.S. Lee, Phys. Rev. B \textbf{80}, 165102 (2009).
\newblock \doi{10.1103/PhysRevB.80.165102}

\bibitem{nayak}
C.~{Nayak}, F.~{Wilczek}, Nucl. Phys. B \textbf{430}, 534 (1994).
\newblock \doi{10.1016/0550-3213(94)90158-9}

\bibitem{mross}
D.F. Mross, J.~McGreevy, H.~Liu, T.~Senthil, Phys. Rev. B \textbf{82}, 045121
  (2010).
\newblock \doi{10.1103/PhysRevB.82.045121}

\end{thebibliography}
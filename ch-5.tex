\chapter{Non-Fermi liquid behaviour at the onset of incommensurate charge-density-wave order}
\label{chap-nfl-2kf}


\abstract{We study the onset of a quantum phase transition in a two-dimensional 
metal, from a conventional Fermi liquid to an incommensurate 
charge-density-wave (CDW) ordered phase, and show that it harbors a 
stable non-Fermi liquid (NFL) fixed point. The CDW bosons 
couples fermions at a single pair of antipodal hot-spots on the Fermi 
surface (FS) connected by the nesting wavevector $\boldsymbol{\mathcal{Q}}$, 
at which the tangent vectors are antiparallel. To tame the strong 
coupling that prevails at the physical dimension $d = 2$, we extend the 
co-dimension of the FS to a continuously tunable value via dimensional 
regularization, while holding the FS dimension itself fixed at one. The 
coupling constant becomes marginal at the upper critical dimension 
$d_c = 5/2$, and a controlled expansion in the small parameter 
$\epsilon = d_c - d$ yields the critical exponents of the stable infrared 
fixed point. Returning to the physical theory by setting $\epsilon = 1/2$, 
we find that the fermionic self-energy acquires a characteristic 
fractional frequency dependence with exponent $2/3$ --- the hallmark of 
NFL behaviour at a quantum critical point.}


%%%%%%%%%%%%%%%%%%%%%%%%%%%%%
\section{Introduction}

At quantum critical points (QCPs) hosting non-Fermi liquids (NFLs), 
the critical order-parameter bosons fall into two broad categories. In 
the first, the bosonic field is centered at zero momentum, causing 
quasiparticles to lose coherence across the entire FS 
\cite{max-isn, Lee-Dalid, ips-uv-ir1, ips-uv-ir2, ips-sc}. In the 
second, the bosonic field is centered at a finite wavevector 
$\boldsymbol{\mathcal{Q}}$ that connects points on the FS --- commonly 
referred to as \textit{hot-spots} --- so that NFL behaviour emerges 
locally in their vicinity \cite{max-isn, chubukov1, Chubukov, shouvik2, 
ips-c2, andres1, andres2, ips-fflo}. The Ising-nematic critical point 
is a prominent example of the first category, while the second 
encompasses ordering transitions to phases such as spin-density wave 
(SDW), charge-density wave (CDW) \cite{max-sdw, chubukov1, Chubukov, 
shouvik2, ips-c2, andres1, andres2}, and 
Fulde-Ferrell-Larkin-Ovchinnikov (FFLO) states \cite{ips-fflo}.

CDW (SDW) bosons with $\boldsymbol{\mathcal{Q}} \neq 0$ drive 
instabilities toward charge (magnetic) order, in which the charge 
(spin) density spontaneously breaks translational symmetry and develops 
a density modulation at wavevector $\boldsymbol{\mathcal{Q}}$. These 
instabilities are further classified along two axes: whether the 
wavevector is commensurate or incommensurate with the underlying 
lattice, and whether $\boldsymbol{\mathcal{Q}}$ constitutes a nesting 
vector of the FS. Commensurate wavevectors can be expressed as linear 
combinations of the reciprocal lattice vectors $\{\boldsymbol{\mathcal{R}}\}$ 
with rational coefficients, while incommensurate ones cannot. Although 
there are infinitely many rational coefficients in principle, the 
quantitative effects of commensurability diminish as the size of their 
denominators grows. When $\boldsymbol{\mathcal{Q}}$ coincides with a 
nesting vector connecting two points on the FS with antiparallel Fermi 
velocities (equivalently, antiparallel tangent vectors), the spin and 
charge orderings are enhanced by a well-known singularity arising from 
the enlarged phase space available for low-energy particle-hole 
excitations. In an inversion-symmetric crystal with valence band 
dispersion $\xi(\mathbf{k})$, the nesting vectors $\boldsymbol{\mathcal{Q}}$ 
are determined by the condition $\xi(\boldsymbol{\mathcal{Q}}/2 + 
\mathbf{G}/2) = \xi_{k_F}$, where $\xi_{k_F}$ is the Fermi energy. 
The nesting-vector condition, combined with inversion symmetry, implies 
$|\boldsymbol{\mathcal{Q}}| = 2k_F$, where $k_F$ is the magnitude of 
the local Fermi momentum. This follows from the fact that the two 
hot-spots are related by inversion symmetry and therefore share the 
same value of $k_F$.

The $2k_F$ wavevector instabilities are ubiquitous in two-dimensional 
systems displaying high-temperature superconductivity. Notable examples 
include the SDW instability at a $2k_F$ wavevector in the ground state 
of the 2d Hubbard model at half-filling \cite{hubbard, hubbard2}, and 
d-wave bond charge order, triggered by antiferromagnetic fluctuations, 
in models for cuprate superconductors \cite{max-sdw, bond-order}. It is 
worth emphasizing that the nesting vector here causes only a 
\textit{partial} nesting of the FS, which is conceptually distinct from 
perfect nesting, wherein entire slices --- rather than discrete points 
--- of the FS are connected by the same wavevector. Experimental 
evidence for incommensurate CDW ordering has been reported in a range 
of materials, including NbSe$_2$ and TaS$_2$ \cite{salvo, Scholz}, 
VSe$_2$ \cite{pai}, SmTe$_3$ \cite{Gweon}, and TbTe$_3$ 
\cite{Kapitulnik}. In some of these compounds, the CDW transition 
temperature can be tuned toward zero by applying high pressure, 
revealing a putative QCP at the onset of CDW order \cite{littlewood}. 
These observations underscore the theoretical importance of understanding 
QCPs associated with incommensurate $2k_F$ wavevector instabilities.

In this chapter, we consider a pair of antipodal points on a 
one-dimensional FS of a two-dimensional (2d) metal, with parallel tangent 
vectors, interacting with an order-parameter boson whose condensation 
gives rise to an incommensurate CDW ordered phase \cite{metzner1, 
metzner2}. Right at the QCP, the CDW boson becomes massless, giving 
rise to strong quantum fluctuations that drive the system across a phase 
transition to an ordered state in which the electron density 
spontaneously breaks translational symmetry and develops a density 
modulation at a wavevector $\boldsymbol{\mathcal{Q}}$ incommensurate 
with the reciprocal lattice. For definiteness, and without loss of 
generality, we take $\boldsymbol{\mathcal{Q}} = 2k_F\,\hat{\boldsymbol{x}}$.

%%%%%%%%%%%%%%%%%%%%%%%%%%%%%%%%%%%%%%%%%
\section{Model}
\label{secmodel-2kf}

%%%%%%%%%%%%%%%%%%%
\begin{figure}[t!]
\centering
\includegraphics[width = 0.35 \textwidth]{nested_FS } 
\caption{Schematics of the one-dimensional Fermi surface, showing the two hot-spots 
connected by the wavevector $\boldsymbol{\mathcal{Q}} = 2k_F\,
\hat{\boldsymbol{x}}$ (blue arrow), which is incommensurate with the 
reciprocal lattice. The fermionic fields near the right and left 
hot-spots are labeled $\psi_+$ and $\psi_-$, respectively, and both 
couple to the CDW order-parameter bosonic fields with momenta centered 
at $\boldsymbol{\mathcal{Q}}$.}\label{fig_FScdw}
\end{figure}
%%%%%%%%%%%%%%%%%%%%%%%%%%%%%%%


Our starting point is the low-energy QFT action for finite-density 
fermions in two spatial dimensions, coupled to an incommensurate CDW 
order parameter centered at momentum $\boldsymbol{\mathcal{Q}} = 
2k_F\,\hat{\boldsymbol{x}}$. As shown in Fig.~\ref{fig_FScdw}, the 
nesting vector $\boldsymbol{\mathcal{Q}}$ connects a pair of hot-spots 
on the FS lying along the $x$-axis. The effective action governing the 
low-energy degrees of freedom near the hot-spots and the CDW 
order-parameter mode in $(2+1)$ dimensions reads \cite{metzner1, metzner2}
\begin{align}
S &= \sum_{s=\pm} \int_k \psi_s^\dagger(k)
\left(-i\,k_0 + s\,k_1 + k_2^2\right)\psi_s(k)
+ \int_k \phi_+(k)\left(k_0^2 + k_1^2 + k_2^2\right)\phi_-(-k) 
\nonumber \\
& \qquad + e\int_{k,q}\Big[\phi_+(q)\,\psi_+^\dagger(k+q)\,\psi_-(k)
+ \phi_-(-q)\,\psi_-^\dagger(k-q)\,\psi_+(k)\Big]\,,
\label{eqs0-2kf}
\end{align}
Here, $k = (k_0, \mathbf{k})$ denotes the three-vector comprising the 
Matsubara frequency $k_0$ and the spatial momentum $\mathbf{k} = 
(k_1, k_2) \equiv (k_x, k_y)$, with $\int_k \equiv \int 
dk_0\,d^d\mathbf{k}/(2\pi)^{d+1}$ and $d = 2$ spatial dimensions. 
The fermionic degrees of freedom near the right and left hot-spots are 
represented by $\psi_+(k)$ and $\psi_-(k)$, respectively, while 
$\phi_+(k)$ and $\phi_-(k)$ denote the bosonic fluctuations carrying 
frequency $k_0$ and momenta $\boldsymbol{\mathcal{Q}} + \mathbf{k}$ 
and $-\boldsymbol{\mathcal{Q}} + \mathbf{k}$. The bosons are massless, 
reflecting the fact that we are working directly at the QCP. The 
fermionic momenta have been rescaled so that the Fermi velocity has 
unit magnitude and the FS curvature at the hot-spots equals $2$. The 
bare bosonic velocity, although generically distinct from its fermionic 
counterpart, can be absorbed into a field redefinition and is accordingly 
set to unity. We further note that, at the critical point, the low-energy 
bosonic dynamics is dominated by particle-hole excitations of the FS, 
rendering the precise value of the bosonic velocity immaterial in the 
infrared (IR).

Since the FS is locally parabolic, we assign the scaling dimensions of $1$ 
and $1/2$ to $k_1$ and $k_2$, respectively. To extract the critical 
scalings in a controlled way, we extend the co-dimension of the FS 
\cite{senshank, Lee-Dalid, shouvik2} to identify the upper critical 
dimension, $d = d_c$, at which the fermionic self-energy develops 
a logarithmic singularity. Maintaining analyticity in momentum space --- 
or equivalently, locality in real space --- for generic co-dimensions 
requires introducing the two-component spinors \cite{Lee-Dalid, 
ips-uv-ir1, ips-uv-ir2, ips-fflo, ips-u1, ips-rafael},
\begin{align}
\Psi(k) = \left(\psi_+(k) \quad \psi_-^\dagger(-k)\right)^T 
\quad \text{and} \quad \bar{\Psi} \equiv \Psi^\dagger\,\gamma_0\,.
\end{align}
Un terms of these spinor, the action describing the one-dimensional FS embedded 
in a $d$-dimensional momentum space, takes the form of
\begin{align}
\label{eqs1}
S &= \int_k \bar{\Psi}(k)\,i\left(\vec{\Gamma}\cdot\vec{K} 
+ \gamma_{d-1}\,\delta_k\right)\Psi(k)
+ \int_k\left(k_d^2 + \tilde{a}\,e_k\right)\phi_+(k)\,\phi_-(-k) 
\nonumber \\
& \quad - \frac{i\,e\,\mu^{x/2}}{2}\int_{k,q}
\Big[\phi_+(q)\,\bar{\Psi}(k+q)\,\gamma_0\,\bar{\Psi}^T(-k)
- \phi_-(-q)\,\Psi^T(q-k)\,\gamma_0\,\Psi(k)\Big]\,, \nonumber \\
x &= \frac{5}{2} - d\,, \quad \delta_k = k_{d-1} + k_d^2\,, \quad 
e_k = k_{d-1} + \frac{k_d^2}{2}\,.
\end{align}
The $(d-1)$-component vector, $\vec{K} \equiv (k_0, k_1, \ldots, 
k_{d-2})$, collects the frequency together with the $(d-2)$ momentum 
components arising from the added co-dimensions. The original momentum 
components along the $x$- and $y$-directions have been relabelled 
$k_{d-1}$ and $k_d$, respectively, so that within the $d$-dimensional 
momentum space, $\{k_1, \ldots, k_{d-1}\}$ spans the $(d-1)$ directions 
perpendicular to the FS while $k_d$ runs parallel to it. The matrix 
vector $\vec{\Gamma} \equiv (\gamma_0, \gamma_1, \ldots, \gamma_{d-2})$ 
carries $(d-1)$ components representing the gamma matrices associated 
with $k_0$ and the extra co-dimensions. Since the goal is ultimately to 
continue to $d = 2$, it suffices throughout to work with the $2\times 2$ 
gamma matrices $\gamma_0 = \sigma_y$ and $\gamma_{d-1} = \sigma_x$.

In the purely bosonic sector, only the $k_d^2$ term in the kinetic 
energy survives, since $(|\vec{K}|^2 + k_{d-1}^2)$ is irrelevant under 
the patch-theory scaling \cite{max-isn, Lee-Dalid, ips-uv-ir1, 
ips-uv-ir2, ips-fflo, ips-u1}, wherein each component of 
$\{\mathbf{K},\,k_{d-1}\}$ carries dimension one and $k_d$ carries 
dimension $1/2$. A dependence on $e_k$ in the bosonic propagator is 
generated dynamically through the susceptibility, driven by strong 
particle-hole fluctuations. We have therefore already included the term 
$\tilde{a}\,e_k$ in the action, whose form is fixed by the divergent 
contribution to the one-loop susceptibility derived below [see 
Eq.~\eqref{eqpi}]. This term carries the same mass dimension as $k_d^2$, 
while $\tilde{a}$ itself has vanishing engineering dimension. Its 
omission would produce infrared divergences in loop integrals involving 
the bosonic propagator --- spurious artifacts of truncating the effective 
action to the $k_d^2$ term alone. The engineering dimension of the 
fermion-boson coupling $e$ is $x/2$, which is why we have introduced an 
explicit factor of $\mu^{x/2}$: it renders $e$ dimensionless, as is 
standard practice in QFT calculations.

A noteworthy contrast with the Ising-nematic case (discussed in Chapter 2) is worth drawing here. 
In that setting, an emergent sliding symmetry \cite{max-isn, Lee-Dalid, 
ips-uv-ir1, ips-uv-ir2} forces the terms proportional to 
$\bar{\Psi}(k)\,k_{d-1}\,\Psi(k)$ and $\bar{\Psi}(k)\,k_d^2\,\Psi(k)$ 
to renormalize in lockstep, so that the fermionic propagator depends on 
$k_{d-1}$ and $k_d^2$ only through the combination $\delta_k$, even 
after loop corrections are included. No such symmetry is operative in 
the present problem, and there is no a priori reason why sole dependence 
on $\delta_k$ should be preserved under renormalization. In principle, 
the corrected terms could conspire to flatten the FS at the hot-spots, 
as reported in the RPA calculations of Ref.~\cite{metzner2}. As far as our 
explicit one-loop computations will show, however, no such flattening 
occurs at one-loop order order of our QFT analysis. The flattening might show up when one computes higher-loop Feynman diagrams.


%%%%%%%%%%%%%%%%%%%%%%%%%%%%
\section{One-loop self-energies and implementation of dimensional regularization}
\label{secselfen}

The value of $x$ reveals that the coupling constant $e$ becomes marginal 
precisely at the upper critical dimension $d_c = 5/2$, remaining 
relevant for $d < 5/2$ and irrelevant for $d > 5/2$. We therefore seek 
to access the interacting phase through a perturbative expansion about 
$d = 5/2 - \epsilon$, where $\epsilon$ serves as the small control 
parameter, and the physical two-dimensional theory is ultimately 
recovered by setting $\epsilon = 1/2$. As a preparatory step toward 
deriving the RG flow equations, we compute the one-loop self-energies 
of the bosonic and fermionic sectors, which supply the key ingredients 
needed to determine the beta functions of the coupling constants $e$ 
and $\tilde{a}$. The bare fermionic and bosonic propagators, that follow 
from the action in Eq.~\eqref{eqs1}, are
\begin{align}
 &G_{(0)} (k) \equiv  \left\langle \Psi(k)\,  \bar{\Psi}(k) \right\rangle_0 
 = \frac{1} {i}\,\frac{\boldsymbol{\Gamma} \cdot  {\mathbf{K}}  + \gamma_{d-1}\, \delta_k}
{ |\mathbf K|^2 + \delta_k^2} \,, \nn &
%%%%%%%%%%%%
 D_{(0)}^+(k) \equiv  \left\langle \phi_+(k) \, \phi_-(-k) \right\rangle_0 
 = \frac{1}{k_d^2 + \tilde a\, e_k } \,,
  \text{and} 
 \nn &  D_{(0)}^- (k)  \equiv  \left\langle \phi_-(k)\, \phi_+(-k) \right\rangle_0 = D_{(0)}^+(k) \,.
\end{align}  

%%%%%%%%%%%%%%%%%%%
\begin{figure}[t!]
\centering
\subfigure[]{\includegraphics[scale=0.25]{boscdw} }\quad
\subfigure[]{\includegraphics[scale=0.32]{fermicdw} } \quad
\subfigure[]{\includegraphics[scale=0.27]{vert1cdw} 
\includegraphics[scale=0.27]{vert2cdw} }
\caption{One-loop Feynman diagrams for the (a) bosonic self-energy, (b) fermionic 
self-energy, and (c) fermion-boson vertices. All the fermion propagators 
are represented by arrowed solid lines corresponding to the bare 
Green's function, $G_{(0)}$, while the dressed bosonic propagator, $D_{(1)}$, 
is depicted by wiggly lines.}\label{figcdw}
\end{figure}


%%%%%%%%%%%%%%%%%%%%%%%%%%%%%%%%%%%%%
\subsection{One-loop bosonic self-energy}
\label{secbos}

The one-loop bosonic self-energy [cf.\ Fig.~\ref{figcdw}(a)] is defined by
\begin{align}
\Pi(k) = -\frac{e^2\,\mu^x}{4}\times 2
\int_q \text{Tr}\left[\gamma_0\,G_{(0)}(q)\,\gamma_0\,G_{(0)}^T(k-q)\right].
\label{eqbos2}
\end{align}
Applying the commutation relations between the gamma matrices alongside 
the identities $\gamma_{d-1}^T = -\gamma_0\,\gamma_{d-1}\,\gamma_0$ 
and $\Gamma^T = -\gamma_0\,\Gamma\,\gamma_0$, this simplifies to
\begin{align}
\Pi(k) = e^2\,\mu^x \int_q
\frac{\vec{Q}\cdot(\vec{Q} - \vec{K}) - \delta_q\,\delta_{k-q}}
{\left(\vec{Q}^2 + \delta_q^2\right)
\left[(\vec{Q} - \vec{K})^2 + \delta_{k-q}^2\right]}\,.
\end{align}
Using $\delta_{k-q} = k_{d-1} + q_{d-1} + (k_d - q_d)^2$, we shift 
$q_{d-1} \rightarrow q_{d-1} - q_d^2$ and apply Feynman parametrization, 
arriving at
\begin{align}
\Pi(k) &= e^2\,\mu^x \int_q \int_0^1 dt\,
\frac{|\mathbf{Q}|^2 - t(1-t)|\mathbf{K}|^2 
- \tilde{e}_{kq}\,q_{d-1} + q_{d-1}^2}
{\left[|\mathbf{Q}|^2 + t(1-t)|\mathbf{K}|^2 + t\,\tilde{e}_{kq}^2 
+ q_{d-1}^2 - 2t\,\tilde{e}_{kq}\,q_{d-1}\right]^2} \nonumber \\
&= e^2\,\mu^x \int d^{d-1}|\mathbf{Q}|\,dq_d \int_0^1 dt\,
\frac{|\mathbf{Q}|^d}
{2^d\,\pi^{\frac{d+1}{2}}\,\Gamma\!\left(\frac{d-1}{2}\right)
\left[|\mathbf{Q}|^2 + t(1-t)\left(\tilde{e}_{kq}^2 
+ |\mathbf{K}|^2\right)\right]^{3/2}} \nonumber \\
&= e^2\,\mu^x \int dq_d\,
\frac{2^{1-2d}\csc\!\left(\frac{d\pi}{2}\right)
\left(\tilde{e}_{kq}^2 + |\mathbf{K}|^2\right)^{\frac{d-2}{2}}}
{\pi^{\frac{d-1}{2}}\,\Gamma\!\left(\frac{d-1}{2}\right)}\,.
\end{align}
The substitution $u = \sqrt{2}\,q_d - k_d/\sqrt{2}$, with Jacobian 
$1/\sqrt{2}$, then gives
\begin{align}
\Pi(k) = e^2\,\mu^x \int_0^\infty du\,
\frac{2^{\frac{3}{2}-2d}\csc\!\left(\frac{d\pi}{2}\right)
\left[\left(u^2 + e_k\right)^2 + |\mathbf{K}|^2\right]^{\frac{d-2}{2}}}
{\pi^{\frac{d-1}{2}}\,\Gamma\!\left(\frac{d-1}{2}\right)}\,.
\end{align}
Since the bare susceptibility diverges at zero temperature at the nesting 
vector $\boldsymbol{\mathcal{Q}}$, a well-defined self-energy is obtained 
by subtracting off the singular zero-momentum contribution:
\begin{align}
\label{eqtildepi}
\tilde{\Pi}(k) = \Pi(k) - \Pi(0)
= \frac{2^{\frac{3}{2}-2d}\csc\!\left(\frac{d\pi}{2}\right)
e^2\,\mu^x}
{\pi^{\frac{d-1}{2}}\,\Gamma\!\left(\frac{d-1}{2}\right)}
\,I_\Pi(k,d)\,,
\end{align}
Upon the further substitution, $z = u^2 + e_k$, we get
\begin{align}
I_{\Pi} (k,d) & \equiv \int_{e_k}^\infty dz\,
\frac{
\left ( z-{  e_k} \right )^{2-d}
-
\left[ z^2+ |\mathbf K|^2\right]^{\frac{2-d}{2}}
}
{ 2\,\sqrt{z-{  e_k}}
\, \left[ z^2+ |\mathbf K|^2\right]^{\frac{2-d}{2}}
\, \left ( z-{  e_k} \right )^{2-d}
}\nn
%%%%%%%%%%%%%%%%%%%%%%%%%%%%%%%%%%%
& =\begin{cases}
\frac{ \Gamma (d-1) \,
 \left(-e_k\right)^{d-\frac{3}{2}}}
{ 2 } 
\left [ \frac{\Gamma \left(\frac{3-d}{2}\right) \, 
\, _2F_1\left(\frac{ 3-2 \,d}{4} ,
\frac{5-2 \,d}{4} ;\frac{3-d}{2};
-\frac{|\mathbf K|^2}{ { e_k}^2}\right)}{\sqrt{\pi }}
+ \frac{_2 {F}_1\big (\frac{1}{2},1;d;1 \big) } {\Gamma(d)}
\right]  & \\
%%%%%%%%
+ \, \frac{ |\mathbf K|^d \,\,
 _3F_2\big (\frac{3}{4},1,\frac{5}{4};\frac{3}{2},\frac{d}{2}+1;
 -{|\mathbf K|^2}/ { { e_k}^2}\big )}
{  4 \,d\, \left(- { e_k} \right)^{3/2}}
+
\frac{\pi^{3/2} \,
|\mathbf K|^{d-1} \sec \left(\frac{ d\,\pi} {2}\right)  
\, \,  _2 {F}_1 \big(\frac{1}{4},\frac{3}{4};\frac{d+1}{2};
-{|\mathbf K|^2} /{ e_k^2}\big)}
{ 4\, \sqrt{- { e_k}  } \,\,\Gamma \left(\frac{2-d}{2}\right)
\, \Gamma \left (\frac{d+1}{2} \right )
} & \\
%%%%%%%%%%%%%  
+ \, |\mathbf K|^{d-2} \,\sqrt{-e_k} \, \,
_3F_2\left(\frac{1}{2},1,1-\frac{d}{2};\frac{3}{4},\frac{5}{4};-\frac{e_k^2} {|\mathbf K|^2}\right)
+ \frac{ \left(-e_k\right)^{d-\frac{3}{2}} }
{3-2\, d}
 \text{ for }  { e_k}<0 &
%%%%%%%%%%%%%%%%%%%%%%
%%%%%%%%%%%%%%%%%%%%%%%%%%%%
\\ & \\ & \\
\frac{\sqrt{\pi } \,  e_k^{ d - \frac{3}{2}} 
\, \Gamma \left(\frac{3-d}{2}\right) \, \,
_2F_1 \big (\frac{3-2 \,d}{4} ,\frac{5-2 \,d}{4} ;\frac{3-d}{2};-
 |\mathbf K|^2 / { {e_k}^2} \big )} 
 {2\, \Gamma (2-d)} \text{ for }   e_k  > 0\,.&
\end{cases} 
\end{align} 
%%%%%%%%%%%%%
The zeros of $1/\Gamma\!\left(\frac{2-d}{2}\right)$ and $1/\Gamma(2-d)$ 
at $d = 2$ are cancelled by the factor $\csc\!\left(\frac{d\pi}{2}\right)$ 
in Eq.~\eqref{eqtildepi}, making overall term non-singular for $d=2$.


In the limit $|\mathbf{K}|^2/e_k^2 \ll 1$, the leading-order behaviour is
\begin{align}
I_\Pi(k,d) = \begin{cases}
%%%%%%%%%%%%%%%%
\dfrac{\Gamma(d-1)(-e_k)^{d-\frac{3}{2}}}{2}
\left[\dfrac{_2\tilde{F}_1\!\left(\frac{1}{2},1;d;1\right)}{\Gamma(d)}
+ \dfrac{\Gamma\!\left(\frac{3}{2}-d\right)}{\sqrt{\pi}}
+ \dfrac{\sqrt{\pi}}{\Gamma\!\left(d-\frac{1}{2}\right)}\right] & \\[8pt]
%%%%%%%%%%%%%%%%%%%
+\, \dfrac{(-e_k)^{d-\frac{3}{2}}}{3-2d}
+ \dfrac{\pi^{3/2}|\mathbf{K}|^{d-1}\sec\!\left(\frac{d\pi}{2}\right)}
{2\sqrt{-e_k}\,\Gamma\!\left(\frac{2-d}{2}\right)
\Gamma\!\left(\frac{d+1}{2}\right)}
%%%%%%%%%%%%%%%%%%%%%%%%%%%
+ \mathcal{O}\!\left(\dfrac{|\mathbf{K}|^d}{(-e_k)^{3/2}}\right)
 \text{for } e_k <  0 & \\[8 pt] & \\[8 pt]
%%%%%%%%%%%%%%%%%%%%%%%%%%%%%%%%%%
\dfrac{\sqrt{\pi}\,\Gamma\!\left(\frac{3}{2}-d\right)
e_k^{d-\frac{3}{2}}}{2\,\Gamma(2-d)}
+ \dfrac{\sqrt{\pi}\,\Gamma\!\left(\frac{7}{2}-d\right)|\mathbf{K}|^2}
{4\,(d-3)\,\Gamma(2-d)\,e_k^{\frac{7-2d}{2}}}
 & \\[8 pt]
+ \mathcal{O}\!\left(\dfrac{|\mathbf{K}|^4}{e_k^{11/2-d}}\right)
 \text{for } e_k >  0 \,.&
\end{cases}
\end{align}
%%%%%%%%%%%%%%%%%%%%%
The poles of $\Gamma(3/2 - d)$ and $1/(3-2d)$ at $d = 3/2$ signal, 
respectively, a logarithmic divergence at $d = 3/2$ and a linear 
divergence at $d = 5/2$ in the Wilsonian language of a momentum cutoff 
$\Lambda \sim \mu$. Within dimensional regularization, UV divergences of 
all orders appear as poles of $\Gamma$-functions, and reinstating an 
explicit Wilsonian cutoff $\Lambda$ allows one to read off their degree. 
While these terms are important for the analysis of UV-stable fixed 
points, they must be set aside in the present context, as we are focused 
on the infrared RG flows in $d = 5/2 - \epsilon$ dimensions, where they 
correspond to IR-irrelevant operators. The situation is closely analogous 
to that of a $\phi^6$ interaction added to a $\phi^4$ scalar field theory 
in $(3+1)$ dimensions: the $\phi^4$ vertex is renormalizable with upper 
critical dimension $4$, whereas the $\phi^6$ vertex has upper critical 
dimension $3$, and its presence destroys renormalizability in four 
spacetime dimensions.

Expanding in $\epsilon$ about $d = 5/2 - \epsilon$ and dropping the 
terms with poles at $d = 3/2$, we find
\begin{align}
& \left[\mu^x\,\frac{2^{\frac{3}{2}-2d}\csc\!\left(\frac{d\pi}{2}\right)}
{\pi^{\frac{d-1}{2}}\,\Gamma\!\left(\frac{d-1}{2}\right)}
\times I_\Pi(k,d)\right]\Bigg|_{d=\frac{5}{2}-\epsilon}
\nn
&= \begin{cases}
-\,\dfrac{\pi^{3/4}\,\dfrac{|\mathbf{K}|^{3/2}}{\sqrt{|e_k|}}
\left(\dfrac{\mu}{|\mathbf{K}|}\right)^\epsilon}
{32\sqrt{2}\,\Gamma^2(3/4)\,\Gamma(7/4)}
+ \mathcal{O}(\epsilon)
& \text{for } e_k < 0 \\[12pt]
-\,\dfrac{\dfrac{|\mathbf{K}|^2}{e_k}
\left(\dfrac{\mu}{e_k}\right)^\epsilon}
{32\pi^{3/4}\,\Gamma(3/4)}
+ \mathcal{O}(\epsilon)
& \text{for } e_k > 0 \,.
\end{cases}\nonumber
\end{align}
The leading-order self-energy correction in Eq.~\eqref{eqtildepi} 
therefore takes the form,
\begin{align}
\label{eqpi}
\tilde{\Pi}(k) = -\,\frac{e^2\,\mu^x\,b\,|\mathbf{K}|^{d-1}}
{\sqrt{|e_k|}}\,\Theta(-e_k)\,,
\quad \text{where} \quad
b = \frac{\pi^{3/4}}
{32\sqrt{2}\,\Gamma^2(3/4)\,\Gamma(7/4)}\,,
\end{align}
in the limit $|\mathbf{K}|^2/e_k^2 \ll 1$.

Since the bare bosonic propagators $D_{(0)}^\pm(k)$ are independent of 
$\mathbf{K}$, loop integrals involving them are ill-defined without 
resumming a class of diagrams that endows the propagator with a 
nontrivial dispersion in these directions. We therefore dress the 
propagator by incorporating the finite one-loop correction 
$\tilde{\Pi}(k) \propto |\mathbf{K}|^{d-1}/\sqrt{|e_k|}$ in all 
subsequent loop calculations. For both $\phi_+(k)$ and $\phi_-(k)$, 
this amounts to working with
\begin{align}
\label{eqbos1}
D_{(1)}(k) = \frac{1}{\left[D_{(0)}^+(k)\right]^{-1} - \tilde{\Pi}(k)}
= \frac{1}{k_d^2 + \dfrac{b\,e^2\,\mu^x\,|\mathbf{K}|^{d-1}
\,\Theta(-e_k)}{\sqrt{|e_k|}}}\,.
\end{align}
This reorganization of the perturbative expansion ensures that the 
$\mathbf{K}$-dependent finite part of the one-loop bosonic self-energy 
is already accounted for at the zeroth order. The term $\tilde{\Pi}(k)$ is 
the well-known \textit{Landau-damping} contribution, which underlies the 
characteristic $\text{sgn}(k_0)|k_0|^{2/3}$ frequency dependence of 
the fermionic self-energy --- a robust signature of NFL behaviour that 
has been identified at the QCPs in a broad range of 
strongly-correlated systems
\cite{max-isn, max-sdw, Lee-Dalid, ips-uv-ir1, ips-uv-ir2, ips-fflo, ips-u1}.


%%%%%%%%%%%%%%%%%%%%%%%%%%%%%%%%%%%%%
\subsection{One-loop fermion self-energy}
\label{secferm}

Turning to the fermionic sector, the one-loop self-energy 
[cf.\ Fig.~\ref{figcdw}(b)] is expressed as the integral
\begin{align}
\label{eqFS 0}
\Sigma(k) & =  e^2 \,\mu^{x}
\int_{q} \,\gamma_0\, G_{(0)}^T(q-k) \,\gamma_0\, D_{(1)}(q)  
%%%%%%%%%%%%
 = i\,\Sigma_1 (k) \, \mathbf{\Gamma} \cdot \mathbf K
+ i\,\Sigma_2 (k) \,\gamma_{d-1} \,,
\end{align}
where
\begin{align}
\Sigma_1 (k) = - \frac{e^2 \,\mu^{x} } { |\mathbf K|^2 }
\int_q \frac{ \mathbf{K} \cdot (\mathbf Q- {\mathbf{K}} ) 
}
{(\mathbf Q- {\mathbf{K}} )^2 
+ \delta^2_{q-k}} \, D_{(1)}(q)
\end{align}
and
\begin{align}
\label{eqFS 2}
\Sigma_2 (k) =  e^2 \,\mu^{x} 
\int_q \frac{ \delta_{q-k} }
{(\mathbf Q- {\mathbf{K}} )^2 + \delta^2_{q-k}} \, D_{(1)}(q)\,.
\end{align}
The evaluation of these two contributions is detailed in the following 
two subsections, which the reader may wish to skip if they prefer to 
bypass the more involved intermediate steps. For convenience, we 
collect the final results here. Setting $d = d_c - \epsilon$, the 
singular part evaluates to
\begin{align}
\Sigma(k) & =- 
 \frac{ e^{4/3} \, \, {\mathcal U}_1 \, } 
{ \left (2- \tilde a  \right )^{2/3}
\,\epsilon} 
\, i\left( \mathbf{\Gamma} \cdot \mathbf K \right)
+\order{\epsilon^0} ,\quad
%%%%%%%%%%%%%%%%
{\mathcal U}_1 = \frac{\sqrt{2} \,\,
 \Gamma \big (\frac{5}{4}\big)}
 {3\, \sqrt 3 \, \pi^{7/4} \, b^{1/3}  }\,,
\label{eqferm1}
\end{align}
where the logarithmic divergence is reflected by a pole at $  \epsilon =0 $.


 



%%%%%%%%%%%%%%%%%%%%%%%%%%%%%%%%%%%%
\subsubsection{Computation of $\mathbf \Gamma$-dependent part}
%%%%%%%%%%%%%%%%%%%%%%%%%%%%%%%%%%%%%

The leading-order dependence of $\Sigma_1(k)$ on $\mathbf{K}$ is captured 
by setting the external momentum components $k_d$ and $k_{d-1}$ to 
zero, reducing the problem to evaluating
\begin{align}
& \Sigma_1 (\mathbf K, 0,0)  \nn & =  \frac{ e^2 \,\mu^{x} } 
{ |\mathbf K|^2 }
\int_q \frac{ \mathbf{K} \cdot (\mathbf K - {\mathbf{Q}} ) 
}
{(\mathbf Q- {\mathbf{K}} )^2 
+  \delta_q^2 }  \times
\frac{1}
{ q_d^2 +  e^2\,\mu^{x} \, b\, |\mathbf Q |^{d-1} \,\Theta(- e_q)
 /\sqrt{| e_q|}  } \,.
\end{align}
%%%%%%%%%%%%%%%%%%%555
Switching to $q_d$ and $e_q$ as integration variables, and splitting 
the integration domain into the regions $e_q < 0$ and $e_q > 0$ via 
$\Sigma_1(\mathbf{K}, 0, 0) = I_1 + I_2$, we obtain
%%%%%%%%%%%%%%%%%%%% 
\begin{align}
I_1 & = \frac{ e^2 \,\mu^{x} } 
{ |\mathbf K|^2 }
\int_{ e_q<0} \frac{d^{d-1} \mathbf{Q}\, dq_d \,d e_q
}
{ (2\,\pi)^{d+1} }  \,
 \frac{ - \mathbf{K} \cdot (\mathbf Q - \mathbf K ) }
{(\mathbf Q- {\mathbf{K}} )^2 
+ \left(  e_q+ q_d^2/2  \right)^2
}  \,
\frac{1}
{ q_d^2  + e^2 \,\mu^{x} \, b \,
|\mathbf Q |^{d-1}  /\sqrt{| e_q|}  } \nn
%%%%%%%%%%%%%%%%%%%
 & =  \frac{ e^2 \,\mu^{x} } 
{ |\mathbf K|^2 }
 \int_0^\infty \frac{ du} {\sqrt { u/2} }
 \int_0^\infty d e_q
  \int_{-\infty}^{\infty}
  \frac{d^{d-1} \mathbf{Q}
}
{  (2\,\pi)^{d+1} }
 \frac{ \mathbf K^2 -\mathbf K \cdot \mathbf Q
}
{(\mathbf Q- {\mathbf{K}} )^2 
+   \left( u- e_q  \right)^2
}  
\nn & \hspace{5 cm } 
%%%%%%%%%%
\times \frac{1}
{ 2\, u  + e^2\,\mu^{x}\, b \,|\mathbf Q |^{d-1} /\sqrt{ e_q} 
}  
\end{align} 
and
%%%%%%%%%%%%%%%%%%%% 
\begin{align}
I_2 & =  \frac{ e^2 \,\mu^{x} } 
{ |\mathbf K|^2 }
\int_{ e_q>0} \frac{d^{d-1} \mathbf{Q}\, dq_d \,d e_q
}
{ (2\,\pi)^{d+1} }  \,
 \frac{ - \mathbf{K} \cdot (\mathbf Q - \mathbf K ) 
}
{(\mathbf Q- {\mathbf{K}} )^2 
+ \left(  e_q+ q_d^2/2  \right)^2
}  \,
\frac{1}
{ q_d^2 }
%%%%%%%%%%%%%
\nn & = 
\frac{ e^2 \,\mu^{x} } 
{ |\mathbf K|^2 }
\int_{ e_q>0} \frac{d^{d-1} \mathbf{Q}\, dq_d \,d e_q
}
{ (2\,\pi)^{d+1} }  \,
 \frac{ - \mathbf{K} \cdot \mathbf Q 
}
{ \mathbf Q^2  + \left(  e_q+ q_d^2/2  \right)^2
}  \,
\frac{1}{ q_d^2 } = 0 \,.
\end{align} 




The integral $I_1$ does not admit an exact closed form, and we proceed 
by identifying the regions that dominate the integrand. The first factor 
concentrates weight near $|\mathbf{Q}| \sim |\mathbf{K}|$ and 
$u \sim e_q$, while the second factor is dominated by 
$e_q \sim |\mathbf{Q}|^{2(d-1)/3} \sim |\mathbf{K}|^{2(d-1)/3}$. 
Since $|\mathbf{K}|^{2(d-1)/3} \gg |\mathbf{K}|$ for small $|\mathbf{K}|$ 
when $2(d-1)/3 < 1$, we may replace $u$ by $e_q$ in both the $\sqrt{u}$ 
factor in the overall denominator and the $2u$ term in the denominator 
of the second factor. Extending the lower limit of the $u$-integration 
to $-\infty$ then yields
%%%%%%%%%%%%%%%%%%%%%%%
\begin{align}
I_1 & \simeq \frac{ e^2 \,\mu^{x} } 
{ |\mathbf K|^2 }
\int_{-\infty}^\infty 
\frac{d^{d-1} \mathbf{Q}\, du} {(2\,\pi)^{d+1}}
  \int_{e_q>0} \frac{  d e_q}
{   \sqrt{ e_q/2 } }\,
 \frac{ \mathbf{K} \cdot (\mathbf K - {\mathbf{Q}} ) 
}
{(\mathbf Q- {\mathbf{K}} )^2 +  u^2
}  \,
\frac{1}
{ 2\,  e_q
+ e^2\, \mu^{x}\, b\,|\mathbf Q |^{d-1} /\sqrt{ e_q } }
% \quad \left [\text{shifting } u \rightarrow u  + e_q \right ] 
\nn
%%%%%%%%%%%%%%%%%%%%%%%
& =  \frac{ e^2 \,\mu^{x} } 
{ |\mathbf K|^2 }
\int_{-\infty}^\infty 
\frac{d^{d-1} \mathbf{Q}\, du} {(2\,\pi)^{d+1}}
  \int_{e_q>0} d e_q \,
 \frac{ \mathbf{K} \cdot (\mathbf K - {\mathbf{Q}} ) 
}
{(\mathbf Q- {\mathbf{K}} )^2 + u^2
}  \,
\frac{ \sqrt 2
}
{ 2\,   e_q^{3/2}
+  e^2\, \mu^{x}\, b \,|\mathbf Q |^{d-1} } \nn
%%%%%%%%%%%%%%%%%%%%%%%%%%%%%%%%%
& = -\frac{
e^{4/3} \, \Gamma \big (\frac{5-2 \,d}{6} \big ) \,
\Gamma \big (\frac{d}{2}\big ) \,
\Gamma \big (\frac{d+2}{6} \big ) 
} 
{ 2^{\frac{4 \,d-1} {6} } \pi^{\frac{d+1}{2}} \times 
3 \, \sqrt{3} \times 2^{2/3}
\, b^{1/3} \, \Gamma \big (\frac{5 \,d-2}{6} \big )}
 \left(  \frac{\mu} { |\mathbf K| } \right)^{\frac{2\, x} {3} }\,.
\end{align}
%%%%%%%%%%%
The integral diverges at $d = 5/2$, thereby confirming this as the 
value of the upper critical dimension, $d_c$. The fermion-boson coupling, 
$e$, is irrelevant for $d > d_c$ and relevant for $d < d_c$, and marginal 
precisely at $d = d_c$. This structure opens the door to a controlled 
perturbative treatment of the strongly interacting NFL state by working 
in $d = 5/2 - \epsilon$, where $\epsilon$ plays the role of a small 
expansion parameter. Within our dimensional regularization scheme, the 
divergence manifests as a $\sim \epsilon^{-1}$ pole, arising from the 
factor $\Gamma\!\left(\frac{5-2d}{6}\right)$ at $d = d_c$. We further 
note that this term reproduces the fermionic self-energy behavior 
$\sim \text{sgn}(k_0)|k_0|^{2/3}$ at $d = 2$, in agreement with the 
uncontrolled RPA result \cite{metzner1, metzner2}. It is worth 
emphasizing that this correct $k_0$-dependence of $\Sigma$ could only 
be captured by incorporating the Landau-damping term into the dressed 
bosonic propagator $D_{(1)}$ from the outset.





%%%%%%%%%%%%%%%%%%%%%%%%%%%%%%%%%%%%
\subsubsection{Computation of $\gamma_{d-1} $-dependent part}
%%%%%%%%%%%%%%%%%%%%%%%%%%%%%%%%%%%%%

The leading dependence of $\Sigma_2(k)$ on $k_d$ and $k_{d-1}$ is 
captured by setting $\mathbf{K} = 0$, reducing the problem to evaluating
\begin{align}
\label{eqFS 20}
\Sigma_2 (\mathbf 0, k_d , k_{d-1})  & =  e^2 \,\mu^{x} e^2 \,\mu^{x} 
\left[ I_3 (k_d , k_{d-1}) + I_4 (k_d , k_{d-1}) \right ] ,
\end{align}
%%%%%%%%%%%%%%%%%%5
where
$ \delta_{q-k}  =  q_{d-1} - k_{d-1} +k_d^2 + q_d^2  - 2 \,k_q\, q_d $
and
\begin{align}
& I_3 (k_d , k_{d-1})= \int_{q, \, e_q <0} \frac{ \delta_{q-k} }
{ |\mathbf Q|^2 + \delta^2_{q-k}}   \times \frac{1} 
{ q_d^2 + e^2 \, \mu^{x} \, b \,|\mathbf Q|^{d-1}  / \sqrt{| e_q|} } 
\text{ and}\nn &
I_4 (k_d , k_{d-1})= \int_{q, \, e_q > 0} \frac{ \delta_{q-k} }
{ |\mathbf Q|^2 + \delta^2_{q-k}}  \times \frac{1} {  q_d^2 } \,.
\end{align}
%%%%%%%%%%%%%%%%%%%%%%%%
To streamline the calculation, we set $k_d = 0$, whereupon
\begin{align}
I_3 ( 0 , k_{d-1}) & = \int_{q, \, e_q > 0} \frac{  q_d^2 /2  -e_q  - k_{d-1}   }
{ |\mathbf Q|^2 + \left( q_d^2 /2  - e_q - k_{d-1}  \right)^2 }  \times \frac{1} 
{  q_d^2  +  e^2 \, \mu^{x} \, b \,|\mathbf Q|^{d-1}  /\sqrt{e_q} } \nn
%%%%%%%%%%%%%%%%%%%%%%%%%%
& = \int_0^\infty \frac{ du} {\sqrt { u/2} }
 \int_0^\infty d e_q
  \int_{-\infty}^{\infty} \frac{d^{d-1} \mathbf{Q}}
{  (2\,\pi)^{d+1} }\, \frac{ u -e_q  - k_{d-1} }
{ |\mathbf Q|^2 
+   \left(  u -e_q  - k_{d-1}  \right)^2 }  \nn &
\hspace{1.5 cm} \times
\frac{1}   { 2\, u  + e^2\,\mu^{x}\, b \,|\mathbf Q |^{d-1} /\sqrt{ e_q} }  
\; \left (\text{where } 2\,u = {q_d^2} \right ) .
\end{align}
%%%%%%%%%%%%%%%%%%%%%%%%%%%%%%%%%%%%%%%%%%
The first factor of the integrand concentrates the dominant contribution 
near $|\mathbf{Q}| \sim 0$ and $u \sim e_q + k_{d-1}$, while the second 
factor is dominated by $e_q \sim |\mathbf{Q}|^{2(d-1)/3}$. We may 
therefore replace $u$ by $e_q + k_{d-1}$ in both the $\sqrt{u}$ factor 
in the overall denominator and the $2u$ term in the denominator of the 
second factor, and extend the lower limit of the $u$-integration to 
$-\infty$. This yields
%%%%%%%%%%%%%%%%%%%%%%%
\begin{align}
I_3 ( 0 , k_{d-1}) & \simeq \int_0^\infty \frac{ du} {\sqrt {e_q +  k_{d-1}} }
 \int_0^\infty d e_q
  \int_{-\infty}^{\infty} \frac{d^{d-1} \mathbf{Q}}
{  (2\,\pi)^{d+1} } 
\frac{ u -e_q  - k_{d-1} }
{ |\mathbf Q|^2 
+   \left(  u -e_q  - k_{d-1}  \right)^2}  
%%%%%%%%%%%%
\nn & \hspace{ 2 cm} \times \frac{ \sqrt 2}
{  2 \, e_q + 2\, k_{d-1} + e^2\,\mu^{x}\, b \,|\mathbf Q |^{d-1} /\sqrt{ e_q} }  = 0 \,.
\end{align}
To leading order, the integral $I_3$ therefore vanishes, leaving no 
$b$-dependent contribution. Proceeding to the next term, we have
%%%%%%%%%%%%%%%%%%%%%%%
\begin{align}
I_4 ( 0 , k_{d-1}) &= \int_{q, \, e_q > 0} \frac{  q_d^2 /2  + e_q  - k_{d-1}   }
{ |\mathbf Q|^2 + \left( q_d^2 /2 + e_q - k_{d-1}  \right)^2 }  \times \frac{1} {  q_d^2 }\nn & 
=
\int_0^\infty \frac{ du} {\sqrt { u/2} }
 \int_{-\infty}^0 d e_q \int_{-\infty}^{\infty} \frac{d^{d-1} \mathbf{Q}}
{  (2\,\pi)^{d+1} }\, \frac{ u + e_q  - k_{d-1} }
{ |\mathbf Q|^2 
+   \left(  u + e_q  - k_{d-1}  \right)^2 }  \,
\frac{1}   { 2\, u}  \,.
\end{align}
Applying the same reasoning as before, the first factor of the integrand 
concentrates the dominant contribution near $|\mathbf{Q}| \sim 0$ and 
$u \sim k_{d-1} - e_q$. Substituting $u \sim k_{d-1} - e_q$ into the 
$\sqrt{u}$ factor in the overall denominator and the $2u$ term in the 
denominator of the second factor, and extending the lower limit of the 
$u$-integration to $-\infty$, we find $I_4(0, k_{d-1}) = 0$. It follows 
that the FS flattening at the hot-spots, which appears in the RPA 
calculations of Ref.~\cite{metzner2}, does not emerge in our one-loop 
computation.



%%%%%%%%%%%%%%%%%%%%%%%%%%%%%%%%%%%%%
\subsection{One-loop vertex correction}
\label{secvert}

The one-loop vertex corrections [cf.\ Fig.~\ref{figcdw}(c)] are 
finite and therefore play no role in the renormalization procedure 
or the resulting RG flows.


%%%%%%%%%%%%%%%%%%%%%%%%%%%%%%%%%%%%%%%%%5
\section{RG flows under the minimal subtraction scheme}
\label{secrg}

The counterterm action, constructed to absorb the singular contributions, 
takes the form
\begin{align}
\label{actcount}
S_{CT} = & \int_k \bar{\Psi}(k)\,i\,\Bigl[A_1\,\vec{\Gamma}\cdot\vec{K}
+ \gamma_{d-1}\left(A_2\,e_k + A_3\,\frac{k_d^2}{2}\right)\Bigr]\Psi(k)
+ \int_k A_4\,k_d^2\,\phi_+(k)\,\phi_-(-k) \nonumber \\
& - \frac{i\,e\,\mu^{x/2}}{2}\int_{k,q} A_6\,
\Big[\phi_+(q)\,\bar{\Psi}(k+q)\,\gamma_0\,\bar{\Psi}^T(-k)
- \phi_-(-q)\,\Psi^T(q-k)\,\gamma_0\,\Psi(k)\Big]\,,
\end{align}
where $A_\zeta = \sum_{n=1}^\infty Z_\zeta^{(n)}/\epsilon^n$ with 
$\zeta \in [1,5]$. The $(d-1)$-dimensional rotational invariance in 
the space perpendicular to the FS ensures that every term in 
$\vec{\Gamma}\cdot\vec{K}$ is renormalized in the same way.

Subtracting $S_{CT}$ from the \textit{bare} action $S_{\text{bare}}$ 
yields the renormalized action, which constitutes the \textit{physical} 
effective action of the theory, expressed entirely in terms of 
non-divergent quantum parameters. While the bare parameters may be 
divergent, the physical observables are identified with the renormalized 
coupling constants, whose evolution is governed by the RG equations. 
These describe how the couplings flow as functions of the floating 
energy scale $\mu\,e^{-l}$, or equivalently, as the logarithmic length 
scale $l$ increases. To set this up, we first introduce the bare action,
%%%%%%%%%%%%%%%%%%%%%%%%
\begin{align}
\label{actren}
S_{\text{bare}}  = &   \int_{k^B} \bar{\Psi}^B (k^B)
\,\, i \left[ 
\,{\vec \Gamma} \cdot { \vec K^B} 
+   \gamma_{d-1} \left \lbrace  e_k^B + 
\frac{ \left( k^B\right)^2 }{2} \right \rbrace  \right ] \Psi^B (k^B)
%%%%%%%%%%%%%%%%%%%%%%%%%%%  
\nn &  + \int_{k^B}  \left (k^B_d \right)^2 \,
\phi_+^B (k^B) \,\, \phi_-^B (-k^B) 
 \nn & 
 %%%%%%%%%%%%%%%%
- \frac{ i\, e^B } {2} 
\int_{k^B,\, \, q^B}
\Big[ \,\phi^B_+(q^B) \,
 \bar \Psi^B  (k^B+q^B) \, \gamma_0 
 \left( {\bar \Psi}^B (-k^B) \right )^T  
\nn & \hspace{2.5 cm}
- \phi^B_-(-q^B) \left( {\Psi^B} (q^B-k^B) \right)^T  \gamma_0 \,
\Psi^B (k^B) \Big] \,,
\end{align}
%%%%%%%%%%%%%
comprising of the \textit{bare quantities}.
The superscript ``$B$'' labels bare fields, couplings, frequencies, and 
momenta throughout. The bare quantities are related to their renormalized 
counterparts (those without the superscript ``$B$'') through the 
multiplicative $Z_\zeta$-factors, via
\begin{align}
& S_{\text{bare}} = S + S_{CT}\,, \quad Z_\zeta = 1 + A_\zeta\,, \quad
\vec{K}^B = \frac{Z_1}{Z_3}\,\vec{K}\,, \quad
e_k^B = \frac{Z_2}{Z_3}\,e_k\,, \quad k_d^B = k_d\,, \nonumber \\
& \Psi^B(k^B) = Z_\Psi^{1/2}\,\Psi(k)\,, \quad
\phi_\pm^B(k^B) = Z_\phi^{1/2}\,\phi_\pm\,,
\end{align}
and
\begin{align}
& Z_\Psi = Z_1\left(\frac{Z_1}{Z_3}\right)^{-d}
\left(\frac{Z_2}{Z_3}\right)^{-1}\,, \quad
Z_\phi = Z_4\left(\frac{Z_1}{Z_3}\right)^{1-d}
\left(\frac{Z_2}{Z_3}\right)^{-1}\,, \nonumber \\
& e^B = Z_e\,e\,\mu^{\frac{\epsilon}{2}}\,, \quad
Z_e = \frac{Z_5\left(\frac{Z_1}{Z_3}\right)^{1-\frac{d}{2}}
\left(\frac{Z_2}{Z_3}\right)^{-1/2}}{\sqrt{Z_1}\,Z_4}\,.
\end{align}
There exists a freedom to rescale both fields and momenta without 
affecting the action, and we exploit this by fixing $k_d^B = k_d$, 
which amounts to measuring the scaling dimensions of all other 
quantities relative to that of $k_d$. The resulting $S$ is the 
renormalized --- or Wilsonian effective --- action, expressed entirely 
in terms of renormalized quantities. In essence, we have written the 
fundamental action of the theory in two equivalent ways, which allows 
the divergent contributions collected in $S_{CT}$ to be systematically 
subtracted off.


%%%%%%%%%%%%%%%%%%%%%%%%%%%%%%
\subsection{RG flow equations from the one-loop results}
%%%%%%%%%%%%%%%%%


At one-loop order, the divergent contributions are extracted from 
Eqs.~\eqref{eqpi} and \eqref{eqferm1}, and yield
\begin{align}
\label{eqZvals}
& Z_1 = 1 - \frac{e^{4/3}\,\mathcal{U}_1}{2^{2/3}\,\epsilon}\,, \quad
Z_2 = Z_3 = Z_4 = Z_5 = 1\,, \nonumber \\
& b = \frac{\pi^{3/4}}{32\sqrt{2}\,\Gamma^2\!\left(\tfrac{3}{4}\right)
\Gamma\!\left(\tfrac{7}{4}\right)}\,, \quad
\mathcal{U}_1 = \frac{\sqrt{2}\,\Gamma\!\left(\frac{5}{4}\right)}
{3\sqrt{3}\,\pi^{7/4}\,b^{1/3}}\,.
\end{align}
At this order, we find that $Z_2 = Z_3$, with neither receiving any 
correction from the loop integrals.

Since $Z_2 = Z_3$, a single dynamical critical exponent suffices for 
the fermions,
\begin{align}
z = 1 + \frac{\partial\ln\!\left(\frac{Z_1}{Z_2}\right)}{\partial\ln\mu}
= 1 + \frac{\partial\ln\!\left(\frac{Z_1}{Z_3}\right)}{\partial\ln\mu}\,,
\end{align}
reflecting the fact that the $\delta_k$ term, taken as a whole, is not 
renormalized at one-loop order. The anomalous dimensions of the fermions 
and bosons are defined by
\begin{align}
\eta_\psi = \frac{1}{2}\frac{\partial\ln Z_\psi}{\partial\ln\mu}
\quad \text{and} \quad
\eta_\phi = \frac{1}{2}\frac{\partial\ln Z_\phi}{\partial\ln\mu}\,,
\end{align}
respectively, and the beta function for $e$ is
\begin{align}
\beta_e = \frac{de}{d\ln\mu}\,.
\end{align}

The mass scale $\mu$ was introduced purely as a regularization device 
to render the loop-integrals finite. Since it is not a parameter of the 
fundamental theory, physical observables must be independent of it, and 
the same must hold for the bare parameters. Imposing this requirement, 
together with the condition that the non-singular parts of the solutions 
admit the small-$\epsilon$ expansions
\begin{align}
\label{eqexp}
z = z^{(0)}\,, \quad
\eta_\psi = \eta_\psi^{(0)} + \eta_\psi^{(1)}\,\epsilon\,, \quad
\eta_\phi = \eta_\phi^{(0)} + \eta_\phi^{(1)}\,\epsilon\,, \quad
\beta_e = \beta_e^{(0)} + \beta_e^{(1)}\,\epsilon\,,
\end{align}
in the limit $\epsilon \rightarrow 0$, one arrives at the differential 
equations,
\begin{align}
& z = 1 + \beta_e^{(1)}\,\frac{\partial Z_1^{(1)}}{\partial e}\,, \quad
\eta_\psi = \frac{1}{4}\left(5 - 5z + 
2\,\frac{\partial Z_1^{(1)}}{\partial e}\,\beta_e^{(1)}\right)
+ \frac{(z-1)\,\epsilon}{2}\,, \nonumber \\
& \eta_\phi = \frac{3-3z}{4} + \frac{(z-1)\,\epsilon}{2}\,, \quad
\frac{4\,\beta_e^{(0)}}{e} = -e\,z\,\frac{\partial Z_1^{(1)}}{\partial e}
+ z - 1\,, \quad
\beta_e^{(1)} = -\frac{e\,z}{2}\,.
\end{align}
These equations are derived by (1) imposing 
$\frac{d}{d\ln\mu}(\text{bare quantity}) = 0$; (2) substituting the 
expressions from Eqs.~\eqref{eqZvals} and \eqref{eqexp}; (3) expanding 
in powers of $\epsilon$; and (4) matching coefficients of regular powers 
of $\epsilon$ on both sides. Solving this system yields
\begin{align}
-\frac{\beta_e}{e} = \frac{6\,\epsilon - 3\times 2^{1/3}\,\mathcal{U}_1\,
\tilde{e}}{12 - 2^{7/3}\,\mathcal{U}_1\,\tilde{e}}\,, \quad
z = \frac{3}{3 - 2^{1/3}\,\mathcal{U}_1\,\tilde{e}}\,, \quad
\eta_\psi = \eta_\phi = \frac{(3-2\,\epsilon)\,\mathcal{U}_1\,\tilde{e}}
{4\,\mathcal{U}_1\,\tilde{e} - 6\times 2^{2/3}}\,,
\end{align}
where $\tilde{e} = e^{4/3}$. Since we are interested in the behavior at 
infrared energy scales, we track the RG flows with respect to the 
logarithmic length scale $l$, through the derivative
\begin{align}
\frac{de}{dl} \equiv -\beta_e
\end{align}
for the coupling constant $e$.



%%%%%%%%%%%%%%%%%%%%%%%%%%%%%%
\subsection{Stability of the fixed points of the RG flows}


The fixed points of the RG-flow differential equation are located where the beta-function, 
$\beta_e$, vanishes. They are readily found to be $e = 0$ and 
$\tilde{e} = 2^{2/3}\,\epsilon/\mathcal{U}_1$. To determine the 
stability of each fixed point, one examines whether the RG-flow lines 
in the IR, generated by $\{-\beta_e\}$, flow toward or away from 
it. This classifies the fixed point as stable or unstable, accordingly. 
For $\epsilon \in (0, 1/2]$, there is precisely one stable interacting 
fixed point for each value of $\epsilon$, located at 
$\tilde{e} = 2^{2/3}\,\epsilon/\mathcal{U}_1$, while $e = 0$ 
corresponds to an unstable Gaussian (non-interacting) fixed point. At 
the stable fixed point, the critical exponents evaluate to 
$z = 1 + 2\,\epsilon/3$ and $\eta_\phi = \eta_\psi = -\,\epsilon/2$.
%%%%%%%%%%%%%%%%%%%%%%%%%%%%%%%%%
\section{Conclusion}
\label{secsum-2kf}

This chapter has been devoted to a QFT analysis of the QCP that governs the continuous phase transition between a normal 
metallic state and a phase exhibiting incommensurate CDW order. A key 
ingredient throughout has been the use of a bosonic propagator 
augmented by the Landau-damping correction $\Pi_{\rm LD}$, which proves 
essential in generating the NFL fixed point. In particular, this dressed propagator is responsible 
for causing the frequency-dependence $ \sim \text{sgn}(k_0)|k_0|^{2/3}$ appearing in the 
fermionic self-energy, that is a defining characteristic of an NFL behaviour 
\cite{Lee-Dalid, ips-uv-ir1, ips-uv-ir2, ips-fflo, ips-u1, ips-rafael, peng}.

%\bibliographystyle{spphys.bst}
%\bibliography{ref.bib}

\begin{thebibliography}{10}
\providecommand{\url}[1]{{#1}}
\providecommand{\urlprefix}{URL }
\expandafter\ifx\csname urlstyle\endcsname\relax
  \providecommand{\doi}[1]{DOI \discretionary{}{}{}#1}\else
  \providecommand{\doi}{DOI \discretionary{}{}{}\begingroup
  \urlstyle{rm}\Url}\fi

\bibitem{max-isn}
M.A. Metlitski, S.~Sachdev, Phys. Rev. B \textbf{82}, 075127 (2010).
\newblock \doi{10.1103/PhysRevB.82.075127}

\bibitem{Lee-Dalid}
D.~Dalidovich, S.S. Lee, Phys. Rev. B \textbf{88}, 245106 (2013).
\newblock \doi{10.1103/PhysRevB.88.245106}

\bibitem{ips-uv-ir1}
I.~Mandal, S.S. Lee, Phys. Rev. B \textbf{92}, 035141 (2015).
\newblock \doi{10.1103/PhysRevB.92.035141}

\bibitem{ips-uv-ir2}
I.~Mandal, Eur. Phys. J. B \textbf{89}(12), 278 (2016).
\newblock \doi{10.1140/epjb/e2016-70509-4}

\bibitem{ips-sc}
I.~Mandal, Phys. Rev. B \textbf{94}, 115138 (2016).
\newblock \doi{10.1103/PhysRevB.94.115138}

\bibitem{chubukov1}
A.~{Abanov}, A.~{Chubukov}, Physical Review Letters \textbf{93}(25), 255702
  (2004).
\newblock \doi{10.1103/PhysRevLett.93.255702}

\bibitem{Chubukov}
A.~{Abanov}, A.V. {Chubukov}, Physical Review Letters \textbf{84}, 5608 (2000).
\newblock \doi{10.1103/PhysRevLett.84.5608}

\bibitem{shouvik2}
S.~Sur, S.S. Lee, Phys. Rev. B \textbf{91}, 125136 (2015).
\newblock \doi{10.1103/PhysRevB.91.125136}

\bibitem{ips-c2}
I.~Mandal, Annals of Physics \textbf{376}, 89  (2017).
\newblock \doi{https://doi.org/10.1016/j.aop.2016.11.009}.
\newblock
  \urlprefix\url{http://www.sciencedirect.com/science/article/pii/S0003491616302585}

\bibitem{andres1}
A.~Schlief, P.~Lunts, S.S. Lee, Phys. Rev. X \textbf{7}, 021010 (2017).
\newblock \doi{10.1103/PhysRevX.7.021010}.
\newblock \urlprefix\url{https://link.aps.org/doi/10.1103/PhysRevX.7.021010}

\bibitem{andres2}
P.~Lunts, A.~Schlief, S.S. Lee, Phys. Rev. B \textbf{95}, 245109 (2017).
\newblock \doi{10.1103/PhysRevB.95.245109}.
\newblock \urlprefix\url{https://link.aps.org/doi/10.1103/PhysRevB.95.245109}

\bibitem{ips-fflo}
D.~Pimenov, I.~Mandal, F.~Piazza, M.~Punk, Phys. Rev. B \textbf{98}, 024510
  (2018).
\newblock \doi{10.1103/PhysRevB.98.024510}

\bibitem{max-sdw}
M.A. Metlitski, S.~Sachdev, Phys. Rev. B \textbf{82}, 075128 (2010).
\newblock \doi{10.1103/PhysRevB.82.075128}

\bibitem{hubbard}
H.J. Schulz, Phys. Rev. Lett. \textbf{64}, 1445 (1990).
\newblock \doi{10.1103/PhysRevLett.64.1445}.
\newblock \urlprefix\url{https://link.aps.org/doi/10.1103/PhysRevLett.64.1445}

\bibitem{hubbard2}
P.A. Igoshev, M.A. Timirgazin, A.A. Katanin, A.K. Arzhnikov, V.Y. Irkhin, Phys.
  Rev. B \textbf{81}, 094407 (2010).
\newblock \doi{10.1103/PhysRevB.81.094407}.
\newblock \urlprefix\url{https://link.aps.org/doi/10.1103/PhysRevB.81.094407}

\bibitem{bond-order}
S.~Sachdev, R.~La~Placa, Phys. Rev. Lett. \textbf{111}, 027202 (2013).
\newblock \doi{10.1103/PhysRevLett.111.027202}.
\newblock
  \urlprefix\url{https://link.aps.org/doi/10.1103/PhysRevLett.111.027202}

\bibitem{salvo}
F.D.S. J.A.~Wilson, S.~Mahajan, Advances in Physics \textbf{24}(2), 117 (1975).
\newblock \doi{10.1080/00018737500101391}.
\newblock \urlprefix\url{https://doi.org/10.1080/00018737500101391}

\bibitem{Scholz}
G.~Scholz, O.~Singh, R.~Frindt, A.~Curzon, Solid State Communications
  \textbf{44}(10), 1455 (1982).
\newblock \doi{https://doi.org/10.1016/0038-1098(82)90030-8}.
\newblock
  \urlprefix\url{https://www.sciencedirect.com/science/article/pii/0038109882900308}

\bibitem{pai}
P.~Chen, W.W. Pai, Y.H. Chan, V.~Madhavan, M.Y. Chou, S.K. Mo, A.V. Fedorov,
  T.C. Chiang, Phys. Rev. Lett. \textbf{121}, 196402 (2018).
\newblock \doi{10.1103/PhysRevLett.121.196402}.
\newblock
  \urlprefix\url{https://link.aps.org/doi/10.1103/PhysRevLett.121.196402}

\bibitem{Gweon}
G.H. Gweon, J.D. Denlinger, J.A. Clack, J.W. Allen, C.G. Olson, E.~DiMasi, M.C.
  Aronson, B.~Foran, S.~Lee, Phys. Rev. Lett. \textbf{81}, 886 (1998).
\newblock \doi{10.1103/PhysRevLett.81.886}.
\newblock \urlprefix\url{https://link.aps.org/doi/10.1103/PhysRevLett.81.886}

\bibitem{Kapitulnik}
A.~Fang, N.~Ru, I.R. Fisher, A.~Kapitulnik, Phys. Rev. Lett. \textbf{99},
  046401 (2007).
\newblock \doi{10.1103/PhysRevLett.99.046401}.
\newblock
  \urlprefix\url{https://link.aps.org/doi/10.1103/PhysRevLett.99.046401}

\bibitem{littlewood}
Y.~{Feng}, J.~{van Wezel}, J.~{Wang}, F.~{Flicker}, D.M. {Silevitch}, P.B.
  {Littlewood}, T.F. {Rosenbaum}, Nature Physics \textbf{11}(10), 865 (2015).
\newblock \doi{10.1038/nphys3416}

\bibitem{metzner1}
T.~Holder, W.~Metzner, Phys. Rev. B \textbf{90}, 161106 (2014).
\newblock \doi{10.1103/PhysRevB.90.161106}.
\newblock \urlprefix\url{https://link.aps.org/doi/10.1103/PhysRevB.90.161106}

\bibitem{metzner2}
J.~S\'ykora, T.~Holder, W.~Metzner, Phys. Rev. B \textbf{97}, 155159 (2018).
\newblock \doi{10.1103/PhysRevB.97.155159}.
\newblock \urlprefix\url{https://link.aps.org/doi/10.1103/PhysRevB.97.155159}

\bibitem{senshank}
T.~Senthil, R.~Shankar, Phys. Rev. Lett. \textbf{102}, 046406 (2009).
\newblock \doi{10.1103/PhysRevLett.102.046406}

\bibitem{ips-u1}
I.~Mandal, Phys. Rev. Research \textbf{2}, 043277 (2020).
\newblock \doi{10.1103/PhysRevResearch.2.043277}.
\newblock
  \urlprefix\url{https://link.aps.org/doi/10.1103/PhysRevResearch.2.043277}

\bibitem{ips-rafael}
I.~Mandal, R.M. Fernandes, Phys. Rev. B \textbf{107}, 125142 (2023).
\newblock \doi{10.1103/PhysRevB.107.125142}.
\newblock \urlprefix\url{https://link.aps.org/doi/10.1103/PhysRevB.107.125142}

\bibitem{peng}
P.~Rao, F.~Piazza, Phys. Rev. Lett. \textbf{130}, 083603 (2023).
\newblock \doi{10.1103/PhysRevLett.130.083603}.
\newblock
  \urlprefix\url{https://link.aps.org/doi/10.1103/PhysRevLett.130.083603}

\end{thebibliography}